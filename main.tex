
\documentclass[11pt,a4paper,oldfontcommands]{memoir}

\usepackage[utf8]{inputenc}
\usepackage[T1]{fontenc}
\usepackage{microtype}
%\usepackage[dvips]{graphicx}
\usepackage{xcolor}
\usepackage{times}
\usepackage[version=3]{mhchem} % Formula subscripts using \ce{}
\usepackage [english]{babel}
\usepackage [autostyle, english = american]{csquotes}
\usepackage[demo]{graphicx}
\usepackage{caption}
\usepackage{subcaption}
\MakeOuterQuote{"}
\newcommand*{\mycommand}[1]{\texttt{\emph{#1}}}
\usepackage[demo]{graphicx}
\usepackage{adjustbox}
\let\Tiny\tiny
\usepackage{lipsum}
\newcommand\Fontvi{\fontsize{7}{7}\selectfont}
\usepackage{xcolor}
\usepackage{graphicx}
\usepackage{subcaption}
\usepackage{mwe}
\usepackage{float}
\usepackage{longtable}
\usepackage{setspace}

%%%%

% Palatino font options
\usepackage{mathpazo}                   % Palatino in LaTeX math
\usepackage{tgpagella}                  % Palatino font
%\newcommand{\ltfamily}{\familydefault}  % Makes sure LaTeX keeps using Palatino in Lithuanian 
%\addtokomafont{disposition}{\rmfamily}  % Palatino for titles etc.
%\setkomafont{descriptionlabel}{         % font for description lists    
%\usekomafont{captionlabel}\bfseries     % Palatino bold
%}
%\setkomafont{caption}{\footnotesize}    % smaller font size for captions
% To make sure LaTeX keeps using Palatino in Lithuanian:
% in OVERLEAF change the following \newcommand to \renewcommand
% \newcommand{\ltfamily}{\familydefault} 

\usepackage{mathabx}                    % allows for nicer looking \cup, \curvearrowbotright, etc. !!IMPORTANT!! These are math symbols and should be surrounded by $dollar signs$
\usepackage[normalem]{ulem}                       % allows for strikethrough with \sout etc.

\usepackage{titling}
%%%%%%

\usepackage[
breaklinks=true,colorlinks=true,
linkcolor=blue,urlcolor=blue,citecolor=blue,% PDF VIEW
linkcolor=black,urlcolor=black,citecolor=black,% PRINT
bookmarks=true,bookmarksopenlevel=2]{hyperref}

\usepackage{geometry}
% PDF VIEW
% \geometry{total={210mm,297mm},
% left=25mm,right=25mm,%
% bindingoffset=0mm, top=25mm,bottom=25mm}
% PRINT
\geometry{total={210mm,297mm},
left=20mm,right=20mm,
bindingoffset=0mm, top=25mm,bottom=25mm}

%\OnehalfSpacing
\linespread{2}



%%% CHAPTER'S STYLE


\usepackage{color,graphicx} 
\definecolor{ared}{rgb}{.647,.129,.149} 
\renewcommand\colorchapnum{\color{ared}} 
\renewcommand\colorchaptitle{\color{ared}} 
\chapterstyle{pedersen}



%\chapterstyle{dash}
%\chapterstyle{lyhne}
%\chapterstyle{bianchi}
%\chapterstyle{ger}
%\chapterstyle{madsen}
%\chapterstyle{ell}
%%% STYLE OF SECTIONS, SUBSECTIONS, AND SUBSUBSECTIONS
\setsecheadstyle{\Large\bfseries\sffamily\raggedright}
\setsubsecheadstyle{\large\bfseries\sffamily\raggedright}
\setsubsubsecheadstyle{\bfseries\sffamily\raggedright}





%%% STYLE OF PAGES NUMBERING
%\pagestyle{companion}\nouppercaseheads 
%\pagestyle{headings}
%\pagestyle{Ruled}
\pagestyle{plain}
\makepagestyle{plain}
\makeevenfoot{plain}{}{\thepage}{}
\makeoddfoot{plain}{}{\thepage}{}




\maxsecnumdepth{subsection} % chapters, sections, and subsections are numbered
\maxtocdepth{subsection} % chapters, sections, and subsections are in the Table of Contents


%\doublespacing

\begin{document}


\date{}
\begin{titlepage}
\thispagestyle{empty}

\begin{center}

%\pagenumbering{gobble}

\begin{vplace}[1.2]


\LARGE \textsc{On Volatility, Outliers, and Uncertainty}

\large \textsc{By} \\ \textsc{Chandler Clemons} \vskip 1em 
%\large \textsc{Associate Editor} \\ \associateeditor

\end{vplace}
\vfill
Claremont Graduate University \\
2021
\end{center}
\end{titlepage}

\clearpage


\thispagestyle{empty}

{%%%
\sffamily
\centering


~\vspace{\fill}



\begin{figure}[H]
\includegraphics[width=0.5\textwidth]{cgulogo.png} 
\centering
\label{}
\end{figure}



$\copyright$ Copyright Chandler Clemons, 2021. \\
All rights reserved 

%%%
}%%%

\cleardoublepage
%%%---%%%---%%%---%%%---%%%---%%%---%%%---%%%---%%%---%%%---%%%---%%%---%%%
%%%---%%%---%%%---%%%---%%%---%%%---%%%---%%%---%%%---%%%---%%%---%%%---%%%

\thispagestyle{empty}

\begin{center}
\LARGE \textsc{Approval of the Dissertation Committee}    

%\linespread{1}
\normalsize
\hfill \break
This dissertation has been duly read, reviewed, and critiqued by the Committee listed below, which hereby approves the manuscript of Chandler Clemons as fulfilling the scope and quality requirements for meriting the degree of Doctor of Philosophy in Economics.  
\hfill \break
\hfill \break

Dr. Pierangelo De Pace, Chair \\
Pomona College \\
Associate Professor of Economics

\hfill \break
\hfill \break

Dr. Monica Capra \\
Claremont Graduate University \\
Professor of Economic Sciences

\hfill \break
\hfill \break

Dr. Thomas J. Kneisner \\
Claremont Graduate University \\
Professor of Economic Sciences




\end{center}

\clearpage

\thispagestyle{empty}

\begin{center}
\LARGE \textsc{Dedication}    

%\linespread{1}
\normalsize
\hfill \break
Dedicated to a mother and father of endless support, encouragement, love. 





\end{center}

\clearpage

\pagenumbering{roman}
\setcounter{page}{5}

\begin{center}
   

\LARGE \textsc{Acknowledgment}    

\end{center}
%\linespread{1}
\normalsize
\hfill \break
This work would not have been possible without, first and foremost, my co-author, research partner, and friend, Minh Pham. My committee, Dr. Pierangelo De Pace, Dr. Monica Capra, and Dr. Thomas Kniesner, also deserve special thanks for their input, feedback, and guidance during this process. I would also like to thank the professors and mentors at CGU, especially Dr. Hisam Sabouni and Dr. Rutledge who introduced me to financial economics, non-linear systems, complexity, and inspired me to explore the topics and methodology addressed in this work.  







\end{center}

\clearpage


\tableofcontents*

\chapter*{Preface}
\addcontentsline{toc}{chapter}{Preface}

\begin{center}



\textsc{On Volatility, Outliers, and Uncertainty}

\textsc{By}

\textsc{Chandler Clemons}

\textsc{Claremont Graduate University: 2021}

\end{center}

\linespread{2}

This work was inspired by a passion and desire to understand the economic crises events that redirect, transmute, and ultimately redefine our socioeconomic lives, as individuals and as nations. I began my economic studies during one of the deepest and most profound crises in recent history, the global financial crisis that began in subprime mortgages and quickly exposed a global economy in peril. Like the straw that break the camel’s back, we never really know when, why, or from where that final straw arrives. But it does, and it falls upon us more frequently than we like but also more that we seem to predict. If these events are so monumental and seemingly assured during our lifetimes, why do they always seem to catch up off guard? Each of the chapters presented in this dissertation seek to understand the fundamental properties of crises, but more specifically I explore the ways in which we interact with uncertainty. Uncertainty is the expression of a crisis, but also. Or is a crisis an expression of uncertainty? Perhaps causality flows in both directions, and that is the point. Economic systems are dynamic and complex, where everything affects everything else, including our perceptions. For financial economist, uncertainty is akin to volatility, the dispersion of asset price returns. Each of the following chapters 










\chapter{A Regime Dependent Relationship - VIX and SP500  \\ \small By Chandler Clemons}
\pagenumbering{arabic}
\setcounter{page}{1}

\begin{abstract}
This paper examines whether expectations are formed in a systematically different manner during periods of low volatility versus periods of high volatility. I achieve this by measuring non-linearities in relationship between the S&P 500 and the VIX across different market regimes. Three distinct market regimes are identified through a Markov Process (Hamilton 1989), allowing for the capture of non-constant behavior in the relationship between contemporaneous price changes and future volatility expectations. The results indicate that the effect of the underlying asset on the supply and demand dynamics of its derivative is strongest during periods of low volatility and weakest during periods of high volatility. The decrease in magnitude of the S&P 500 coefficient as the market switches from low volatility to high, suggests that information scarcity (low volatility) makes additional data (price changes) more impactful. Measures to limit market volatility may make market participant prone to expect changes in the state of the system.




\end{abstract}
\clearpage

\section{Introduction}

\makeevenhead{plain}{Chapter 1}{}{\textit{Clemons}}
\makeoddhead{plain}{\textit{Clemons}}{}{Chapter 1}

Financial derivative products form a marketplace that is unique in its complexity, velocity, and exposure. These instruments are financial securities with a value that is derived from an underlying asset or group of assets and make up a global marketplace estimated to have a contract value of \$15.5 trillion and a notional value, or exposure, \$606.8 trillion in June 2020.\footnote{Bank for International Settlements} To put this in perspective, the entire U.S. Equity Market Value is estimated at approximately \$35.5 trillion in June 2020; understanding the derivatives market is of significant financial interests for practitioners and regulators, and taxpayers.\footnote{The massive exposure that derivatives markets equals taxpayer exposure in a too-big-to-fail economy (e.g. AIG bailout).} For this study, the derivative products of interests are equity options with values derived from changes in the price of an underlying publicly traded equity. This can be achieved at an aggregate level by utilizing market indices. For example the VIX index, which aggregates near and next term out-of-the-money calls and puts across the S\&P 500, can be thought of as a derivative of the S\&P 500 index. Each of these aggregated calls and puts has a value that is derived from an individual equity within the S\&P 500. For example, a Facebook call option has value relative to the current market price of Facebook equity. The VIX aggregates these calls and puts in a way such that the index represents the \textit{implied volatility} that results from the clearing prices of all S\&P 500 options.\footnote{Not quite \textit{all} S\&P 500 options are considered. Some deep out-of-the-money calls and puts may not be considered if they are proceeded by two consecutive null bids for strikes above (puts) or below (calls). See the VIX White paper for additional information.} Implied volatility in this sense is the amount of expected volatility required to set the options' expected value equal to zero, given the contracted prices. It is therefore of interest to assess how changes in the value of the VIX index (expected volatility) varies with changes in the price of the S\&P 500. One would expect changes in the price of the underlying asset to correlate with the expected value of its derivative product, since after all the value of a derivative depends on the future price movements of the underlying asset. Understanding the fundamental structures of this relationship should therefore be of value. 

The velocity of derivative markets makes their study relevant and useful from a microeconomic perspective. Contract terms, at least for the derivatives analyzed in this study, have short-term expirations that determine both the risk and value of the contract. In essence, the market for short-term equity options is a repeated and natural decision making under uncertainty experiment. Each trader, faced with a number of possible actions, makes a risk-adjusted choice under uncertainty. In the aggregate, these choices determine the supply and demand dynamics of the options market. In theory, the rational procedure is to identify all possible outcomes, determine their values (positive or negative) and the probabilities that will result from each course of action, and multiply the two to give an "expected value", or the average expectation for an outcome; the action to be chosen should be the one that gives rise to the highest total expected value. A typical trader performs this procedure numerous times across years of activity. Further, these market participants are generally sophisticated agents due to the complexity of products and transactions.\footnote{Sophisticated in that they are typically not noise traders who make decisions without the support of professional advice or advanced fundamental or technical analysis.} Aggregating each individual buy-sell action offers a rather robust dataset of short-term expectations. Using this understanding of options-implied volatility, this paper aims to achieve a parsimonious assessment of the dynamic properties of short-term volatility expectation formation. 


Born out of both academic and financial interests, a wide field of research has been devoted to uncovering the characteristics of this relationship. Most of this work is centered around how to precisely and efficiently price a derivative product, with the most famous and influential being the Black-Scholes Merton pricing formula; the price of an equity option is assumed to be a function of a constant volatility of the underlying stock, the time value of money, the option's strike price, and the time to the option's expiry (Black, Scholes 1973). Subsequent literature has added additional levels of complexity to the original linear Black-Scholes Merton approach, such as considerations for stochastic volatility (Wiggins 1987; Ghysels, Harvey, Renault 1996), non-continuous price jumps (Merton 1976; Kou 2002), counterparty risk (Klein 1996; Burgard, Kjaer 2011), illiquidity (Feng 2011), trasnaction costs (Barles, Soner 1998; P. Amster, C.G. Averbuj, Mariani, Rial 2005), and implied volatility skews (Skiadopoulos et al 2000; Cont and da Fonseca, 2002; Fengler, Hardle, and Villa, 2003; Benko, Hardlee, and Kneip, 2009; Bernales and Guidolin, 2015).


This study, however, takes a different approach. Rather than seeking to estimate an equilibrium price through partial differential equations, this analysis focuses on how an aggregation of contracted option prices (i.e. the VIX) varies with contemporaneous price changes in the underlying asset. It is purely a correlative assessment, not a predictive model. Principally, the methodology applied in the paper will capture how this relationship varies depending on the current state of the market. That is, a three-state Markov regime-switching model captures non-linearities in the supply and demand dynamics of equity options between periods of stability and instability. The three-state Markov chain governs the probabilistic model driving the change between regimes, such that the contemporaneous state, $s_t$, is inferred from observed behavior of $VIX_t$. The probability law governing $VIX_t$ is described by the variance of the Gaussian innovation $\sigma^2$, the autoregressive coefficients ($VIX_{t-1, \dots,4$), the intercept, $S\&P500_t$, and the transition probabilities, $\omega_{1}$, $\omega_{2}$, and $\omega_{3}$ (Hamilton 2005). If market participants always employed a formulaic approach to options pricing, such as the Black-Scholes Merton model, one would find a time-invariant structure between an option and its underlying, producing coefficients for $S\&P500_t$ that are not statistically different between each regime. That is, if time to expiration, historic volatility, interest rates, and strike price are properly accounted for, contemporaneous changes in an asset's price would correspond to same increase or decrease in the price of the derivative across all market regimes; all other variables are latent in the pricing formula. Therefore, capturing non-linearities leads to insights into how market participants actually form their expectations, which may rely more on complex heuristics than complex and assumption-sensitive formulae (Haug \& Taleb 2010). Markets are complex, non-constant, and non-linear systems (Mandelbrot 1963; Lux 1996; Liu, et al. 1999; Guillaume, et al 1997; Gopikrishnan, Plerou, Gabaix, Stanley 2000). We shouldn’t expect “expectation formation” to be time-invariant and linear either.


This paper is an effort to provide statistical evidence that implied volatility dynamics vary depending on the current state of the market: low volatility regime with jumpy expectations, normal regime with low to moderate volatility, and crisis regime with extremely high volatility. Furthermore, the results show that implied volatility’s response to changes in the S\&P 500 decreases as volatility increases. That is, in states of the world when volatility is already high, the impact of additional price variation on volatility expectations diminishes; or when information is scarce (i.e. low volatility), additional data is more consequential. There are a few primary implication that result for the results presented in this paper. For monetary policy, the first suggestion may be counterintuitive; measures to limit market volatility by over-managing and over-intervening may increase the system's instability and susceptibility to large market corrections. When traders are not accustomed to volatility, the slightest price variation will be attributed to insider information, or to changes in the state of the system, and will cause panics.\footnote{Taleb 2019} An active area of research on stochastic resonance has shown that one way to amplify a signal normally too weak to be detected is to add white noise to the signal. This phenomenon has been shown to exist in self-organizing and complex systems. Such as it may be with biological evolution, financial and economic systems may benefit from some volatility and may be harmed by the limiting of a natural selection process. The second implication is not new to the literature but is worth reinforcing. Linear derivative models, such as Black-Scholes and its stochastic variants, are not informative of real world behavior. Nor can they be trusted to reliably characterize equilibrium prices across \textit{all} states of the market. While derivatives pricing has been, and continues to be, an active area of research, this is the first study to explore the connection between a derivative and its underlying asset using a regime switching approach to my knowledge.

The rest of the paper is organized as follows. Section 2 reviews the formation of the CBOE VIX Index, standard options pricing models, the concept of implied volatility, and the implied volatility surface. Section 3 introduces the data and presents the Markov Switching method used for regime identification. Section 4 studies the dynamics of the relationship between implied volatility and price changes in the underlying. Multi-regime regression results presented and discussed in Sections 4.1 and 4.2. Section 4.3 addresses tail index estimation for the residuals of each regime-conditional model across the entire data series. Understanding differences in the tail behavior of the conditional and linear models helps quantify the significance of a multi-regime approach. Finally, the concluding remarks are presented in Section 5. 




%In this paper, we examine non-linearities in the supply-demand mechanics for out-of-the-money calls and puts across the S\&P 500. Specifically, we assess the relationship between options-implied volatility and contemporaneous price changes in the underlying security or index across various states of the market. Differences in parameter estimates, model fit, and tail indices between regimes offer insight into how volatility expectations are formed in a manner that is dependent on market conditions. Our research is among the first analyses to explore the non-linear and dynamic behavior of the relationship between implied volatility and price changes in the underlying using a Markov switching model of Hamilton (1989). The Markov approach uses multiple structures (equations) that characterize the time series behaviors in different regimes, capturing more complex dynamic patterns. Using this approach, our paper aims to provide an indirect, yet parsimonious, assessment of previously explored topics in the derivatives literature, with particular attention on the supply and demand dynamics of out-of-the-money options and the implied volatility surface. We aim to provide further evidence that market relationships are not time invariant. Understanding the role of non-linearities can be useful to both practitioners and academics alike.  



%In a related study, Low (2004) uses a linear regression to investigate the contemporaneous relation between changes in the VIX and the S\&P 100 returns (the VIX was calculated from S\&P 100 options at the time of publication). The results suggest an asymmetric response depending on whether the contemporaneous return was positive or negative. This implies that risk perceptions tend to increase more when downside volatility increases relative to upside volatility, and that prior gains appear to have some mitigating effect on volatility expectations. 

%Prior literature has also explored the VIX under a switching framework. Baba and Sakurai (2011) use a regime switching approach to investigate the role of US macroeconomic variables as leading indicators of regime shifts in the VIX index. They find that there are three distinct regimes in the VIX index during the 1990 to 2010 period: tranquil regime with low volatility, turmoil regime with high volatility and crisis regime with extremely high volatility. They also show that the regime shift from the tranquil to the turmoil regime is significantly predicted by lower term spreads. In this paper, we consider only the dynamics of the impulse-response relationship between implied volatility and daily prices changes in the underlying across different regimes. However, we follow Baba and Sakurai (2011) and employ a three-regime model when exploring the dynamics of the VIX, albeit with a slightly different interpretations of regime states.

%The majority of prior literature on this topic, however, is aimed at assessing the supply and demand for out-of-the-money calls and puts through measurements of the implied volatility (IV) smile or smirk. The IV smile comes from the empirical observation that the clearing price for out-of-the-money calls and puts is typically higher than that which is predicted by the Black-Scholes Options Pricing Model. That is, as the strike price moves further from the current price, the volatility implied by the option’s price increases at an increasing rate. While this paper does not directly measure the shape of the IV surface, we do so indirectly. Our results indicate that future work exploring changes in slope of the smile across different market regimes may provide useful insight into how the market anticipates outlier events. 




\section{Theoretical background}

\subsection{Implied Volatility and the Supply and Demand of OTM Options}

The Black-Scholes pricing model, introduced in 1973, has been an area of significant interest for both academic and commercial pursuits. Its exact predictions provide a benchmark to which empirical data can be compared. Similarly, for practitioners, Black-Scholes pricing sets a theoretical level from which market participants can base their purchase and sell decisions. Divergences between the theoretical predictions and the realized values have thus become an active area of research. An additional area of research emerged by inverting the Black-Scholes model to assess the implied volatility for various strike prices, K, and various times to maturity, t, with the same underlying security. The concept of implied volatility comes from the idea that, given a contract price, an options pricing formula can be inverted to calculate the amount of volatility required to set the option's expected value at zero. Thus, each price has an associated volatility expectation, which varies depending on the pricing model employed. Using this definition, implied volatility measures are model dependent. Calculating implied volatility is useful for several reasons, one of which is interpretability. The methodology normalizes the price quote for a given option over different strike prices, time horizons, and underlying security prices. This concept, however, requires a probability density function of the underlying return series, which must be assumed, estimated, or simulated, and is therefore not a market driven index but rather a theoretical concept. For this paper, we are concerned with market-implied, not model-implied expectations. 

In 1993 Cboe Global Markets, Incorporated (Cboe) introduced the Cboe Volatility Index, commonly known as the VIX Index. The VIX index was originally designed to measure the market’s expectation of 30-day volatility implied by at-the-money S\&P 100 Index (OEX Index) option prices. The VIX Index quickly became the premier benchmark for U.S. stock market volatility. In 2003, the VIX was updated by Cboe together with Goldman Sachs to reflect a new way to measure expected volatility. Taken from the S\&P 500 Index (SPXSM). the new VIX algorithm estimates expected volatility by aggregating the weighted prices of SPX puts and calls over a wide range of strike prices. Because the algorithm captures and aggregates actual market prices, the index does not rely on an inversion of a pricing model to derive expected volatility. Instead, the VIX calculation measures 30-day expected volatility of the S\&P 500 Index using near and next-term put and call options with more than 23 days and less than 37 days to expiration using the following generalized formula (see VIX White paper for a full explanation of formula):
\[\sigma^2 =\frac{2}{T}\sum_i{\frac{\Delta{K_i}}{K^2_i}e^{RT}Q(K_i) -\frac{1}{T}[\frac{F}{K_0}-1]^2}\]
Where: 
\begin{flushleft}
$\sigma = \frac{VIX}{100}$\\
$T = Time\ to\ expiration$\\
$F = Forward\ index\ level\ desired\ from\ index\ option\ prices$\\
$K_0 = First\ strike\ below\ the\ forward\ index\ level, F$\\
$K_i= Strike\ price\ of\ the\ ith\ out-of-the-money\ option; a\ call\ if\ K_i>K_0; and\ a\ put\ if\ K_i<K_0; both\ put\ and\ call\ if\ K_i=K_0.$\\
$\Delta{K_i} = Interval\ between\ strike\ prices\ - half\ the\ difference\ between\ the\ strike\ on\ either\ side\ of\ K_i: \Delta{K_i} = \frac{K_{i+1}-K_{i-1}}{2}$\\
$R = Risk-free\ interest\ rate\ to\ expiration$\\
$Q(K_i) = The\ midpoint\ of\ the\ bid-ask\ spread\ for\ each\ option\ with\ strike\ K_i$\\
\end{flushleft}
The contribution of a single option to the VIX Index value is proportional to $\Delta K$ and the price of that option, and inversely proportional to the square of the option’s strike price, $\sqrt{K_i}$. The key relevance of this index lies in its ability to approximate market-implied volatility using the clearing prices of a multitude of options contracts. The index does not rely on assumptions about the distribution of returns, the stochastic nature of volatility, and other assumptions typically worked into derivative models that generate implied volatility statistics. Instead, the VIX provides an excellent, even if noisy, sentiment index - model-free and options-implied volatility.

A long literature has been devoted to assessing the predictive power of the VIX index with varied results; see Bekaert and Hoerova (2014), Sarwar (2012). As simple Granger Causality test indicates a bi-directional relationship between the VIX and the S\&P 500, and the majority existing literature does not support the use of the VIX as a consistent predictor of future volatility, especially for practitioners seeking profit generating signals.\footnote{SPY granger causes VIX with F statistic = 0.012; VIX granger causes SPY with F statistic = 0.03} In contrast, this paper seeks to enhance the understanding of how volatility expectations are formed by contributing to the branch of research surrounding the dynamics of implied volatility, its surface, and its relationship to the underlying asset(s). 

In a related study, Low (2004) uses a linear regression to investigate the contemporaneous relation between changes in the VIX and the S\&P 100 returns (the VIX was calculated from S\&P 100 options at the time of publication). The results suggest an asymmetric response depending on whether the contemporaneous return was positive or negative. This implies that risk perceptions tend to increase more when downside volatility increases relative to upside volatility, and that prior gains appear to have some mitigating effect on volatility expectations. The majority of prior literature on this topic, however, is aimed at assessing the supply and demand for out-of-the-money calls and puts through measurements of the implied volatility (IV) smile or smirk. 


\subsection{The Implied Volatility Surface}

The Black-Scholes model assumes that implied volatility is constant with respect to strike prices and time to maturity. However, this theoretical prediction is not typically found in practice. Divergences between the Black-Scholes implied volatility and market implied volatility were found in Rubinstein (1994). When plotted against various strike prices while holding time to expiration constant, Black-Scholes implied volatility produces a flat line while market implied volatility produces a parabola or smile. The parabola shape indicates that implied volatility increases as moneyness (the distance between the strike and spot price) increases. One explanation for the increased price is that option sellers require an additional risk premium to compensate for the tail risk observed in financial time series. This finding has been confirmed across several different markets and assets classes; see Xu and Taylor (1994); Heynen (1994); Dumas, Fleming, and Whaley (1998); Lin, Chang, and Paxson (2008).

It has been observed that the implied volatility smile exists in certain markets, such as currency markets, whereas an implied volatility smirk is commonly observed in equity markets. The smirk differs in that the implied volatility for out-of-the-money puts increases much more sharply than out-of-the-money calls. This suggests that the market places a higher probability (or higher price premium) on large declines relative to large gains in equity markets, an effect which has been found to be more pronounced for equity index options relative to individual index options (Lin, Chang, and Paxson 2008). This may be explained by a relative abundance of demand for downside protection, an extra risk premium to cover tail risk, or a function of the option market’s own supply and demand dynamics, which according to Cont and da Fonseca (2002) have become increasingly autonomous. This fact is also supported by recent empirical evidence of violations of qualitative dynamical relations between options and their underlying (Bakshi, Cao, and Chen 2000).

While this paper does not directly measure changes in the slope of the implied volatility surface across regimes, it does so indirectly given that the contribution of a single option to the VIX Index value is proportional to $\Delta K$ and the price of that option, and inversely proportional to the square of the option’s strike price. That is, holding moneyness constant, an increase in the price of an option with strike K will increase the level of the VIX. Similarly, holding contract prices constant, increases in the amount of contracted options with greater moneyness will increase the VIX. Therefore, the simple model used in this paper allows for an indirect assessment of the impact that changes in the underlying price has on the implied volatility surface. Our results indicate that future work exploring changes in slope of the smile across different market regimes may provide useful insight into how the market anticipates outlier events. 


\section{Data and Methodology}

\subsection{Variables and Transformations}

Data used in the modeling process is detailed below. Each independent variable is a State Street Global Advisors SPDR Exchange Traded Fund (ETF) listed on the NYSE ARCA Stock Exchange. The SPDR S\&P 500 ETF Trust (SPY) seeks to provide investment results that, before expenses, correspond generally to the price and yield performance of the S\&P 500 Index. The S\&P 500 itself is a capitalization-weighted stock market index that measures the stock performance of 505 large companies listed on stock exchanges in the United States. Due to its aggregation of the country's largest publicly traded companies, the index is one of the factors in computation of the Conference Board Leading Economic Index, used to forecast the direction of the economy.

The additional ETFs used in the modeling process for Model 2 correspond to the price and yield performance of each sector comprising the S\&P 500. As dicussed below, these component sectors allow for a more granular approach to the question addressed with Model 1. The sector naming and grouping conventions follow the Global Industry Classification Standard (GICS) and include: XLB (Materials), XLE (Energy), XLF (Financials), XLI (Industrials), XLK (Technology), XLP (Consumer Staples), XLU (Utilities), XLV (Health Care), and XLY (Consumer Discretionary).

The dependent variable, VIX, is a calculation used to estimate expected volatility by aggregating the weighted prices of SPX puts and calls over a wide range of strike prices on the Chicago Board Options Exchange, as discussed in Section 2.1. All variables, dependent and independent, were sourced from Yahoo! Finance. Several measures were taken to ensure the validity of the data provided, including a comparison between the chosen dataset and a comparable dataset provided by WRDS CRSP. No data discrepancies were found between the two datasets, and an outlier and missing value assessment did not necessitate any data treatments. The data provided by Yahoo! Finance was found to be accurate, easy work with, and replicable. To ensure time consistency across all extracted variables, the daily adjusted closing price was used to construct the time series of each variable. 

\begin{table}[h] \centering 
\begin{tabular*}{\textwidth}{c @{\extracolsep{\fill}} ccclll}
\hline
\multicolumn{1}{c}{\textbf{Exogenous Variables}}                                                                                                                                                                         \\ \hline \hline 
 \textbf{ETFs}                            & \multicolumn{3}{c}{\textbf{Sectors}}                          & \multicolumn{3}{c}{\textbf{Sample Time}} \\ \hline \hline

\textit{\textbf{Model 1}}                     & \multicolumn{3}{c}{}                                                                                                                                                      \\ \hline
                                               SPY                                      & \multicolumn{3}{c}{S\&P 500}                                  & \multicolumn{3}{c}{1/29/1993 to 11/27/2020} \\ \hline

{\textit{\textbf{Model 2}}} & \multicolumn{3}{l}{}                                                                                                                                                      \\ \hline
                                                XLF                                   & \multicolumn{3}{c}{Financial}                                 & \multicolumn{3}{c}{12/22/1998 to 11/27/2020} \\ \hline
                                               XLK                                      & \multicolumn{3}{c}{Technology}                                & \multicolumn{3}{c}{12/22/1998 to 11/27/2020} \\ \hline
                                               XLE                                      & \multicolumn{3}{c}{Energy}                                   & \multicolumn{3}{c}{12/22/1998 to 11/27/2020} \\ \hline
                                               XLV                                       & \multicolumn{3}{c}{Health Care}                               & \multicolumn{3}{c}{12/22/1998 to 11/27/2020} \\ \hline
                                               XLY                                       & \multicolumn{3}{c}{Consumer Discretionary}                    & \multicolumn{3}{c}{12/22/1998 to 11/27/2020} \\ \hline
                                               XLI                                       & \multicolumn{3}{c}{Industrial}                                & \multicolumn{3}{c}{12/22/1998 to 11/27/2020} \\ \hline
                                               XLP                                       & \multicolumn{3}{c}{Consumer Staples}                          & \multicolumn{3}{c}{12/22/1998 to 11/27/2020} \\ \hline
                                               XLU                                       & \multicolumn{3}{c}{Utilities}                                 & \multicolumn{3}{c}{12/22/1998 to 11/27/2020} \\ \hline
                                               XLB                                       & \multicolumn{3}{c}{Materials}                                 & \multicolumn{3}{c}{12/22/1998 to 11/27/2020} \\ \hline

\multicolumn{1}{c}{\textbf{Endogenous Variables}}                                                                                                                                                                          \\ \hline \hline 
VIX & \multicolumn{3}{c}{CBOE Volatility Index} & \multicolumn{3}{c}{1/29/1993 to 11/27/2020} \\ \hline
\end{tabular*}
\end{table}



For modeling purposes, the log difference of each price series was calculated. In performing this operation, we not only get an interpretable value (returns), but we also impose stationarity on each series. This allows us to derive the expected value of the VIX at time \emph{t}, $E[VIX_t]$. ADF, KPSS trend, and KPSS linear tests results all indicate that stationarity is achieved through the variable transformation for each variable using in the modeling process. 

\subsection{The VIX and Realized Volatility}

Additional variables representing of the independent variable(s) variance were produced to show the correlation between VIX implied volatility and realized volatility. For the S\&P 500, three different time series are constructed. The first is a 30-day rolling standard deviation of S\&P 500 daily log returns is estimated and subsequently squared to derive an estimate of return variance. The second is an estimated S\&P 500 volatility using a GARCH (1,1) model. Third, for the estimated variance of the hypothetical portfolio of S\&P 500 constituent sectors, I apply an orthogonal change of basis on the 30-day rolling variance-covariance matrix of the nine sector ETFs. The variance-covariance matrix at each time step is re-expressed by pre-multiplying a vector containing the 30-day average market capitalization weights for each ETF, defined as $MktCap\_Weight = \dfrac{MktCap_i}{\sum_{i=1}^{9}MktCap_i}$, and post-multiplying by the transpose of the 30-day average market capitalization weights vector. In matrix notation, this operation is represented by $ABA^T$. This change of basis matrix is a $1\times 1$ scalar representing the market capitalization-weighted variance of the ETF portfolio. 

Several characteristics are observed and presented in Figures 1 and 2 regarding the different volatility series. First, the aggregated Sector ETF volatility is nearly identical to that derived directly from the 30-day rolling variance of S\&P 500 daily log returns. This provides confirming evidence that our sector ETFs are an appropriate decomposition of the S\&P 500 index. Second, the GARCH (1,1) process produces more extreme volatility estimates, particularly during the crisis periods, which is likely due to the fact that GARCH point estimates are not smoothed over a 30-day window. As a result, the GARCH estimate likely provides a better reflection of the severity of crisis periods. Third, VIX implied volatility underestimates realized volatility, particularly during crisis periods. This point is of particular interest given the results of our Crisis regime model, which indicates that contemporaneous price movements in the underlying have \textit{less} of an effect on the VIX during these periods. This underestimation appears to be present during all significant crises over the sample period and is apparent in Figure 2, which show contemporaneous GARCH estimated volatility of the S\&P 500 less contemporaneous VIX implied volatility. The color-coded series of \textit{VIX residuals} shows the magnitude of the market's underestimation of volatility during crises relative to other states of the market. The Crisis, Normal, and Low periods shown in Figure 3 are defined by the smooth probabilities generated by the Markov model.

%\begin{figure}[h]
%\caption{VIX and Estimated Volatilities}
%\includegraphics[width=.8\textwidth]{allvolsandvix.png} 
%\centering
%\label{}
%\end{figure}

\begin{figure}[H]
\caption{VIX and Estimated Volatilities}
\includegraphics[width=1\textwidth]{vols_ind.png} 
\centering
\label{}
\end{figure}

\begin{figure}[H]
\caption{GARCH Estimated Volatility Less VIX}
\includegraphics[width=1\textwidth]{vol_minus_vix.png} 
\centering
\label{}
\end{figure}

\subsection{Markov Switching Methodology}

To capture the non-linear relationship between options-implied volatility and contemporaneous price changes in the underlying, a Markov Switching is employed to estimate differences in parameter estimates across different market regimes. Following Psaradakis and Spagnolo (2002), I select the number is regimes based on a minimization of AIC. Psaradakis and Spagnolo’s Monte Carlo analysis revealed that selection procedures based on the so-called three-pattern method (TPM) and the AIC are generally successful in choosing the correct state dimension, provided that the sample size and parameter changes are not too small. BIC and HQC have a tendency to underestimate the state dimension. Given the large sample size and observation of large parameter changes, regime specification was based on AIC, interpretability, and prior literature. Between a two-state (AIC = -25168.88) and a three-state model (AIC = -25851.57), the latter performs better in terms of AIC. Three volatility states is also easily interpretable as 1) a low volatility regime with jumpy expectations, 2) a normal regime with mild volatility, and 3) a crisis regime with extreme volatility. Prior literature has explored the VIX under a switching framework. Baba and Sakurai (2011) use a regime switching approach to investigate the role of US macroeconomic variables as leading indicators of regime shifts in the VIX index. They found that there are three distinct regimes in the VIX index during the 1990 to 2010 period: tranquil regime with low volatility, turmoil regime with high volatility and crisis regime with extremely high volatility. They also show that the regime shift from the tranquil to the turmoil regime is significantly predicted by lower term spreads. In this paper, I consider only the dynamics of the impulse-response relationship between implied volatility and daily prices changes in the underlying across different regimes. However, I follow Baba and Sakurai (2011) and employ a three-regime model when exploring the dynamics of the VIX, albeit with a slightly different interpretations of regime states.

The three-state Markov chain that governs the probabilistic model driving the change between regimes in this paper does so in the follow manner;

\begin{center}

$Pr(s_t = j|s_{t-1} = i,s_{t-2} = k,...,y_{t-1},y_{t-2},...) = Pr(s_t = j|s_{t-1} = i) = p_{ij} $
\end{center}
\begin{flushleft}
The state space, $s_t$, is thus inferred from observed behavior of $VIX_t$. The probability law governing $VIX_t$ is described by the variance of the Gaussian innovation $\sigma^2$, the autoregressive coefficients ($VIX_{t-1, \dots,4$), the intercept, $S\&P500_t$, and the transition probabilities, $\omega_{1}$, $\omega_{2}$, and $\omega_{3}$ (Hamilton 2005). This data-driven approach allows for the assignment of volatility regimes, $s_t = 1,2,\ or \ 3$, that do not rely on subjectivity, making the research design more rigorous. The three-state model then becomes:
\end{flushleft}

$$ VIX_{t} = 
\begin{cases}
      c_{1t} + \beta_{1} \Delta SP500_{t} + \sum_{i=1}^{4}\phi_{1} VIX_{t-i} +  a_{1t} & \text{if}\ s_t=1 \\
      c_{2t} + \beta_{2} \Delta SP500_{t} + \sum_{i=1}^{4}\phi_{2} VIX_{t-i} +  a_{2t} & \text{if}\ s_t=2 \\
      c_{2t} + \beta_{3} \Delta SP500_{t} + \sum_{i=1}^{4}\phi_{3} VIX_{t-i} +  a_{3t} & \text{if}\ s_t=3 
    \end{cases}
$$
\vspace{}
\begin{flushleft}
The magnitude of coefficients $\beta_{1}$, $\beta_{2}$, and $\beta_{3}$ can then be determined to be statistically different from one another or not, which provides support for our against the non-linear hypothesis.
\end{flushleft}

Further, the first-order Markov chain used to generate the regime switching process is governed by transition probabilities, as presented below. These probabilities, denoted here as $\omega_{ij}$, show the probability of switching from one regime to another, conditional on the regime of the previous observation. 

\begin{center}

P(s_{t} = 1 | s_{t-1} = 2) = \omega_{1,2}, \quad\quad P(s_{t} = 1 | s_{t-1} = 3) = \omega_{1,3}

P(s_{t} = 2 | s_{t-1} = 1) = \omega_{2,1}, \quad\quad P(s_{t} = 2 | s_{t-1} = 3) = \omega_{2,3}

P(s_{t} = 3 | s_{t-1} = 1) = \omega_{3,1}, \quad\quad P(s_{t} = 3 | s_{t-1} = 2) = \omega_{3,2}

\end{center}

\begin{flushleft}
As mentioned, this setup allows the model the capture changes in parameter weights if the structural relationship between variables is deemed, through the Markov process, to have varied significantly. This dynamic parameter estimation serves four main functions. First, the Markov process produces state probabilities and respective parameter weights that differentiate between periods of low, high, and crisis volatility, thus generating conditional residuals that more closely follow a desired white-noise process. 
\end{flushleft}

Second, because the time series properties governing expected volatility are markedly different between periods of low and high volatility, the Markov process generates regime probabilities that correspond nicely to periods that we would define as crisis periods in the absence of the regime estimating process. That is, the data speaks for itself when seeking to define when a crisis begins and ends. For this dataset spanning from 01/29/1993 to 05/22/2020, the crisis periods include the market turbulence during the early 2000s dot-com bubble, the financial crisis in the late 2000s, and the Covid-19 crisis in 2020. 

Third, the multi-regime model yields three sets of parameter estimates, one for each regime. We can subsequently test the differences in parameters through simple tests of statistical significance. This allows one to determine whether or not the change in state space is driven primarily by changes in the dependent variable, changes in the relationship between the dependent variable and the independent variables, or a combination of the two. Thus, one can assess if the modeled relationship is non-linear. Model 2 allows for further granularity in our assessment of non-linearities by disaggregating the S\&P 500 into its component sectors, allowing us to determine of which sectors primarily drive regime changes within the dataset. 

Finally, the Markov process yields transition probabilities, which measure the persistence of each regime. The transition matrix allows us to examine the likelihood that we will observe a switch from one regime to another. Similarly, we can assess the persistence of low, jumpy, and high states. True state permanence would be represented by $\omega_{i} = 1$. However, the Markov formulation allows for the more general possibility that $\omega_{i} < 1$. Within business and market cycles we know that any given situation, though perhaps enduring, is persistent but not permanent. Furthermore, if the regime change reflects a fundamental change in monetary policy, fiscal policy, investor sentiment, liquidity, leverage, or debt-deflation, it would be prudent to allow the possibility for the regime to change back to its prior state or a new state entirely. This suggests that $\omega_{i} < 1$ is a correct specification given our data and objective. 

\subsubsection{Model Specification and Variable Switching}

For the regime switching models, correct specification of the appropriate lag order for our dependent variable helps ensures that the results are not biased by autocorrelation in the error term. Akaike information criterion (AIC) is used to identify the correct number of autoregressive terms. When considering the S\&P 500 as an exogenous variable, an AR(4) model was shown to minimize the AIC statistic and was therefore chosen as the preferred lag specification.\footnote{AIC = -22980.96} 

Furthermore, when constructing the specification and constraints of the switching model, the researcher has an option to hold certain variables constant across all regimes. I chose to produce an unconstrained model, which permits parameter switching for all variables and autoregressive terms.

\subsection{Tail Index Methodology}

The central point of this study is investigate whether or not volatility expectations form in manner that is non-linear with respect to the underlying asset. It is therefore important to quantify the differences between a linear approach and a non-linear approach to the stated problem. One way to capture the statistical consequences of an ill-fitted model is to measure the distributional characteristics of the residuals. Specifically, it is useful to measure conditional probabilities such that a residual exceeds a given threshold. Simply put, we want to assess the model's proclivity to produce large errors. This is known as the complementary cumulative distribution function (ccdf) or survival function, which captures both the location and shape of the tail distribution. %The parameters estimated from this procedure can then be compared to guide our understanding of the statistical consequences of failing to capture non-linearities in the relationship under study. 

Certain tail decays, such as a power law decay, are thought to be a sign of complexity and self-organized criticality in dynamical systems (Bak 1996 and Mandelbrot 2001), and the shape of the resulting tail distribution can characterize the magnitude and frequency of extreme events. It has been observed that the distributions of many economic and financial series exhibit a power-law decay in the tails (Mantegna and Stanley 1995; Clauset, Shalizi and Newman 2009; Sornette 2004). Understanding these distributional characteristics is of particular importance for risk analysis, hedging strategies, and derivatives pricing. For this paper, the tail characteristic of the models' distributions are estimated to better understand the importance of a multi-regime approach from a model error perspective. When it comes to (mis)pricing, it is the extreme misses, the rare miscalculation, that can be consequential for the both the individual and the system. For example, we want to characterize the conditional probabilities of model errors such that the error exceeds some minimum threshold, $x_{min}$. Assuming that the family of extreme value distributions falls into the Frechet type, residuals above this threshold will follow,

$$
P(\epsilon_t > x \ | \ \epsilon_t > x_{min}) = k|x|^{-\alpha}
$$

\begin{flushleft}

where $\epsilon_t$ is the model residual, $x_{min}$ is the minimum threshold that defines the location of the tail, and the parameter $\alpha$, the tail exponent, determines the rate at which the probability density drops off as one moves out into the tail (LeBaron 2009). The same is easily applied to the left tail. 
\end{flushleft}

Hill (1975) suggested that, for certain situations, it would be of interest to draw inference about the behavior of a distribution function in the tails without assuming that a particular parametric form for the distribution function holds globally. However, this requires that we first identify the location of the tail. That is if $\epsilon_1, \epsilon_2,\dots, \epsilon_k$ is a sample drawn from a population with distribution G and $\epsilon^{(1)} \ge  \epsilon^{(2)} \ge \dots \ge \epsilon^{(k)}$ are the order statistics, than there may exist some number $x_{min}$ such that $\epsilon\ge x_{min}$ defines the region where G is believed to form a Pareto distribution. Because $x_{min}$ is not known in practice, determining the tail region of the distribution is non-trivial. If $x_{min}$ is chosen to be too high, the variance of the estimator increases. If $x_{min}$ is too low, the bias of the estimator increases. One method for finding $x_{min}$ is to apply the Hill estimator on the entirety of the ordered sample population and determine the maximum value for $x_{min}$ such that the weighted Hill estimate converges on a single number. Beyond this number, $x_{min}^\ast$, the weighted Hill estimate will diverge, suggesting that the inclusion of data beyond $x_{min}^\ast$ produces a sample that is not Pareto distributed. Another method for finding $x_{min}^\ast$ is to start with a density plot of sorted data to visually identify a lower bound and ‘strict power law’ behavior. Further heuristics such as log-log plots and log-log rank plots and can identify a linear relationship to identify power law behavior, though these methods are not fool proof Nair et al. (2019). In Hubert et al. (2013) there is another Pareto test which looks for linear behavior in the QQ-plots of the log transformed data against standard exponential data. To determine $x_{min}^\ast$ for each residual series, both heuristics and quantitative methods were tested and explored. Ultimately, the appropriate $x_{min}^\ast$ was chosen using the extremefit package in R, which relies on a weighted version of the Hill estimator and on the pointwise data driven procedure of Durrieu et al 2015.

Once the $x_{min}^\ast$ threshold has been identified, the tail index, $\alpha$, can be estimated using the Maximum Likelihood Estimation (MLE) method of Newman (2005), which produces a biased estimate of $\hat{\alpha}$:

\begin{center}
$\hat{\alpha}= N\cdot [\sum_{i=1}^{N} \dfrac{x_i}{x_{min}^\ast}]^{-1}$
\end{center}

\begin{flushleft}

This MLE estimate can be converted to an unbiased version $\alpha^{\ast}$ following Rizzo (2009) with the following adjustment: 
\end{flushleft}

\begin{center}
$\alpha^{\ast}= \dfrac{{n-2}}{n}\cdot\hat{\alpha}$
\end{center}

Using these methods, I was able to characterize and compare each model's proclivity for large errors. Doing so allows us to quantitatively understand how a non-linear approach outperforms its linear alternative. These results are presented in Section 4.3.

\section{Non-linear response of implied volatility to innovations in the underlying across regimes}

The research question is studied in two separate models. The first model is a simple and parsimonious approach to assessing the response of S\&P 500 implied volatility to innovations in the underlying index, SPY, under different market regimes. The second approach decomposes the S\&P 500 into its component sectors. This approach allows us to determine if any non-linearities between states is consistent across all sectors ETFs.

\subsection{The Aggregate Model - VIX \& S\&P 500}

\subsubsection{Identifying Market Regimes}

The aggregate model (S\&P 500 regressed on the VIX) allows for a simple and parsimonious assessment on the non-linearity present in the relationship between a derivative and its underlying. As discussed in Section 3.2, the regime probabilities are identified through the Markov process. Figures 3, 4, and 5 show how each regime corresponds to the S\&P 500 return series over the sample period. The data driven process identifies a few distinct characteristics. First, crisis periods are unique and persistent, being defined primarily by the dot com volatility of the late 90s to early 2000s, the financial crisis during the late 2000s, and the more recent market turmoil resulting from the Covid-19 pandemic. Second, the regime labeled "Low Volatility" is observed primarily during the eight to nine years following the last financial crisis. As will be discussed below, sensitivity to price changes during periods of very low volatility appears to have increased in more recent years. This idea is captured visually in Figure 6, which shows the increasing volatility of the VIX during the past decade.\footnote{Volatility of the VIX is estimated using a GARCH(1,1) model} Third, aside from the infrequent jumps in expected volatility captured by the Low regime model, the Normal regime captures almost all periods when the market is not in turmoil. This suggests that it may be more accurate to describe the VIX - S\&P 500 relationship as a two state model (Crisis and Normal regimes) with infrequent jumps (Low regime). This description also fits the common notion of boom and bust cycles of markets. Finally, it is reassuring to visually confirm that the Markov Switching process does effectively capture our a priori assumption about when crisis periods begin and end; we are able to let the data speak for itself as opposed to subjecting the model to outside assumptions about when market regimes occur. 

\begin{figure}[H]
\caption{Low Regime Smooth Probability}
\includegraphics[width=.8\textwidth]{low_prob.png} 
\centering
\label{}
\end{figure}

  

\begin{figure}[H]
\caption{Moderate Regime Smooth Probability}
\includegraphics[width=.8\textwidth]{norm_prob.png} 
\centering
\label{}
\end{figure}

  

\begin{figure}[H]
\caption{Crisis Regime Smooth Probability}
\includegraphics[width=.8\textwidth]{crisis_prob.png} 
\centering
\label{}
\end{figure}


\begin{figure}[H]
\caption{Volatility of VIX}
\includegraphics[width=.8\textwidth]{vol_of_vix.png} 
\centering
\label{}
\end{figure}


%This model appears to capture the the increase in the volatility of the VIX, which is apparent in Figure 8. The Low regime model captures the jumpy behavior of the VIX in the recent, and relatively more tranquil, years. While it is beyond the scope of this paper to investigate the causes of the increase in VIX volatility, we speculate that this may result from increases in systematic and high frequency trading. 4) Aside from the infrequent jumps in expected volatility captured by the Low regime model, the Normal regime model captures almost all periods when the market is not in turmoil. This suggests that it may be more accurate to describe the VIX S\&P 500 relationship as a two state model (Crisis and Normal regimes) with infrequent jumps (Low regime). 

In order to assess how the proposed switching model performs, it is instructive to plot the model residuals over the sample periods to track performance across time. The same can be done for each regime-specific model output, which provides a visual of the non-linearity present in the system; we can quickly see how each estimated regime fit is uniquely tuned and produces large errors in "out-of-regime" periods. For example, if we were to assume a constant linear relationship between the VIX and the S\&P 500 using the Normal period model, we would drastically overestimate changes in the VIX during more turbulent times. Conversely, using the Crisis model would significantly underestimate changes in the VIX during more tranquil periods. Figures 7 and 8 demonstrate these findings. 

%Figure 8 below shows the performance of each regime-specific model, as well as a conditional model that switches between the three models based on the probability of being in each regime. A few distinct characteristics emerge from the following figures; 1) There is strong visual evidence of a non-linear relationship between the VIX and the S\&P 500. The residual plot provides a visual indicator of how each model is tuned to a certain market environment, outside of which performance is poor. For example, if we were to assume a constant linear relationship between the VIX and the S\&P 500 using the Normal period model, we would drastically overestimate changes in the VIX during more turbulent times. 





\begin{figure}[h]
\caption{Residuals by Regime - Model 1}
\includegraphics[width=1\textwidth]{resid_plot.png} 
\centering
\label{}
\end{figure}

\begin{figure}[H]
\caption{Residuals by Regime - Model 1}
\includegraphics[width=1\textwidth]{resid_plot_ind.png} 
\centering
\end{figure}


\subsubsection{Model Estimates}

Table 1 below provides the model output for each state of the market: low volatility regime with jumpy expectations, normal regime with low to moderate volatility, and crisis regime with extremely high volatility. The results evidence the statement put forth in Section 1; in states of the world when volatility is already high, the impact of additional price variation on volatility expectations diminishes. That is, informational scarcity (i.e. low volatility) makes additional data is more impactful. As discussed previously, the Low regime captures the VIX's jumpy behavior, producing a very strong relationship between the VIX and changes in the S\&P 500 when volatility is low. If S\&P 500 decreases by 1\%, we would expect to see an 11\% increase in the VIX. Given that the Low regime does not correspond specifically to one particular state of the market, this behavior is plausibly driven by the option market's own supply and demand mechanisms. The parameter estimate from the Normal regime falls between the Low and Crisis regime estimates. During periods of moderate volatility, a 1\% decrease in the S\&P 500 corresponds to a 6\% increase in the VIX. Finally, the results indicate that crisis periods exhibit a relative weakening in the relationship between the VIX and its underlying; a 1\% decrease in the S\&P 500 corresponds to a 2.6\% increase in the VIX. I suggest a few possible explanations for this seemingly counterintuitive finding. First, volatility expectations may be driven more by latent variables, such as policy announcements and news, during turbulent periods. Second, volatility expectations may be bounded from above, and thus do not respond as strongly during periods when actual volatility is already high. Third, it may also be the case the high volatilility is already "priced in" to options contracts during turbulent periods, and therefore traders do not update their beliefs when presented with additional price changes. Fourth, the unique supply and demand mechanism of the options market may be more prevalent during times of high uncertainty. Therefore, movements in the underlying explain less variation in the VIX relative to normal periods. Regardless of the exact explanation, there is strong evidence of a multi-regime relationship. This is further evidenced by the results in Table 1, which show that the differences in the S\&P 500 coefficients between regimes are highly significant. 



\begin{table}[H] \centering 
\small
  \caption{Markov Switching Regression Results - Model 1} 
  \label{} 
\begin{tabular}{@{\extracolsep{0pt}}lccc|ccc} 
\\[-1.8ex]\hline 
\hline \\[-1.8ex] 
 & \multicolumn{3}{c}{\textit{Dependent variable: VIX$_{t}$}}  & \multicolumn{3}{c}{\textit{Difference in $\beta$}}\\ 
\cline{2-4} 
\cline{4-7} 

\\[-1.8ex] & Low & Normal  & Crisis  & Low -  & Low - & Normal -\\ 
\\[-1.8ex] & Regime & Regime &  Regime & Normal & Crisis &  Crisis\\ 
\hline \\[-1.8ex] 

 S\&P 500$_t$ & $-$10.98$^{***}$ & $-$6.11$^{***}$ & $-$2.62$^{***}$  & 4.87$^{***}$ & 8.36$^{***}$ & 3.49$^{***}$\\ 
  & (0.34) & (0.12) & (0.06)  & (13.61) & (24.51) & (25.83) \\ 
  & & \\ 
 Constant & 0.01$^{***}$ & 0.001$^{*}$ & $-$0.0003 & $-$0.01$^{***}$ & $-$0.01$^{***}$ & $-$0.001$^{*}$\\ 
  & (0.002) & (0.001) & (0.001) & ($-$2.4) & ($-$3.0) & ($-$1.48)\\ 
  & & \\ 
 VIX$_{t-1}$ & $-$0.08$^{***}$ & $-$0.08$^{***}$ & $-$0.04$^{**}$ & $-$0.00004 & 0.04$^{**}$ & 0.04$^{***}$\\ 
  & (0.02) & (0.01) & (0.01) & (0.002) & (1.55) & (2.19)\\ 
  & & \\ 
 VIX$_{t-2}$ & $-$0.05$^{**}$ & $-$0.08$^{***}$ & $-$0.07$^{***}$& $-$0.02 & $-$0.01 & 0.01 \\ 
  & (0.02) & (0.01) & (0.013) & ($-$0.85) & ($-$0.47) & (0.55) \\ 
  & & \\ 
 VIX$_{t-3}$ & $-$0.08$^{***}$ & $-$0.05$^{***}$ & $-$0.05$^{***}$ & $-$0.03 & 0.03 & 0.001\\ 
  & (0.02) & (0.01) & (0.01) & (1.14) & (1.12) & (0.04)\\ 
  & & \\ 
 VIX$_{t-4}$ & $-$0.02 & $-$0.05$^{***}$ & $-$0.06$^{***}$ & $-$0.03 & $-$0.03 & $-$0.01  \\ 
  & (0.02) & (0.01) & (0.01)& ($-$0.97) & ($-$1.26) & ($-$0.57) \\ 
  & & \\ 
\hline \\[-1.8ex] 
Total Obs & 6,979 \\ 
R$^{2}$ & 0.78 & 0.75 & 0.56 \\ 
Resid Std. Error & 0.051 & 0.03 & 0.037 \\ 
\hline 
\hline \\[-1.8ex] 
\textit{Note:}  & \multicolumn{2}{r}{$^{*}$p$<$0.1; $^{**}$p$<$0.05; $^{***}$p$<$0.01} & & \multicolumn{2}{c}{\textit{z-scores in parenthesis}} \\ 
\end{tabular} 
\end{table} 

The model shows that the regime corresponding to the Crisis periods is the most persistent at 0.98, followed by the Normal regime at 0.95, and the Low regime at 0.86. The transition matrix is presented below.   

\begin{table}[!htbp] \centering 
  \caption{Transition Matrix - Model 1} 
  \label{} 
\begin{tabular}{@{\extracolsep{5pt}} cccc} 
\\[-1.8ex]\hline 
\hline \\[-1.8ex] 
 & Crisis Regime & Normal Regime & Low Regime \\ 
\hline \\[-1.8ex] 
Crisis Regime & $0.98$ & $0.02$ & $0.002$ \\ 
Normal Regime & $0.02$ & $0.95$ & $0.14$ \\ 
Low Regime & $0.001$ & $0.04$ & $0.86$ \\ 
\hline \\[-1.8ex] 
\textit{Note:}  & \multicolumn{3}{r}{Columns = $VIX_{t-1}$ regime; Rows = $VIX_{t}$ regime} \\ 
\end{tabular} 
\end{table} 

\clearpage

\subsection{Model 2 }

As expected, Model 2 shows very similar behavior to Model 1 when comparing the residuals of each regime. The purpose of Model 2, however, is to identify which sectors within the S\&P 500 drive the regime switching behavior. For example, between the Normal regime and Crisis regime, the only two sectors that display significantly different behavior is the Financial Sector and the Technology Sector. We also observe that not all sectors are statistically significant across all regimes. In fact, the Financial Sector is the only sector that show statistically significant differences between \textit{all} regimes. The Materials Sector, for example, is significant during normal times but does not seem influence volatility expectations during crises present in the sample. Nor does this sector contribute to the VIX's jumpy behavior. Additional sector-specific behavior can be found in Table 3 below.   

\begin{figure}[h]
\caption{Residuals by Regime - Model 2}
\includegraphics[width=1\textwidth]{sector_resid_plot.png} 
\centering
\label{}
\end{figure}

%\begin{figure}[H]
%\caption{Residuals by Regime - Model 2}
%\includegraphics[width=.8\textwidth]{ResidualsbyRegimeindSector.png} 
%\centering
%\end{figure}





\begin{table}[H] \centering 
\small
  \caption{Markov Switching Regression Results - Model 1} 
  \label{} 
     \scalebox{0.9}{
\begin{tabular}{@{\extracolsep{0pt}}lccc|ccc} 
\\[-1.8ex]\hline 
\hline \\[-1.8ex] 
 & \multicolumn{3}{c}{\textit{Dependent variable: VIX$_{t}$}}  & \multicolumn{3}{c}{\textit{Difference in $\beta$}}\\ 
\cline{2-4} 
\cline{4-7} 

\\[-1.8ex] & Low & Normal  & Crisis  & Low -  & Low - & Normal -\\ 
\\[-1.8ex] & Regime & Regime &  Regime & Normal & Crisis &  Crisis\\ 
\hline \\[-1.8ex] 
 Financial & $-$2.026$^{***}$ & $-$0.659$^{***}$ & $-$0.235$^{***}$ & -1.366$^{***}$ & -1.790$^{***}$ &   -0.424$^{***}$\\ 
  & (0.362) & (0.106) & (0.053)  & (-3.591) &  (-4.888)   &   (-3.618)\\ 
  & & \\ 
 Technology & $-$3.272$^{***}$ & $-$1.011$^{***}$ & $-$0.787$^{***}$  & -2.262 & -2.486$^{***}$  &  -0.224$^{***}$ \\ 
  & (0.371) & (0.124) & (0.062)  & (-1.611) &  (-6.602)   &   (-5.775) \\ 
  & & \\ 
 Energy & $-$0.457$^{**}$ & $-$0.519$^{***}$ & $-$0.362$^{***}$ & 0.062$^{*}$ & -0.095  &  -0.157\\ 
  & (0.204) & (0.073) & (0.060) & (-1.670) &  (-0.447)    &   (0.285)   \\ 
  & & \\ 
 Health Care & $-$1.449$^{***}$ & $-$1.003$^{***}$ & $-$0.549$^{***}$ & -0.446$^{***}$ & -0.900$^{**}$  &  -0.454 \\ 
  & (0.364) & (0.137) & (0.091)  & (-2.758) &  (-2.395)    &  (-1.147)  \\ 
  & & \\ 
 Consumer Discretionary & $-$1.639$^{*}$ & $-$0.999$^{***}$ & $-$0.250$^{***}$ & -0.640$^{***}$ & -1.389$^{**}$  &  -0.749  \\ 
  & (0.462) & (0.167) & (0.090)   & (-3.943)  & (-2.951)    &  (-1.302) \\ 
  & & \\ 
 Industrial & $-$1.052$^{**}$ & $-$0.462$^{***}$& $-$0.170  & -0.590 & -0.882$^{*}$  &  -0.292 \\ 
  & (0.456) & (0.168) & (0.123)  & (-1.402) &  (-1.867)    &  (-1.213) \\ 
  & & \\ 
 Consumer Staples & $-$0.701$^{*}$ & $-$0.759$^{***}$ & $-$0.329$^{***}$  & 0.058$^{**}$ & -0.372  &  -0.430 \\ 
  & (0.407) & (0.163) & (0.100)   & (-2.249) &  (-0.887)    &  (0.132) \\ 
  & & \\ 
 Utilities & $-$0.133 & $-$0.218$^{**}$ & 0.179$^{**}$ & 0.086$^{***}$ & -0.311  &  -0.397 \\ 
  & (0.267) & (0.105) & (0.077)  & (-3.057) &  (-1.120)  &    (0.299)  \\ 
  & & \\ 
 Materials & $-$0.364 & $-$0.547$^{***}$ & $-$0.022 & 0.183$^{***}$ & -0.342  &  -0.525 \\ 
  & (0.335) & (0.118) & (0.084)  & (-3.631)  & (-0.990)   &   (0.514) \\ 
  & & \\ 
 Constant & 0.006$^{***}$ & 0.001 & $-$0.002$^{**}$  & 0.006$^{***}$ &  0.008$^{***}$   &  0.003$^{***}$\\ 
  & (0.002) & (0.001) & (0.001)  & (2.321)  &  (3.927)    &   (2.764) \\ 
  & & \\ 
 VIX$_{t-1}$ & $-$0.067$^{***}$ & $-$0.083$^{***}$ & $-$0.047$^{***}$  & 0.016$^{*}$ & -0.019  &  -0.035\\ 
  & (0.019) & (0.011) & (0.015)  & (2.321)  &  (3.927)    &   (2.764) \\ 
  & & \\ 
 VIX$_{t-2}$ & $-$0.068$^{***}$ & $-$0.074$^{***}$ & $-$0.072$^{***}$  & 0.006 & 0.005 &   -0.001 \\ 
  & (0.020) & (0.011) & (0.015)   & (-0.077) &  (0.194)  &    (0.274) \\ 
  & & \\ 
 VIX$_{t-3}$ & $-$0.042$^{***}$ & $-$0.058$^{**}$ & $-$0.064$^{***}$  & 0.016 & 0.022   &  0.006  \\ 
  & (0.021) & (0.011) & (0.015)   & (0.305)  &  (0.851)   &    (0.680)  \\ 
  & & \\ 
 VIX$_{t-4}$ & $-$0.035 & $-$0.053$^{***}$ & $-$0.056$^{***}$  & 0.018 & 0.021  &   0.003  \\ 
  & (0.023) & (0.011) & (0.014)   & (0.170)  &  (0.793)   &    (0.727) \\ 
  & & \\ 
\hline \\[-1.8ex] 
Total Observations & 5,267 \\ 
R$^{2}$ & 0.83 & 0.77 & 0.61 \\ 
Residual Std. Error & 0.04 & 0.03 & .04 \\ 
\hline 
\hline \\[-1.8ex] 
\textit{Note:}  & \multicolumn{2}{r}{$^{*}$p$<$0.1; $^{**}$p$<$0.05; $^{***}$p$<$0.01} & & \multicolumn{2}{c}{\textit{z-scores in parenthesis}} \\ 
\end{tabular} 
}
\end{table} 





Model 2's transition matrix is similar to that of Model 1. The Crisis regime is the most persistent, followed by the Normal regime, followed by the Low regime. The transition matrix is presented below.   

\begin{table}[H] \centering 
  \caption{Transition Matrix - Model 2} 
  \label{} 
\begin{tabular}{@{\extracolsep{5pt}} cccc} 
\\[-1.8ex]\hline 
\hline \\[-1.8ex] 
 & Low Regime & Normal Regime & Crisis Regime \\ 
\hline \\[-1.8ex] 
Low Regime & $0.89$ & $0.03$ & $0.003$ \\ 
Normal Regime & $0.10$ & $0.95$ & $0.022$ \\ 
Crisis Regime & $0.01$ & $0.02$ & $0.98$ \\ 
\hline \\[-1.8ex] 
\textit{Note:}  & \multicolumn{3}{r}{Columns = $VIX_{t-1}$ regime; Rows = $VIX_{t}$ regime} \\ 
\end{tabular} 
\end{table} 



\subsection{Tail Behavior and Model Performance}

Given the above estimation results of the regime-switching model, it is important to understand how the non-linear approach compares to its linear counterpart. To achieve a comparison that appropriately captures the risk associated with large mispricings\footnote{Large mispricings are the true consequential events that determine large periodic transfers of wealth.}, I aim to characterize the tail behavior of each model's residuals (see Appendix for plot of residual distributions). Using the approach described in Section 3.3, one can identify the location and shape of the tails, which helps describes the decay in the likelihood of observing an event far from the mean. Further, because the residual distributions are two-tailed, this method can also recognize possible asymmetries between the decay of the left and right tail. Asymmetries in the left and right tail behavior describe whether the model has a propensity large under-predictions or large over-predictions, where model errors are defined as $\epsilon= y - \hat{y}$.\footnote{The left tail contains over-predictions of volatility increase and under-predictions of volatility decreases ($\epsilon < 0$). The right tail contains over-predictions of volatility decreases and under-predictions of volatility increases ($\epsilon > 0$).} 

The first step in the tail index estimation procedure is to determine the location of the tail. The Hill estimation approach discussed in Section 3.3 offers a method for determining the values, beyond which plausibly form a Pareto distribution. Weighted Hill estimations for each series, and for both tails, converge and stabilize on a certain value within the identified tails (see appendix for Hill estimation results). Further support for the Pareto hypothesis can be shown by plotting the empirical observations within the tail against the theoretical quantiles of a generalized Pareto distribution. These QQ plots are presented in the appendix and show additional support for the Pareto hypothesis.\footnote{A straight line on the QQ plot indicates agreement with the Pareto hypothesis} The Maximum Likelihood Estimation procedure outlined above yields estimates for the $\alpha$ parameter that describes the decay of the tails. For the (Conditional) Markov Switching model, the tail index for the right tail is $\alpha = 3.7$ and the left tail is $\alpha = 4.01$. Further, the $x_{min}$ value is 0.069 for the right tail and 0.066 for the left tail. For the (Linear) AR(4) model, the tail index for the right tail is $\alpha = 2.84$ and the left tail is $\alpha = 3.62$. The slower decay present in the Linear model relative to the Non-linear model indicates a higher propensity for extreme errors. This is particularly present in the right tail, which represents over-predictions of volatility decreases and under-predictions of volatility increases. Further, the $x_{min}$ value is 0.093 for the right tail and 0.074 for the left tail. Again, the higher $x_{min}$ relative to the non-linear indicates that the expected value within the tails is higher, where $E[\epsilon] = \frac{\alpha x_{min}}{\alpha -1}$. Therefore, holding $\alpha$ constant, increases in the $x_{min}$ correspond to higher expected value for an error conditional on it being within the tail region. Table 5 below demonstrates how the parameter estimates translate in the expected values, conditional on a residual being above the $x_{min}$ threshold. We see that the Linear model has a higher expected value in both tails, especially the right tail. 

\begin{table}[H] \centering 
  \caption{Expected Value - Residual Tails} 
  \label{} 
\begin{tabular}{@{\extracolsep{5pt}} cc} 
\\[-1.8ex]\hline 
\hline \\[-1.8ex] 
 & Expected Value\\ 
\hline \\[-1.8ex] 
Conditional Right Tail & $0.094$ \\ 
Conditional Left Tail & $0.089$ \\ 
\hline \\[-1.8ex] 
Linear Right Tail & $0.142$ \\ 
Linear left Tail & $0.103$ \\ 
\hline \\[-1.8ex] 
\end{tabular} 
\end{table} 

Figure 10 below shows the log-log tail densities of both the Switching model and the Linear model. The steepness of the slope represents the rate of decay. To put this in visual perspective, Pareto distributions are simulated in Figure 11 using the above estimated shape and location parameters. We see that the right tail of the linear model is shifted to the right and presents a slower decay relative to the conditional model. Log-log tail densities, as well as simulated Pareto distributions, are plotted and presented for each model in the Appendix. 


\begin{figure}[H]
\caption{Log Log Plot of Tail Distribution - Switching vs. Linear Model}
\includegraphics[width=.7\textwidth]{log_log_cond_lin.png} 
\centering
\label{}
\end{figure}

\begin{figure}[H]
\caption{Pareto Simulation - Conditional vs. Linear Model}
\includegraphics[width=.7\textwidth]{pareto_sim.png} 
\centering
\label{}
\end{figure}


%Figure 12 demonstrates the possible existence of heavy tailed and long tailed distributions. Following the methodology detailed in Section 3.3, we identify the location of the left and right tails and determine whether or not the resulting data is Pareto distributed (see appendix for qq plots - a linear qq plot indicates that the data is likely Pareto distributed). We then estimate the Weighted Hill parameter, which converges on a single value when data is limited to the left or right tail of the distribution, providing evidence that we have correctly identified the location of the tail. We then use Maximum Likelihood  Estimation to determine the shape and scale parameters of each Pareto distribution. Obtaining these values allows us to compare the length and decay of the tails across models, the results of which indicate significant differences. Finally, we plot the tail distributions on a log-log graph so that we can visualize asymmetries between left and right tails. Given that the regime-specific models are tuned to certain market conditions and not the others, we expect the tails to exhibit poor performance. Thus, we finish this discussion be identifying the differences between the Conditional Switching Model and the Linear Model. We observe that the linear model exhibits much slower decay and a higher expected value in the right tail when compared to either the left tail or both tails of the Conditional Switching Model. This indicates that a simple linear model tends to underestimate positive movements in the VIX relative to a multi-regime model.  


\section{Conclusion}

As demonstrated above, there is evidence that option prices vary in a non-linear and regime-dependent fashion with contemporaneous price changes in their underlying asset. The correlative results show that the supply and demand dynamics of equity options are markedly different between periods of stability and instability, which provides insight into how expectations are formed when participants are faced with \textit{repeated}, uncertain, and financially significant decisions. 

%That is, the model provides evidence against the time-invariant structure between an option and its underlying assumed by the Black-Scholes Merton pricing formula.

The decrease in magnitude of the S\&P 500 coefficient as the market switches from low volatility to high, suggests that information scarcity (low volatility) makes additional data (price changes) is more impactful. Conversely, crisis periods are accompanied by meaningful changes in variables that are \textit{not} included in this model, such as monetary and fiscal policy. Latent variables, therefore, play a more significant role in risk determination when such information is more plentiful, noisy, and multidimensional. In such instances, new information concerning the price of the underlying asset is less significant. In fact, volatile price movement may already be "priced in" to the option contract once the Crisis regime is initiated. Market participants may therefore form their expectations through the use of complex heuristics rather than strict pricing formulae that do not fully capture the complexity of markets. Further, the idea that volatility expectations show increased sensitivity to new price information during periods of low volatility is instructive for monetary policy. Over-intervention in financial markets through interest rate policy does not allow the pricing mechanism to reflect accurate information about the stability of the system. For example, low interest rate environments may allow poorly performing companies to obtain financing that would otherwise be out of reach when credit is tight, thus allowing negative information to remain hidden. These volatility-limiting actions may decrease system stability similar to the way a biological system's strength might deteriorate in the absence of mild and frequent stressors. Therefore, when "normal" volatility is limited, traders view price variations as a potential regime change, causing more reactionary behavior and possibly providing the behavioral foundation for a subsequent change in system's state.

While this paper has aimed to provide evidence of a non-linear relationship between a derivative and its underlying, the analysis does not go beyond a correlative exploration. Nor does this work offer an alternative pricing formula that incorporates the non-linear structure found above. The goal of this paper was to demonstrate that one-size-fits-all approaches are not applicable to financial markets, and doing so can be costly. Further, this paper also aims to understand the rationale behind the increased price sensitivity during periods of low volatility. To do so, I have reached beyond the economics literature to explore concepts within the field of signal processing and stochastic resonance. While I believe there is good reason to apply these concepts to help explain the non-linearity found in the studied relationship, further research should explore the relevance of signal-to-noise ratios of non-linear \textit{economic} systems. Further, I put forth the idea that monetary policy is a proximate cause of volatility reductions. Future research can further analyze how monetary policy changes impact the relationship modeled in this paper. 




%\clearpage

\begin{thebibliography}{53}

\bibitem{Hamilton 1989}
Hamilton, J. (1989). \textit{A New Approach to the Economic Analysis of Nonstationary Time Series and the Business Cycle.} Econometrica, 57(2), 357-384.

\bibitem{Mantegna and Stanley 1995}
R. N. Mantegna and H. E. Stanley, \textit{Scaling behaviour in the dynamics of an economic index,} Nature, vol. 376, no. 6535, pp. 46-49, 1995. 

\bibitem{Taleb 2019}
Taleb NN (2019). \textit{The Statistical Consequences Of Fat Tails.} Stem Academic Publishing, 2019.

\bibitem{McDonnell, Mark D, and Derek Abbott 2009}
McDonnell, Mark D, and Derek Abbott (2009) \textit{What is stochastic resonance? Definitions, misconceptions, debates, and its relevance to biology.} PLoS computational biology vol. 5,5 (2009):

\bibitem{Hill 1975}
Hill, Bruce M. \textit{A Simple General Approach to Inference About the Tail of a Distribution.} Ann. Statist. 3 (1975), no. 5, 1163--1174.

\bibitem{Cont, Fonseca 2002}
Rama Cont & José da Fonseca, \textit{Dynamics of implied volatility surfaces}, Quantitative Finance, 2002, 2:1, 45-60

\bibitem{Miller 1977}
Miller, Edward M. \textit{Risk, Uncertainty, and Divergence of Opinion.} The Journal of Finance, vol. 32, no. 4, 1977, pp. 1151–1168

\bibitem{Kearns, Pagan 1997}
Kearns, Phillip, and Adrian Pagan. \textit{Estimating the Density Tail Index for Financial Time Series.} The Review of Economics and Statistics, vol. 79, no. 2, 1997, pp. 171–175

\bibitem{Psaradakis, Spagnolo 2003}
Psaradakis, Z. and Spagnolo, N., \textit{On the Determination of the Number of Regimes in Markov- Switching Autoregressive Models.} Journal of Time Series Analysis, 2003, 24: 237-252. 

\bibitem{Auinger 2015}
Auinger, F. \textit{The causal relationship between the S&P 500 and the VIX index: Critical analysis of financial market volatility and its predictability.} 2015

\bibitem{Clauset, Shalizi, Newman 2009}
A. Clauset, C. R. Shalizi, and M. E. J. Newman, \textit{Power-Law Distributions in Empirical Data,} SIAM Review, vol. 51, no. 4, pp. 661-703, 2009. 

\bibitem{Christensen, Prabhala 1998}
Christensen, Bent Jesper and Prabhala, N. R., The relation between implied and realized volatility, Journal of Financial Economics, 1998, 50, issue 2, p. 125-150

\bibitem{Cont 2001}
R. Cont, \textit{Empirical properties of asset returns: Stylized facts and statistical issues,} Quantitative Finance, vol. 1, pp. 223- 236, feb 2001. 

\bibitem{Low 2004}
Low, Cheekiat. \textit{The Fear and Exuberance from Implied Volatility of S&P 100 Index Options.} The Journal of Business, vol. 77, no. 3, 2004, pp. 527–546.

\bibitem{Gabaix, Gopikrishnan, Plerou, Stanley 2003}
X. Gabaix, P. Gopikrishnan, V. Plerou, and H. E. Stanley, \textit{A theory of power-law distributions in financial market fluctuations,} Nature, vol. 423, pp. 267-270, may 2003. 

\bibitem{Gabaix 2009}
X. Gabaix, \textit{Power Laws in Economics and Finance,} Annual Review of Economics, vol. 1, no. 1, pp. 255-294, 2009. 

\bibitem{Begušić, Kostanjcar, Stanley, Podobnik 2018}
Begušić, Stjepan & Kostanjcar, Zvonko & Stanley, H. & Podobnik, Boris. (2018). \textit{Scaling properties of extreme price fluctuations in Bitcoin markets.} Physica A: Statistical Mechanics and its Applications. 510. 

\bibitem{Durrieu, Grama, Jaunatre, Pham, Tricot 2018}
Durrieu G, Grama I, Jaunatre K, Pham Q, Tricot J (2018). \textit{extremefit: A Package for Extreme Quantiles.} Journal of Statistical Software, 87 (12), 1-20.

\bibitem{Saez-Castillo, Prieto, Sarabia 2015}
Antonio Jose Saez-Castillo, Faustino Prieto and Jose Maria Sarabia (2015). \textit{ParetoPosStable: Computing, Fitting and Validating the PPS Distribution.} R package version 1.1.

\bibitem{Munasinghe, Kossinna, Jayasinghe, Wijeratne 2019}
Ranjiva Munasinghe, Pathum Kossinna, Dovini Jayasinghe and Dilanka Wijeratne (2019). \textit{ptsuite: Tail Index Estimation for Power Law Distributions.} R package version 1.0.0.

\bibitem{Mandelbrot 2001}
Mandelbrot, Benoit. (2001). \textit{Scaling in Financial Prices: I. Tails and Dependence.} Quantitative Finance. 1. 113-123

\bibitem{Whaley 2002}
Whaley, R., 2002. \textit{On the return and risk of the CBOE Buy Write monthly index.} J. Deriv. 10, 35-42.

\bibitem{Sarwar 2012}
Sarwar, Ghulam, 2012. \textit{Is VIX an investor fear gauge in BRIC equity markets?,} Journal of Multinational Financial Management, Elsevier, vol. 22(3), pages 55-65.

\bibitem{Jackwerth and Rubinstein 1996}
Jackwerth, J., and M. Rubinstein, 1996, \textit{Recovering Probability Distributions from Option Prices,} Journal of Finance 51, 1611-1631.

\bibitem{Coval and Shumway 2001}
Coval, J., and T. Shumway, 2001, \textit{Expected Option Returns,} Journal of Finance 56, 983-1009.

\bibitem{Bakshi & Kapadia 2003}
Bakshi, G., and N. Kapadia, 2003, \textit{Delta-hedged Gains and the Negative Volatility Risk Premium,} Review of Financial Studies 16, 527-566.

\bibitem{Doran, Banerjee, Peterson 2006}
Doran, James and Banerjee, Prithviraj and Peterson, David R., \textit{Implied Volatility and Future Portfolio Returns (December 4, 2006).} Journal of Banking and Finance, Vol. 31, October 2007.

\bibitem{Ben-David and Moussawi 2018}
Ben-David, Itzhak & Moussawi, Rabih. (2018). \textit{Do ETFs Increase Volatility?.} The Journal of Finance. 

\bibitem{Cochran, Mansur, Odusami 2015}
Cochran, Steven J. & Mansur, Iqbal & Odusami, Babatunde, 2015. \textit{Equity market implied volatility and energy prices: A double threshold GARCH approach,} Energy Economics, Elsevier, vol. 50(C), pages 264-272.

\bibitem{Yuan and Mitra 2016}
Yuan, Yuan and Mitra, Gautam, \textit{Market Regime Identification Using Hidden Markov Models}, 2016 

\bibitem{Fink, Klimova, Czado, Stöber 2017}
Holger Fink & Yulia Klimova & Claudia Czado & Jakob Stöber, 2017. \textit{Regime Switching Vine Copula Models for Global Equity and Volatility Indices,} Econometrics, MDPI, Open Access Journal, vol. 5(1), pages 1-38, January.

\bibitem{Hesse and González-Hermosillo 2009}
Hesse, Heiko and González-Hermosillo, Brenda, \textit{Global Market Conditions and Systemic Risk (October 2009).} IMF Working Papers, Vol. , pp. 1-22, 2009. 

\bibitem{Ying et al. 2011}
Ma, Ying et al. \textit{A Portfolio Optimization Model with Regime-Switching Risk Factors for Sector Exchange Traded Funds.}, 2011.

\bibitem{Husson and McCann 2011}
Husson, Tim and McCann, Craig J., \textit{The VXX ETN and Volatility Exposure (July 15, 2011).} PIABA Bar Journal, Volume 18, No. 2, 2011

\bibitem{Papanicolaou and Sircar 2014}
Papanicolaou, Andrew and Sircar, Ronnie, \textit{A Regime-Switching Heston Model for VIX and S\&P 500 Implied Volatilities (April 25, 2013).} Quantitative Finance, Volume 14, Issue 10, (2014) pp. 1811-1827. 

\bibitem{Goutte, Ismail, Pham 2017}
Stéphane Goutte, Amine Ismail, Huyên Pham. \textit{Regime-switching Stochastic Volatility Model : Estimation and Calibration to VIX options.} 2017.

\bibitem{Tsay 1998}
Tsay, R., 1998. \textit{Testing and modeling multivariate threshold models.} J. Am. Stat. Assoc. 93, 1188-1202.

\bibitem{Durrieu, Grama, Pham, et al 2015}
Durrieu, G., Grama, I., Pham, Q. et al. \textit{Nonparametric adaptive estimation of conditional probabilities of rare events and extreme quantiles.} Extremes 18, 437–478 (2015). 

\bibitem{Black, Fischer, Scholes 1973}
Black, Fischer, and Myron Scholes. \textit{The Pricing of Options and Corporate Liabilities.} Journal of Political Economy, vol. 81, no. 3, 1973, pp. 637–654. JSTOR

\bibitem{Ghysels, Harvey, Renault 1996}
Ghysels, E. & Harvey, A. & Renault, E., 1996. \textit{Stochastic Volatility,}
Cahiers de recherche 9613, Centre interuniversitaire de recherche en economie quantitative, CIREQ.

\bibitem{Wiggins 1987}
James B. Wiggins,\textit{Option values under stochastic volatility: Theory and empirical estimates,} Journal of Financial Economics, Volume 19, Issue 2, 1987, Pages 351-372, ISSN 0304-405X, 

\bibitem{Merton 1976}
Robert C. Merton, \textit{Option pricing when underlying stock returns are discontinuous,} Journal of Financial Economics, Volume 3, Issues 1–2, 1976, Pages 125-144, ISSN 0304-405X, 

\bibitem{Kou 2002}
Kou, S G, \textit{A Jump-Diffusion Model for Option Pricing, Management Science,} 48, issue 8, 2002 p. 1086-1101, 

\bibitem{Klein 1996}
Peter Klein, \textit{Pricing Black-Scholes options with correlated credit risk,} Journal of Banking & Finance, Volume 20, Issue 7, 1996, Pages 1211-1229, ISSN 0378-4266, 

\bibitem{Burgard, Kjaer 2011}
C. Burgard and M. Kjaer. \textit{Partial differential equation representations of derivatives with counterparty risk and funding costs.} The Journal of Credit Risk, Vol. 7, No. 3, 1-19, 2011

\bibitem{Feng 2011}
Shih-Ping Feng, \textit{The Liquidity Effect In Option Pricing: An Empirical Analysis,} The International Journal of Business and Finance Research, The Institute for Business and Finance Research, vol. 5(2), pages 35-43, 2011.

\bibitem{Barles, Soner 1998}
Barles, G. and H. Soner. \textit{Option pricing with transaction costs and a nonlinear Black-Scholes equation.} Finance and Stochastics 2 (1998): 369-397.

\bibitem{Amster 2005}
Amster, P. et al. \textit{A Black-Scholes option pricing model with transaction costs.} Journal of Mathematical Analysis and Applications 303 (2005): 688-695.

\bibitem{Skiadopoulos, et al 2000}
Skiadopoulos, G. et al. \textit{The Dynamics of the S&P 500 Implied Volatility Surface.} Review of Derivatives Research 3 (2000): 263-282.

\bibitem{Fengler, et al 2003}
Fengler, Matthias R. et al. \textit{The Dynamics of Implied Volatilities: A Common Principal Components Approach.} Review of Derivatives Research 6 (2003): 179-202.

\bibitem{Benko, et al 2009}
Benko, M. et al. \textit{Common functional principal components.} Annals of Statistics 37 (2009): 1-34.

\bibitem{Bernales, Guidolin 2015}
Bernales, A. and M. Guidolin. \textit{Learning to smile: Can rational learning explain predictable dynamics in the implied volatility surface?☆.} Journal of Financial Markets 26 (2015): 1-37.

\bibitem{Haug, Taleb 2010}
Haug, E. G. and N. Taleb. \textit{Option traders use (very) sophisticated heuristics, never the Black- Scholes-Merton formula 1.} (2010).


\end{thebibliography}

\section{Appendix}

\begin{figure}[H]
\centering
\begin{subfigure}{.5\textwidth}
  \centering
  \includegraphics[width=1\linewidth]{right_hill.png}
  \caption{Hill Estimate - Right Tail}
  \label{fig:sub1}
\end{subfigure}%
\begin{subfigure}{.5\textwidth}
  \centering
  \includegraphics[width=1\linewidth]{left_hill.png}
  \caption{Hill Estimate - Left Tail}
  \label{fig:sub2}
\end{subfigure}
\caption{Hill Estimates}
\label{fig:hill}
\end{figure}

\begin{figure}[H]
\centering
\begin{subfigure}{.5\textwidth}
  \centering
  \includegraphics[width=1\linewidth]{dist_of_resid.png}
  \caption{Distribution of Residuals}
  \label{fig:sub1}
\end{subfigure}%
\begin{subfigure}{.5\textwidth}
  \centering
  \includegraphics[width=1\linewidth]{pareto_sim_all.png}
  \caption{Simulated Pareto Distribution}
  \label{fig:sub2}
\end{subfigure}
\caption{Residual Distributions}
\label{fig:res_dist}
\end{figure}




%\begin{figure}[h]
%\caption{Log Log Plot of Tail Distribution - Conditional Model}
%\includegraphics[width=.8\textwidth]{cond_loglog.png} 
%\centering
%\label{}
%\end{figure}

%\begin{figure}[h]
%\caption{Log Log Plot of Tail Distribution - Linear Model}
%\includegraphics[width=.8\textwidth]{Linear_loglog.png} 
%\centering
%\label{}
%\end{figure}

%\begin{figure}[h]
%\caption{Log Log Plot of Tail Distribution - Low Regime}
%\includegraphics[width=.8\textwidth]{low_loglog.png} 
%\centering
%\label{}
%\end{figure}

%\begin{figure}[h]
%\caption{Log Log Plot of Tail Distribution - Normal Regime}
%\includegraphics[width=.8\textwidth]{norm_loglog.png} 
%\centering
%\label{}
%\end{figure}

%\begin{figure}[h]
%\caption{Log Log Plot of Tail Distribution - Crisis Regime}
%\includegraphics[width=.8\textwidth]{crisis_loglog.png} 
%\centering
%\label{}
%\end{figure}





\begin{figure}[h]
\caption{Log Log Plot of Tail Distribution - All Models}
\includegraphics[width=.5\textwidth]{log_log_all.png} 
\centering
\label{}
\end{figure}



\begin{figure}[H]
\centering
\begin{subfigure}{.5\textwidth}
  \centering
  \includegraphics[width=1\linewidth]{Crisis_qq.png}
  \caption{Crisis Model - Right Tail}
  \label{fig:sub1}
\end{subfigure}%
\begin{subfigure}{.5\textwidth}
  \centering
  \includegraphics[width=1\linewidth]{crisis_qqleft.png}
  \caption{Crisis Model - Left Tail}
  \label{fig:sub2}
\end{subfigure}
\caption{Crisis Model QQ Plot}
\label{fig:crisisqq}
\end{figure}


\begin{figure}[H]
\centering
\begin{subfigure}{.5\textwidth}
  \centering
  \includegraphics[width=1\linewidth]{normal_qq.png}
  \caption{Normal Model - Right Tail}
  \label{fig:sub1}
\end{subfigure}%
\begin{subfigure}{.5\textwidth}
  \centering
  \includegraphics[width=1\linewidth]{normal_qqleft.png}
  \caption{Normal Model - Left Tail}
  \label{fig:sub2}
\end{subfigure}
\caption{Normal Model QQ Plot}
\label{fig:normalqq}
\end{figure}


\begin{figure}[H]
\centering
\begin{subfigure}{.5\textwidth}
  \centering
  \includegraphics[width=1\linewidth]{low_qq.png}
  \caption{Low Model - Right Tail}
  \label{fig:sub1}
\end{subfigure}%
\begin{subfigure}{.5\textwidth}
  \centering
  \includegraphics[width=1\linewidth]{low_qq_left.png}
  \caption{Low Model - Left Tail}
  \label{fig:sub2}
\end{subfigure}
\caption{Low Model QQ Plot}
\label{fig:lowqq}
\end{figure}


\begin{figure}[H]
\centering
\begin{subfigure}{.5\textwidth}
  \centering
  \includegraphics[width=1\linewidth]{conditional_qq.png}
  \caption{Conditional Model - Right Tail}
  \label{fig:sub1}
\end{subfigure}%
\begin{subfigure}{.5\textwidth}
  \centering
  \includegraphics[width=1\linewidth]{cond_qqleft.png}
  \caption{Conditional Model - Left Tail}
  \label{fig:sub2}
\end{subfigure}
\caption{Conditional Model QQ Plot}
\label{fig:Conditionalqq}
\end{figure}



\begin{figure}[H]
\centering
\begin{subfigure}{.5\textwidth}
  \centering
  \includegraphics[width=1\linewidth]{ar_qqright.png}
  \caption{Linear Model - Right Tail}
  \label{fig:sub1}
\end{subfigure}%
\begin{subfigure}{.5\textwidth}
  \centering
  \includegraphics[width=1\linewidth]{ar_qqleft.png}
  \caption{Linear Model - Left Tail}
  \label{fig:sub2}
\end{subfigure}
\caption{Linear Model QQ Plot}
\label{fig:Linearqq}
\end{figure}






\chapter{Volatility Tails - Actual and Expected Volatility \\ \small By Chandler Clemons}

\makeevenhead{plain}{}{}{\textit{}}
\makeoddhead{plain}{\textit{}}{}{}

\begin{abstract}
The presence of slowly decaying tails signals a system susceptible to unpredictable and consequential events. In such cases where fat tails are identified, typical values such as the average and variance, do not properly characterize the risk and unpredictability of the dynamic process under study. Prior research has identified asset prices and asset volatility as being drawn from a power law distribution. This paper aims to quantitatively confirm this characterization, specifically for market volatility. Further, this paper identifies whether or not volatility expectations exhibit similar power law characteristics. Goodness of fit and log likelihood tests indicate that most realized volatility series are plausibly drawn from a power-law distribution. However, none of the studied implied volatility series show evidence of power-law behavior, suggesting that risk premia may exist for lower levels of volatility but does not scale proportionally to the more extreme crisis events. That is, risk premia does not scale proportionally as values move farther into the tail.

%The presence of slowly decaying tails signals a system susceptible to unpredictable and consequential events. In such cases where fat tails are identified, typical values such as the average and variance, do not properly characterize the risk and unpredictability of the dynamic process under study. Prior research has identified asset prices and asset volatility as being drawn from a power law distribution. This paper aims to quantitatively confirm this characterization, specifically for market volatility. Further, this paper identifies whether or not volatility expectations exhibit similar power law characteristics. I estimate the tail index ($\alpha$) of each series using Maximum Likelihood Estimation (Newman 2005), and subsequently identify if the tail decay can plausibly be described as following a power law using nonparametric bootstrap methodology. Goodness of fit and log likelihood tests indicate that most realized volatility series are plausibly drawn from a power-law distribution. However, none of the studied implied volatility series show evidence of power-law behavior, suggesting that risk premia may exist for lower levels of volatility but does not capture the "Black Swan" events. That is, risk premia does not scale proportionally as values move farther into the tail.
\end{abstract}

\clearpage


\section{Introduction}
\makeevenhead{plain}{Chapter 2}{}{\textit{Clemons}}
\makeoddhead{plain}{\textit{Clemons}}{}{Chapter 2}

Most everything of interest and consequence occurs far away from normalcy. As social scientists or students of history, we don't typically chronicle the normal, the day-to-day, or the expected. We care more about the events that disrupt our sense of equilibrium. These are the instances that cause ruin, riches, war, famine, and other consequences that challenge our notions of status quo and understanding. A natural question thus arises, how well do individuals, or markets, predict such events? A quantitatively advantageous way to investigate this question is through data rich financial markets, where participants implicitly, or explicitly, bet on future states of the world. In short-term derivative markets, these bets form repeated samples of the perceived likelihood of large price fluctuations. By capturing these expectations, we can attempt to quantify any differences between the likelihood of out-of-the-ordinary events and the \textit{expected} likelihood of out-of-the-ordinary events. 

We have, however, a relatively limited knowledge and tool set for answering this question quantitatively. This is particularly true of economic sciences and financial economics. Although Benoit Mandelbrot provided methodology for quantifying the non-Gaussian properties of asset returns in 1963, the statistical consequences of far-from-equilibrium dynamics were not fully appreciated and explored within the academic economics literature until much later. Even as of this writing, such methods and approaches are not widely studied, taught, or practiced within economics. The aim of this work is to add to the growing literature on the consequences of rare economic events and overcome the typical measurement issues that impinge the quantitative analysis of a distribution's tail behavior. In doing so, this work's higher goal is to look closely at the statistical properties of both historical volatility and expected volatility, because while we now have tools to identify a distribution's tail behavior, we have yet to understand how expectations of extreme events compare to the realization of extreme events. Specifically, the objective is to quantitatively identify the differences between actual tail risk and expected tail risk is asset markets.


Characterizing the tail behavior of a random variable's distribution has attracted increasing scientific interest. Of particular interest are "fat", "heavy", or "long" tailed distributions, which characterize a wide range of both natural and man-made phenomena. The presence of slowly decaying tails signals a system susceptible to unpredictable and consequential events. In such cases, typical values such as the average and variance, do not properly characterize the risk and unpredictability of the dynamic process under study. This is particularly true in finance, where major transfers of wealth are the result of tail events, not minor fluctuations around the mean. This fact is not fully captured by classical finance or econometric theory, and modern attempts to address the breakdown of Gaussian assumptions have proved unsatisfactory. For example, Value-at-Risk models (VaR) used for risk management help the practitioner set limits for their exposure based on historical data and the assumed likelihood of far from equilibrium events. These methods can even be modified to control for heteroskedasticity often observed in financial time series. Under long tailed distributions, however, this concept breaks down and may even \textit{increase} risk if the practitioner is overconfident in the precision of their VaR limits. The reason for the breakdown is that long tailed variables have the following property (Asmussen, S. R., 2003):

\[\lim_{x\to\infty} Pr[\,X > x + t \, \vert \, X > x\,] = 1 \quad  \forall \quad  t > 0\]

That is, for long tailed distributions with certain characteristics, we cannot predict any one tail event's location within the tail. Once an observations is beyond some threshold, the probability approaches 1 that it will exceed any other higher level. For example, once a virus or some other disease reaches a certain level of connectivity, say 10,000 cases, it is not possible a priori to predict if the ensuing pandemic will be mild or disastrous. All we can say is that it is a tail event and should therefore be handled with caution. This is because the a priori characteristic markers of a mild pandemic are the same as one that is devastating. The same is true of market corrections and crashes; the initial properties of a downturn are scale invariant, meaning that the initial properties of a mild correction are the same as those of a major crash. This scale invariance is a key marker of power law distributions. Therefore, the identification of tail behavior is important for understanding, rather than predicting, the scale of the risk that may hide in the tail. For the purposes of this paper, it is crucial to identify whether or not expected events are drawn from the same class of distribution as realized events. 

The methodology, drawn from extreme value theory, that is employed on the data in this study indicates that realized tail risk is drawn from a power law distribution, while expected tail risk is not. This is the case for all major large cap U.S. equity indices under examination, including S&P 500, NASDAQ, and Dow Jones Industrial Average. This phenomenon may indicate a systematic underprediction of large price dispersions or that risk premium does not scale proportionally as values move farther into the tail. Finally, the results show that the bootstrapping procedures used in this study help alleviate model sensitivity to slight changes in the tail threshold value that is often found when using alternative procedures. The procedure used in the paper may therefore provide a better method for studying and identifying how extreme event impact economic systems. 

The rest of the paper is organized as follows. Section 2 discusses power laws, scale invariance, and egodicity. Section 3 introduces the data and discusses the derivation of both implied volatility and the estimation procedure for realized volatility. Section 4 outlines the empirical and procedural steps used to investigate the hypothesis. Section 5 presents the empirical results and interprets the findings in the context of the research question. Finally, the concluding remarks are presented in Section 5.

\section{Power Laws, Scale Invariance, and Egodicity}

Why is it significant that realized volatility follows a power law and expected volatility does not? Fat tailed distributions, particularly power law distributions, are thought to be the signature of a complex and self-organizing system (Bak 1996; Sornette 2003). Mandelbrot (1963) first noted that power laws appear to characterize the distribution of financial variable fluctuations. Additionally, Lux (1996), Guillaume, et al (1997), and Gopikrishnan, Plerou, Gabaix, Stanley (2000) show that power laws characterize a number of relevant financial returns, including index prices, individual company prices, foreign exchange markets, and trade volume. Surprisingly, the exponents that characterize these power laws are similar for different types and sizes of markets, for different market trends and even for different countries—suggesting that a generic theoretical basis may underlie these phenomena (Gabaix, Gopikrishnan, Plerou et al., 2003). Liu, Y. et al (1999) demonstrates that the asymptotic behavior of SP 500 realized volatility is described by a power law distribution characterized by an exponent $1 + \mu \approx 4$. There is, therefore, a precedent literature suggesting that realized volatility has the scale invariant and self-organizing properties that describe power law distributions. This paper aims to quantitatively confirm this characterization, while also seeking to understand if the market's expectations of volatility match the power law description of volatility realizations. 

This growing literature which documents and discusses the idea that financial market fluctuations do not obey the Gaussian assumptions has been both experienced and appreciated. There are numerous historical events, such as 19 October 1987 (Black Monday), that simply should not occur under a normal distribution. These large-scale events happen far too often to be thought of simply as outliers. In fact, this paper aims to expand the literature claiming that these "outliers" constitute the most economic and statistically significant information in our economic histories. The effects of economic and market shocks are far-reaching, long-lasting, and recurrent, and therefore understanding the likelihood of such events and learning how to deal with them must be taken seriously. Further, the presence of fat tails in the distribution of price and volatility fluctuations (i.e. the high likelihood of shocks) suggest the the proper approach to risk management lies within this domain is Extreme Value Theory (EVT), which is concerned with phenomena in which extremes are the fundamental source of risk rather than averages. For the purposes of this paper, the discussion on so-called fat tails takes two perspectives; 1) the Informational Consequences, and 2) Economic Consequences. 

%Because big economic shocks affect the economy around the world (‘everything depends on everything else’), the possibility of an economic ‘earthquake’ is one that we must take seriously. Big changes in stocks affect not only people with large amounts, but also those on the margins of society. One person’s portfolio collapse is another’s physical starvation; e.g., literal starvation in some areas was one result of the recent Indonesian currency collapse.Another example is the recent Merriwether LTCM (long-term capital management) collapse, caused in part by the use of models that do not take into account these catastrophic rare events. Thus there are many reasons that we physicists might be interested in understanding economic fluctuations.

\subsection{Informational Consequences and Moment Finiteness}

Broadly speaking, the fatter or longer the tail, the more a system’s properties are determined by the exceptional and less so by the body of the distribution. As the tail thickens, the body becomes less informationally significant for guiding inference. For example, under fat tailed distributions, the law of large numbers works slowly, and moments — even when they exist — may become uninformative and unreliable (Taleb, N. N. Statistical Consequences of Fat Tails (STEM Academic Press, 2020)). A useful visual approximation of the finiteness of the statistical moments is the Maximum-to-Sum plot, or MS Plot. The MS Plot relies on simple consequence of the law of large numbers (P. Embrechts, C. Klüppelberg, T. Mikosch (2003). Modelling Extremal Events. Springer.). For a sequence $X_1, X_2, \dots, X_n$ of nonnegative i.i.d. random variables, if for p = 1, 2, 3..., $E[X^p] < \infty$, then $R_n^p = \frac{M_n^p}{S_n^p}  \overset{a.s.}{\to} 0$ as $n \to \infty$, where $S_n^p = \sum_{i=1}^{n} =  X_i^p$ is the partial sum, and $M_n^p = max(X_1^p,..., X_n^p)$ the partial maximum. The ratio is a simple tool for detecting heavy tails of a distribution and for giving a rough estimate of the order of its finite moments. Sharp increases in the curves of a MS Plot are a sign for heavy tail behavior, and convergence to zero indicate that moment of order P is finite. The plots below indicate that the \textit{expected} volatility series show less evidence of heavy tail behavior than \textit{actual} volatility series. Further, the plots for actual \textit{actual} volatility suggest that the third and fourth moments may not be finite, particularly for the S\&P 500, Dow Jones Industrial Average, and Nasdaq. 


\begin{figure}[H]
\centering
\begin{subfigure}{.5\textwidth}
  \centering
  \includegraphics[width=1\linewidth]{MSplotVIX.png}
  \caption{SP500 VIX MS Plot}
  \label{fig:sub1}
\end{subfigure}%
\begin{subfigure}{.5\textwidth}
  \centering
  \includegraphics[width=1\linewidth]{MSplotSP500.png}
  \caption{SP500 VOL MS Plot}
  \label{fig:sub2}
\end{subfigure}
\caption{SP500}
\label{fig:maxsumratio1}
\end{figure}

\begin{figure}[H]
\centering
\begin{subfigure}{.5\textwidth}
  \centering
  \includegraphics[width=1\linewidth]{MSplotDJIAvix.png}
  \caption{DJIA VIX MS Plot}
  \label{fig:sub1}
\end{subfigure}%
\begin{subfigure}{.5\textwidth}
  \centering
  \includegraphics[width=1\linewidth]{MSplotDJIA.png}
  \caption{DJIA VOL MS Plot}
  \label{fig:sub2}
\end{subfigure}
\caption{Dow Jones Industrial Average}
\label{fig:maxsumratio2}
\end{figure}


\begin{figure}[H]
\centering
\begin{subfigure}{.5\textwidth}
  \centering
  \includegraphics[width=1\linewidth]{MSplotNDXvix.png}
  \caption{NDX VIX MS Plot}
  \label{fig:sub1}
\end{subfigure}%
\begin{subfigure}{.5\textwidth}
  \centering
  \includegraphics[width=1\linewidth]{MSplotNDX.png}
  \caption{NDX VOL MS Plot}
  \label{fig:sub2}
\end{subfigure}
\caption{Nasdaq}
\label{fig:maxsumratio3}
\end{figure}


\begin{figure}[H]
\centering
\begin{subfigure}{.5\textwidth}
  \centering
  \includegraphics[width=1\linewidth]{MSplotRUTvix.png}
  \caption{Russell VIX MS Plot}
  \label{fig:sub1}
\end{subfigure}%
\begin{subfigure}{.5\textwidth}
  \centering
  \includegraphics[width=1\linewidth]{MSplotRUT.png}
  \caption{Russell VOL MS Plot}
  \label{fig:sub2}
\end{subfigure}
\caption{Russell 2000}
\label{fig:maxsumratio4}
\end{figure}


\subsection{Economic Consequences and Ergodicity}

The Economic Consequences of Fat Tails refers to the fact that market and economic shock occur with relatively high periodicity. In context of this paper, it also refers to the fact markets appear to place a lower probability on extreme volatility events than what would be considered sufficient ex post. For market and economic participants, whose payoffs are non-ergodic, this can mean economic ruin when caught wrong-footed. A non-ergodic system refers to one in which the time average does not equal the ensemble average. For example, averaging over many systems (i.e. many portfolios) is different than the average performance of a single system (i.e. a single portfolio) through time. A single entity's time average can go to 0 and stay at zero when confronted with an extreme event. However, a group of independently and identically distributed entities can survive that same event collectively. For macroeconomies, especially in a global context, an under-appreciation of tail risk and the consequences of non-ergodicity can have devastating and far-reaching effects. This is particularly true when seemingly uncorrelated entities become correlated during market crashes. Therefore, understanding how our expectation of risk differs from actual risk is paramount.

\subsection{Universality and Scale Invariance}

Another way to interpret power law behavior is through the concept of universality, which expresses the idea that different microscopic physics can give rise to the same scaling behaviour at a phase transition. For example, the likelihood of an avalanche in a sand pile is in power-law proportion to the size of the avalanche, and avalanches are seen to occur at all size scales. This is also true of financial markets, where severe crashes appear to be nothing more than a mild crash that doesn't stop. This similarity in the characteristics of mild and severe crashes makes it difficult, if not impossible, to predict the concluding severity once a crash has begun, making the identification of power law scaling behavior important for determining the appropriate response once one has identified a tail event. As mentioned in the introduction, power laws typically signal scale invariance, therefore scaling the argument by a constant factor causes only a proportionate scaling of the function itself. Just as a small earthquake is the same as a massive earthquake is every way expect size, power law behavior in market volatility indicates that moderate price variation is similar to extreme price variation except in magnitude. 




%We argue that a natural and empirically correct framework for assessing (and managing) the real risk of pandemics is provided by extreme value theory (EVT), an approach that has historically been developed to treat phenomena in which extremes (maxima or minima) and not averages play the role of the protagonist, being the fundamental source of risk.





\section{About the Data}

Because volatility is a statistical measure and not a pre-existing data point, such as quoted prices, it's value must be derived. There are several ways to complete this calculation, each with its own intended purpose and relative advantages, however each aims captures a sense of dispersion of asset price returns for a given security or index. Within this broad context there are two main ways to approach the measuring of volatility, historical volatility and implied volatility. Historic volatility measures a time series of past market prices, while implied volatility looks forward in time and is derived from the market price of a market-traded derivative (in particular, an option). Both capture a sense of market risk and uncertainty, albeit in different temporal direction. It is therefore useful to think about historical volatility as \textit{realized} volatility and implied volatility as \textit{expected} volatility. Further, because options-implied volatility, such as the VIX, is a model free calculation, such indices provide a useful aggregation of market opinion concerning future uncertainty. VIX indices aggregate call and put options in a way such that the index represents the implied volatility that results from the clearing prices of all equity options for a particular market index (e.g. S\&P 500). Implied volatility in this sense is the amount of volatility required to set the options’ expected value equal to zero, given the contracted prices. VIX variables exist for several major market indices, most of which are explored in this paper, include S\&P 500 VIX, Dow Jones Industrial Average (DJIA) VIX, Nasdaq (NDX) VIX, and Russell 2000 (RUT) VIX. Those VIX indices that were excluded were done so due to short sample periods, particularly where the sample did not contain significant market events.

Measures of historical volatility can be estimated in several different ways, such as a rolling standard deviation, the local average of absolute price changes over a suitable time interval T, or using a generalized autoregressive conditional heteroscedasticity (GARCH) model. This paper relies on a GJR-GARCH model, which offers what vanilla GARCH has to offer, plus the leverage effect. This more flexible GARCH variant is preferred since the goal is to fit in-sample historical volatility as precisely as possible. In this case, overfitting the data in not a concern, and is perhaps preferred, since this model is not intended for predictions but rather for emulating the past. The GJR-GARCH (1,1) model is as follows:

\begin{center}
    

$\sigma^2_t=\omega+(\alpha+\gamma I_{t})\epsilon^2_{t-1}+\beta\sigma^2_{t-1}$
\vspace{}
where I_{t} = 
\begin{cases}
      0 & \text{if}\ r_{t-1} \geq \mu \\
      1 & \text{if}\ r_{t-1} < \mu 
    \end{cases}

\end{center}


Because the main goal of this work is to assess potential differences in the tail behavior between realized and expected volatility, it is important that apples are compared to apples. That means that each market index volatility and its corresponding VIX index should have matching sample windows by using the greatest \textit{common} data period. For example, S\&P 500 volatility is available from 1993-02-01, while the S\&P 500 VIX index is available from 1990-01-02. In order to match the data periods, the S\&P 500 VIX series was truncated such that the first observation occurs on 1993-02-01. This same approach was taken for each specific index, as shown in the table below. 


 \begin{table}[H] \centering 
  \caption{Sample Data} 
  \label{} 

\begin{tabular}{@{\extracolsep{5pt}} ccccccc} 
\\[-1.8ex]\hline 
\hline \\[-1.8ex] 
 
Index & Dates & n   & Mean & Median & Min & Max\\ 
\hline \\[-1.8ex] 
VIX & 1993-02-01 - 2020-09-02 & 6919  & 19.47  & 17.34 & 9.14 & 82.69 \\ 
SP500 VOL & 1993-02-01 - 2020-09-02 & 692  & 16.44 & 13.63 & 6.65 & 116.06  \\ 
\\[-1.8ex]\hline 
\hline \\[-1.8ex] 
DJIA VIX & 1997-10-07 - 2020-09-02 & 5764   & 19.43 & 17.42 & 7.58 & 74.60\\ 
{DJIA VOL} & {1997-10-07 - 2020-09-02} & {5764}   & 16.61  & 13.86  & 7.17 & 120.32 \\  
\\[-1.8ex]\hline 
\hline \\[-1.8ex] 
Nasdaq VIX & 2001-02-02 - 2020-09-02 & 4925   & 24.68 & 20.20 & 10.31 & 80.64 \\ 
{Nasdaq VOL} & {2001-02-02 - 2020-09-02} & {4925}   & 22.59 & 18.15 & 8.74 & 113.56\\ 
\\[-1.8ex]\hline 
\hline \\[-1.8ex] 
Russell 2000 VIX & 2004-01-02 - 2020-09-02 & 4195   & 24.06 & 20.90 & 11.83  & 87.62\\ 
Russell 2000 VOL & 2004-01-02 - 2020-09-02 & 4195    & 21.56 & 17.45 & 7.26 & 146.11\\ 
\hline \\[-1.8ex] 
\end{tabular} 

\end{table} 


Figure~\ref{fig:Densities - Expected vs Actual Vol} below further depict each historical volatility series along with its implied volatility counterpart. Density plots visually indicate that the center mass of the implied volatility series are shifted right relative to historical volatility. In fact, for each index studied, the mean and median values are always greater for implied volatility relative to its historical counterpart. This phenomenon may be interpreted as a risk premium to cover the downside risk faced by option sellers. However, I argue that this conclusion is not complete unless the significance of extreme tail events present within the actual volatility series are weighed and considered. 


 \begin{figure}[H]
        \centering
        \begin{subfigure}[b]{0.475\textwidth}
            \centering
            \includegraphics[width=\textwidth]{spy_density.png}
            \caption[]%
            {{\small SP500}}    
            \label{fig:SP500 Density}
        \end{subfigure}
        \hfill
        \begin{subfigure}[b]{0.475\textwidth}  
            \centering 
            \includegraphics[width=\textwidth]{djia_density.png}
            \caption[]%
            {{\small DJIA}}    
            \label{fig:DJIA Density}
        \end{subfigure}
        \vskip\baselineskip
        \begin{subfigure}[b]{0.475\textwidth}   
            \centering 
            \includegraphics[width=\textwidth]{ndx_density.png}
            \caption[]%
            {{\small NDX}}    
            \label{fig:NDX Density}
        \end{subfigure}
        \hfill
        \begin{subfigure}[b]{0.475\textwidth}   
            \centering 
            \includegraphics[width=\textwidth]{rut_density.png}
            \caption[]%
            {{\small RUT}}    
            \label{fig:RUT Density}
        \end{subfigure}
        \caption[ Densities - Expected vs Actual Vol ]
        {\small Densities - Expected vs Actual Vol} 
        \label{fig:Densities - Expected vs Actual Vol}
    \end{figure}

Financial time series data has the advantage of being abundant and readily available, which is why it has attracted researchers from an abundance of disciplines, particularly where the objective is to study a complex system with a large and consistent dataset. The data used in this paper comes from two sources. First, all VIX variables were sourced from the Federal Reserve Bank of St. Louis FRED database, which retrieves data from the Chicago Board Options Exchange for each Implied Volatility Index. Second, all price variables were sourced from Yahoo! Finance through the open API in R programming software. The data was then merged together using common dates. No further data preparation was required.


\section{Tail Index Estimation}

Determining that the observations within the distribution's tail are drawn from a power law is non-trivial and often contains high level of researcher judgement. Much of the subjectivity is related to selecting the location of the tail, the threshold value above which all observations are considered to be part of the distribution's tail. Further, a line can be fit to any set of data. The question is whether or not the line provides an appropriate description of the data. This has posed a challenge for researchers seeking to quantitatively identify power law behavior. To overcome these issues as best as possible, I use the methodology of Clauset et al (2009), which relies on MLE estimate and bootstrapping for parameter and tail estimation. First, the bootstrapping procedure provides a more quantitatively informed selection of the tail threshold than alternative methods. This helps eliminate parameter sensitively due to slight variation in the threshold value. Second, a calculation of the goodness-of-fit between the data and the power law fit quantifies the distance between the distribution of the empirical data and the hypothesized model. The Kolmogorov-Smirnoff (KS) statistics is compared with distance measurements for comparable synthetic data sets drawn from the same model, and the p-value is defined to be the fraction of the synthetic distances that are larger than the empirical distance. If p is large (close to 1), then the difference between the empirical data and the model can be attributed to statistical fluctuations alone; if it is small, the model is not a plausible fit to the data (Clauset et al, 2009). This helps eliminate a priori biases about the distribution's class. Finally, because this, or any other methodology to my knowledge, does not allow one to be certain that an observed quantity is drawn from a power-law distribution, I compare the power law to alternative hypotheses via a likelihood ratio test.  

\subsection{Separating the Tail from the Body}

Hill (1975) suggested that, for certain situations, it would be of interest to draw inference about the behavior of a distribution function in the tails without assuming that a particular parametric form for the distribution function holds globally. However, this requires that we first identify the location of the tail. That is if $\epsilon_1, \epsilon_2,\dots, \epsilon_k$ is a sample drawn from a population with distribution G and $\epsilon^{(1)} \ge  \epsilon^{(2)} \ge \dots \ge \epsilon^{(k)}$ are the order statistics, than there may exist some number $x_{min}$ such that $\epsilon\ge x_{min}$ defines the region where G is believed to form a Pareto distribution. Because $x_{min}$ is not known in practice, the primary challenge when estimating tail indices is determining the precise location of the tail. For example, in this context we are concerned with volatility above a desired threshold. It is easy to see from a visual inspection of the variable's distribution that subtle deviations from the true $x_{min}$ can bias the estimated tail index. In essence, moving the threshold value higher or lower can drastically alter the shape of the tail being estimated. As such, the validation of power-law claims is an active field of scientific research. The approach taken in this paper aims to alleviate such issues, whereby $x_{min}$ is estimated by minimising the Kolmogorov-Smirnoff statistic between the CDFs of the data and the fitted model. 

\begin{center}
    
$D= \max_{\{x \geq x_{min}\}} |S(x) - P(x)|$
\end{center}

\begin{flushleft}
Here $S(x)$ is the CDF of the data for the observations with value at least $x_{min}$, and $P(x)$ is the CDF for the power-law model that best fits the data in the region $x \ge x_{min}$. Further, I make use of a nonparametric bootstrap method to derive principled estimates of the uncertainty in the originally estimated parameters. 

\end{flushleft}


Again, determining the threshold value is critical. If $x_{min}$ is chosen to be too high, the variance of the estimator increases. If $x_{min}$ is too low, the bias of the estimator increases. A standard method for finding $x_{min}$ is to apply the Hill estimator on the entirety of the ordered sample population and determine the maximum value for $x_{min}$ such that the weighted Hill estimate converges on a single number. Beyond this number, $x_{min}^\ast$, the weighted Hill estimate will diverge, suggesting that the inclusion of data beyond $x_{min}^\ast$ produces a sample that is not Pareto distributed. Another method for finding $x_{min}^\ast$ is to start with a density plot of sorted data to visually identify a lower bound and ‘strict power law’ behavior. Further heuristics such as log-log plots and log-log rank plots and can identify a linear relationship to identify power law behavior, though these methods are not fool proof (Nair et al. 2019). In Hubert et al. (2013) there is another Pareto test which looks for linear behavior in the QQ-plots of the log transformed data against standard exponential data. The primary method used in this paper is compared to the aforementioned standard estimation procedures, such as the Hill Estimator and visual identifiers. Each method produces similar results, however bootstrapping validates parameter estimations and alleviates threshold sensitivity, specifically for the Nasdaq historical volatility series. 


\subsection{Estimating the Exponent $\alpha$}
   
We want to characterize the conditional probabilities of each series such that the volatility measure exceeds some minimum threshold, $x_{min}$. Assuming that the family of extreme value distributions falls into the Frechet type, volatility above this threshold will follow,

$$
P(X_t > x \ | \ X_t > x_{min}) = k|x|^{-\alpha}
$$

\begin{flushleft}

where $X_t$ is the measure of volatility on day $t$, $x_{min}$ is the minimum threshold that defines the location of the tail, and the parameter $\alpha$, the tail exponent, determines the rate at which the probability density drops off as one moves out into the tail (LeBaron 2009). 
\end{flushleft}


Once the $x_{min}^\ast$ threshold has been identified in the manner described above, the tail index, $\alpha$, can be estimated using the Maximum Likelihood Estimation (MLE) method of Newman (2005), which produces a biased estimate of $\hat{\alpha}$:

\begin{center}
$\hat{\alpha}= N\cdot [\sum_{i=1}^{N} \dfrac{x_i}{x_{min}^\ast}]^{-1}$
\end{center}

\begin{flushleft}

This MLE estimate can be converted to an unbiased version $\alpha^{\ast}$ following Rizzo (2009) with the following adjustment, where $n$ is the number of observations in the sample: 
\end{flushleft}

\begin{center}
$\alpha^{\ast}= \dfrac{{n-2}}{n}\cdot\hat{\alpha}$
\end{center}

\subsection{Validating the Power Law Hypothesis}

In order to validate the power law hypothesis, it is critical that the approach taken above is not simply fitting a naive line to a set of data points. Nor should one presume that a proposed power law is the best fit amongst other candidate distributions. These hypotheses should be validated, which I have done using the bootstrap procedure from Clauset et al (2009). Using this approach, goodness of fit (GOF) between the data and the power law fit is measured using the Kolmogorov-Smirnov (KS) statistic. The KS statistics is compared with distance measurements for comparable synthetic data sets drawn from the same model. This procedure defines a p-value to test the power law hypothesis, which is defined to be the fraction of the synthetic distances that are larger than the empirical distance. If p is large (close to 1), then the difference between the empirical data and the model can be attributed to statistical fluctuations alone; if it is small, the model is not a plausible fit to the data. Further, I compare the power law with the alternative hypotheses via a likelihood ratio test. Both the significance level of the ratio test and the sign of the test statistic indicate whether the alternative is favored over the power-law model or not. Positive values indicate that a power-law model is favored, and vice versa. For instances where the likelihood ratio test statistic is not significant, we cannot make a conclusion as to whether the power law or the alternative distribution is a more appropriate fit. 


\section{Empirical Results}


Table~\ref{table:EmpiricalResults1} presents both the MLE results and the Bootstrap results for each time series. Blue rows signify series where the p-value for the power law hypothesis is significant ($p \ge .10$), thus making it a possible fit for the data. This is found to be the case for the historical volatility of S\&P 500, DJIA, and Nasdaq. Conversely, the power law hypothesis does not appear to be a valid characterization for any implied volatility series, suggesting some fundamental differences between realized and expected market risk. 

There are some additional insights to be made from Table~\ref{table:EmpiricalResults1}. First, differences between the MLE and bootstrap results appear to be minor in all cases except Nasdaq's historical volatility. The disparity between the MLE and the Bootstrap results of this particular series demonstrates the usefulness of bootstrapping to alleviate threshold sensitivity. Second, it is interesting, and perhaps useful, to understand what the empirical procedure considers to be the tail of the distribution. For those series that show evidence of power law behavior, the threshold value, $x_{min}$, is not too dissimilar from the mean of the series. This finding suggests that once risk surpasses its average value, it displays the scale invariant and universality properties that characterize power law behavior. Finally, for the historical volatility of SP500, DJIA, and Nasdaq, the tail exponents, $\alpha \approx 3$, are found to be outside the Levy regime ($0 < \alpha < 2$). However, the Nasdaq historical volatility tail index of $\alpha = 2.12$ is close to the range where it would lack a finite variance ($\alpha \leq 2$). Note, however, that $\alpha \approx 3$ suggests that the third and fourth moments of these series are not defined, which is also demonstrated by the Maximum/Sum Ratio plots in Figures~\ref{fig:maxsumratio1},~\ref{fig:maxsumratio2}, and~\ref{fig:maxsumratio3}. Variance of a finite variance random variable with tail exponent $<$ 4 will be infinite, which causes problems for stochastic volatility models when the real process can actually be of infinite variance (Taleb 2019).


 \begin{table}[H] \centering 
  \caption{Empirical Results} 
  \label{table:EmpiricalResults1} 
   \scalebox{.8}{
\begin{tabular}{@{\extracolsep{5pt}} ccc|ccc|cccc} 
\\[-1.8ex]\hline 
\hline \\[-1.8ex] 
 & & & \multicolumn{3}{c}{\textit{MLE Results}} & \multicolumn{4}{c}{\textit{Bootstrap Results}} \\ 
\cline{4-6} 
\cline{7-10} 
 & Dates & n & $\hat{x}_{min}$ & $\hat\alpha$ & $\hat{n}_{tail}$ &  $\hat{x}_{min}$ & $\hat\alpha$ & $\hat{n}_{tail}$ & p \\ 
\hline \\[-1.8ex] 
VIX & 1993-02-01 - 2020-09-02 & 6919 & 23.85 & 4.05 & 1542 & 24.13 & 3.9 & 1530 & 0.08  \\ 
\color{blue}{SP500 VOL} & \color{blue}{1993-02-01 - 2020-09-02} & \color{blue}{6919} & \color{blue}{21.38} & \color{blue}{3.06} & \color{blue}{1361} & \color{blue}{22.36} & \color{blue}{2.99} & \color{blue}{1261} & \textbf{\color{blue}{0.28}} \\ 
\\[-1.8ex]\hline 
\hline \\[-1.8ex] 
DJIA VIX & 1997-10-07 - 2020-09-02 & 5764 & 21.47 & 3.88 & 1853  & 21.46 & 3.95 & 1882 & 0 \\ 
\color{blue}{DJIA VOL} & \color{blue}{1997-10-07 - 2020-09-02} & \color{blue}{5764} & \color{blue}{18.91} & \color{blue}{2.93} & \color{blue}{1498} & \color{blue}{18.5} & \color{blue}{2.99} & \color{blue}{1611} & \textbf{\color{blue}{0.29}} \\  
\\[-1.8ex]\hline 
\hline \\[-1.8ex] 
Nasdaq VIX & 2001-02-02 - 2020-09-02 & 4925 & 15.07 & 2.1 & 4255  & 15.05 & 2.09 & 4283 & 0  \\ 
\color{blue}{Nasdaq VOL} & \color{blue}{2001-02-02 - 2020-09-02} & \color{blue}{4925} & \color{blue}{73.43} & \color{blue}{9.77} & \color{blue}{59} & \color{blue}{17.7} & \color{blue}{2.12} & \color{blue}{2606} & \textbf{\color{blue}{0.9}} \\ 
\\[-1.8ex]\hline 
\hline \\[-1.8ex] 
Russell 2000 VIX & 2004-01-02 - 2020-09-02 & 4195 & 18.97 & 2.93 & 2680  & 17.88 & 2.95 & 3128 & 0 \\ 
Russell 2000 VOL & 2004-01-02 - 2020-09-02 & 4195 & 18.82 & 2.42 & 1811 & 17.35 & 2.46 & 2154 & 0.02 \\ 
\hline \\[-1.8ex] 
\end{tabular} 
}
\end{table} 

\begin{table}[H] \centering 
  \caption{Support for Power-law decay} 
  \label{table:EmpiricalResults2} 
   \scalebox{0.8}{ 
\begin{tabular}{@{\extracolsep{5pt}} ccccccc} 
\\[-1.8ex]\hline 
\hline \\[-1.8ex] 
 & Power Law & \multicolumn{2}{c}{Log Normal} & \multicolumn{2}{c}{Expoential} & Support for \\ 
 & p & LR & p & LR & p & Power Law \\ 
\hline \\[-1.8ex] 
VIX & 0.080 & -1.240 & 0.210 & 2.750 & \textbf{0.010} & none \\ 
\color{blue}{SP500 Realized VOL} & \color{blue}{\textbf{0.280}} & \color{blue}{-0.160} & \color{blue}{0.880} & \color{blue}{5.760} & \color{blue}{\textbf{0}} & \color{blue}{moderate} \\ 
\\[-1.8ex]\hline 
\hline \\[-1.8ex] 
DJIA VIX & 0 & -2.760 & \textbf{0.010} & 0.180 & 0.860 & none \\ 
\color{blue}{DJIA Realized VOL} & \color{blue}{\textbf{0.290}} & \color{blue}{-0.970} & \color{blue}{0.330} & \color{blue}{4.680} & \color{blue}{\textbf{0}} &  \color{blue}{moderate}\\ 
\\[-1.8ex]\hline 
\hline \\[-1.8ex] 
Nasdaq VIX & 0 & -9.980 & \textbf{0} & -5.580 & \textbf{0} &  none \\ 
\color{blue}{Nasdaq Realized VOL} & \color{blue}{\textbf{0.900}} & \color{blue}{0.070} & \color{blue}{0.950} & \color{blue}{0.370} & \color{blue}{0.710} & \color{blue}{moderate} \\ 
\\[-1.8ex]\hline 
\hline \\[-1.8ex] 
Russell 2000 VIX & 0 & -4.130 & \textbf{0} & 1.040 & 0.300 &  none\\ 
Russell 2000 Realized VOL & 0.020 & -1.720 & \textbf{0.090} & 5.940 & \textbf{0} &  none\\ 
\hline \\[-1.8ex] 
\end{tabular} 
}
\end{table} 


Table~\ref{table:EmpiricalResults2} present the power law hypothesis p-values and the likelihood ratios for each alternatives hypothesis. p-values are also quoted for the significance of each of the likelihood ratio tests ($p \leq .10$). Where the p-value is significant, positive values of the log likelihood ratios indicate that the power-law model is favored over the alternative and vice versa. The final column of the table lists a judgment of the statistical support for the power-law hypothesis for each data set. “None” indicates data sets that are probably not power law distributed; “moderate” indicates that the power law is a good fit but that there are other plausible alternatives as well; “good” indicates that the power law is a good fit and that none of the alternatives considered is plausible. Given the p-value test statistics and likelihood ratio test, moderate power law support is found for most realized volatility series. The only exception is the Russell 2000 volatility series. This lack of power law behavior may be due to the relatively large number of constituent securities that make up the index (2000 securities vs. 500 securities), or the fact that the Russell 2000 is a small-medium cap index. Therefore further research can test whether the the power law volatility phenomenon is specific to large cap and concentrated indices. It is also clear from the results table that none of the studied implied volatility series (VIX) show evidence of power-law behavior.




\section{Concluding Remarks}

For certain long tailed distributions, once an observations is beyond some threshold, the probability approaches 1 that it will exceed any other higher level. Identifying this distributional behavior is crucial for risk management and for guiding policy at the onset of a crisis-like event. This work provided evidence that historical volatility shows support for power law or fat tailed distributions, which may invalidate the assumptions of the central limit theorem. Further, expected (implied) volatility does not show support for the power law hypothesis. While it may be the case the implied volatility has fat tail characteristics, it does appear to possess the same scale invariant and universality properties that plausibly explain the behavior of actual volatility. Visual evidence and summary statistics do however show that the distributional mass of implied volatility is higher than its historical counterpart, indicating the presence of risk premium at low to mild levels of market risk. However, this risk premium does not appear to scale proportionally as data moves into the tail of the distribution where events are the most consequential. 

There are, however, several limitations to the work. While there is evidence of power law behavior for large cap historical volatility, one can not be certain that this characterization is the most appropriate. An attempt was made to compare alternative hypotheses, which yielded inconclusive results. The evidence presented in this paper only provides support for, but does not confirm, scale invariance, universality, and other power law characteristics. And while maximum/sum ratio plots help confirm the fat tailedness of the historical volatility series, this method is a heuristic approach and not quantitative confirmation. Additionally, this work is dependent on the validity of the data used. The VIX calculation methodology can only attempt to aggregate option prices in a was such that the result is the best representation of market-implied volatility. For example, the VIX algorithm aggregates out-of-the-money call and put options, but stops once two calls/puts with consecutive strike prices are found to have a zero bid prices. That is, no calls (puts) with higher (lower) strikes are considered for inclusion once they are preceded by two zero bids. This technical detail may omit important deep-out-of-the-money bids during crisis periods and may limit the ability of the index to fully capture market sentiment during periods of extremely high volatility. Similarly, the derivation of historical volatility relies on an empirical model. The choice to use a GJR-GARCH model was a subjective decision which may alter the results of this analysis. However, it was chosen for its flexibility as well as its ability to capture instantaneous price variation, which I believe make it superior to other estimation procedures. 

While limited in some respects, the aim of this work is to contribute to the growing literature on risk management concerning heavy tailed variables by providing additional quantitative confirmation to previous findings. This same approach should be done for other financial variables, such as asset price returns, that have been characterized as power law distributed in order to fully understand the risk and consequences of financial and economic crises. Further, this work hopes to provide a useful contribution to the area of research concerning expectations. It is important that we understand the impact of outlier events, both in terms of the impact on our economic life and the way in which we anticipate their occurrences as economic agents. 



\clearpage

\begin{thebibliography}{39}

\bibitem{Asmussen, S. R. 2003}
Asmussen, S. R. "Steady-State Properties of GI/G/1". Applied Probability and Queues. Stochastic Modelling and Applied Probability (2003). 51. pp. 266–301.

\bibitem{Clauset, Cosma, Newman 2009}
Clauset, Aaron, Cosma Rohilla Shalizi, and M. E. J. Newman. "Power-Law Distributions in Empirical Data." SIAM Review 51.4 (2009): 661–703.

\bibitem{Gillespie 2015}
Gillespie, Colin. "Fitting Heavy Tailed Distributions: The poweRlaw Package." Journal of Statistical Software [Online], 64.2 (2015): 1 - 16. 

\bibitem{Mantegna and Stanley 1995}
R. N. Mantegna and H. E. Stanley, "Scaling behaviour in the dynamics of an economic index", Nature, vol. 376, no. 6535, pp. 46-49 (1995). 

\bibitem{Taleb 2019}
Taleb NN (2019). "The Statistical Consequences Of Fat Tails." Stem Academic Publishing, 2019.

\bibitem{Gabaix, Gopikrishnan, Plerou, et al 2003}
Gabaix, X., Gopikrishnan, P., Plerou, V. et al. "A theory of power-law distributions in financial market fluctuations." Nature 423, 267–270 (2003). 

\bibitem{Liu et al 1999}
Liu, Y. et al. "The statistical properties of the volatility of price fluctuations." Phys. Rev. E 60, 1390–1400 (1999)

\bibitem{Guillaume et al 1997}
Guillaume, D. M. et al. "From the bird's eye to the microscope: a survey of new stylized facts of the intra-daily foreign exchange markets." Fin. Stochastics 1, 95–129 (1997)

\bibitem{Mandelbrot 1963}
Mandelbrot, B. B. "The variation of certain speculative prices." J. Business 36, 394–419 (1963)

\bibitem{Lux 1996}
Lux, T. "The stable Paretian hypothesis and the frequency of large returns: an examination of major German stocks." Appl. Fin. Econ. 6, 463–475 (1996)

\bibitem{Gopikrishnan, Plerou, Amaral, Stanley 1999}
Gopikrishnan, P., Plerou, V., Amaral, L. A. N., Meyer, M. & Stanley, H. E. "Scaling of the distributions of fluctuations of financial market indices". Phys. Rev. E 60, 5305–5316 (1999)

\bibitem{Gopikrishnan, Plerou, Gabaix, Stanley 2000}
Gopikrishnan, P., Plerou, V., Gabaix, X. & Stanley, H. E. "Statistical properties of share volume traded in financial markets". Phys. Rev. E 62, R4493–R4496 (2000)

\bibitem{Hill 1975}
Hill, Bruce M. "A Simple General Approach to Inference About the Tail of a Distribution." Ann. Statist. 3 (1975), no. 5, 1163--1174. doi:10.1214/aos/1176343247. https://projecteuclid.org/euclid.aos/1176343247


\bibitem{Kearns, Pagan 1997}
Kearns, Phillip, and Adrian Pagan. "Estimating the Density Tail Index for Financial Time Series." The Review of Economics and Statistics, vol. 79, no. 2, 1997, pp. 171–175. JSTOR, www.jstor.org/stable/2951449.



\bibitem{Cont 2001}
R. Cont, "Empirical properties of asset returns: Stylized facts and statistical issues", Quantitative Finance, vol. 1, pp. 223- 236, feb 2001. 



\bibitem{Gabaix, Gopikrishnan, Plerou, Stanley 2003}
X. Gabaix, P. Gopikrishnan, V. Plerou, and H. E. Stanley, "A theory of power-law distributions in financial market fluctuations", Nature, vol. 423, pp. 267-270, May 2003. 

\bibitem{Gabaix 2009}
X. Gabaix, "Power Laws in Economics and Finance", Annual Review of Economics, vol. 1, no. 1, pp. 255-294, 2009. 

\bibitem{Begušić, Kostanjcar, Stanley, Podobnik 2018}
Begušić, Stjepan & Kostanjcar, Zvonko & Stanley, H. & Podobnik, Boris. "Scaling properties of extreme price fluctuations in Bitcoin markets". Physica A: Statistical Mechanics and its Applications. 510. (2018)

\bibitem{Durrieu, Grama, Jaunatre, Pham, Tricot 2018}
Durrieu G, Grama I, Jaunatre K, Pham Q, Tricot J. "extremefit: A Package for Extreme Quantiles." Journal of Statistical Software, *87*(12), 1-20. 2018 

\bibitem{Saez-Castillo, Prieto, Sarabia 2015}
Antonio Jose Saez-Castillo, Faustino Prieto and Jose Maria Sarabia. "ParetoPosStable: Computing, Fitting and Validating the PPS Distribution". R package version 1.1. 2015

\bibitem{Munasinghe, Kossinna, Jayasinghe, Wijeratne 2019}
Ranjiva Munasinghe, Pathum Kossinna, Dovini Jayasinghe and Dilanka Wijeratne. "ptsuite: Tail Index Estimation for Power Law Distributions". R package version 1.0.0. 2019

\bibitem{Mandelbrot 2001}
Mandelbrot, Benoit. "Scaling in Financial Prices: I. Tails and Dependence". Quantitative Finance. 1. 113-123. 2001

\bibitem{Whaley 2002}
Whaley, R., "On the return and risk of the CBOE Buy Write monthly index". J. Deriv. 10, 35-42. 2002

\bibitem{Sarwar 2012}
Sarwar, Ghulam, "Is VIX an investor fear gauge in BRIC equity markets?", Journal of Multinational Financial Management, Elsevier, vol. 22(3), pages 55-65. 2012

\bibitem{Jackwerth and Rubinstein 1996}
Jackwerth, J., and M. Rubinstein, "Recovering Probability Distributions from Option Prices", Journal of Finance 51, 1611-1631. 1996

\bibitem{Coval and Shumway 2001}
Coval, J., and T. Shumway, "Expected Option Returns", Journal of Finance 56, 983-1009. 2001

\bibitem{Bakshi & Kapadia 2003}
Bakshi, G., and N. Kapadia, 2003, \textit{Delta-hedged Gains and the Negative Volatility Risk Premium,} Review of Financial Studies 16, 527-566.

\bibitem{Doran, Banerjee, Peterson 2006}
Doran, James and Banerjee, Prithviraj and Peterson, David R., "Implied Volatility and Future Portfolio Returns". Journal of Banking and Finance, Vol. 31, October 2007. 

\bibitem{Ben-David and Moussawi 2018}
Ben-David, Itzhak & Moussawi, Rabih. "Do ETFs Increase Volatility?" The Journal of Finance. 2018

\bibitem{Cochran, Mansur, Odusami 2015}
Cochran, Steven J. & Mansur, Iqbal & Odusami, Babatunde, "Equity market implied volatility and energy prices: A double threshold GARCH approach", Energy Economics, Elsevier, vol. 50(C), pages 264-272. 2015

\bibitem{Yuan and Mitra 2016}
Yuan, Yuan and Mitra, Gautam, "Market Regime Identification Using Hidden Markov Models" (September 18, 2016

\bibitem{Fink, Klimova, Czado, Stöber 2017}
Holger Fink & Yulia Klimova & Claudia Czado & Jakob Stöber, "Regime Switching Vine Copula Models for Global Equity and Volatility Indices", Econometrics, MDPI, Open Access Journal, vol. 5(1), pages 1-38, January. 2017

\bibitem{Hesse and González-Hermosillo 2009}
Hesse, Heiko and González-Hermosillo, Brenda, "Global Market Conditions and Systemic Risk". IMF Working Papers, Vol. , pp. 1-22, 2009

\bibitem{Ying et al. 2011}
Ma, Ying et al. "A Portfolio Optimization Model with Regime-Switching Risk Factors for Sector Exchange Traded Funds." 2011

\bibitem{Husson and McCann 2011}
Husson, Tim and McCann, Craig J., "The VXX ETN and Volatility Exposure". PIABA Bar Journal, Volume 18, No. 2, 2011

\bibitem{Papanicolaou and Sircar 2014}
Papanicolaou, Andrew and Sircar, Ronnie, "A Regime-Switching Heston Model for VIX and S\&P 500 Implied Volatilities". Quantitative Finance, Volume 14, Issue 10, pp. 1811-1827, 2001 

\bibitem{Goutte, Ismail, Pham 2017}
Stéphane Goutte, Amine Ismail, Huyên Pham. "Regime-switching Stochastic Volatility Model : Estimation and Calibration to VIX options". 2017

\bibitem{Tsay 1998}
Tsay, R., "Testing and modeling multivariate threshold models". J. Am. Stat. Assoc. 93, 1188-1202. 1998

\bibitem{Durrieu, Grama, Pham, et al 2015}
Durrieu, G., Grama, I., Pham, Q. et al., "Nonparametric adaptive estimation of conditional probabilities of rare events and extreme quantiles". Extremes 18, 437–478, 2015 


\end{thebibliography}


\chapter{Volatility Overload  \\ \small By Minh Pham \& Chandler Clemons}

\makeevenhead{plain}{}{}{\textit{}}
\makeoddhead{plain}{\textit{}}{}{}

\begin{abstract}
    

We investigate how economic uncertainty, specifically stock market uncertainty, correlates to individuals’ life satisfaction. Using expected price volatility (VIX) as our anticipatory indicator and life satisfaction as our measure of utility, our hypothesis is built on the Anticipatory Utility framework, which suggests that people also derive utility from their beliefs. After accounting for associations with the unemployment rate and stock ownership, we find a positive relationship between the VIX and low self-reported life satisfaction. This analysis captures the contemporaneous effects of future beliefs and indicate that economic sentiment about the future plays an important role in individuals’ feelings about the present.

\end{abstract}

\clearpage

\section{Introduction}
\makeevenhead{plain}{Chapter 3}{}{\textit{Pham \& Clemons}}
\makeoddhead{plain}{\textit{Pham \& Clemons}}{}{Chapter 3}

Coping with uncertainty is a fundamental necessity of life. Our ability to do so allows us to navigate the stochastic world we inhabit. Uncertainty is, of course, not a static concept, but instead varies with the confidence in our predictions about that which we anticipate. The more confidently we predict, the less uncertain we are about the consequences of our actions and others' actions, and the more stable we feel in the present. This paper hypothesizes that increases in future uncertainty negatively affect our current outlook, specifically our self-reported life satisfaction. While uncertain times may be a harbinger of opportunity for some, for most, unpredictability is met with contemporaneous stress and worry. Capturing the immediate impacts of anxiety about the economic future motivates this work.  

Existing psychological evidence shows that stock market uncertainty correlates with individuals' decisions to engage in unhealthy behavior. In a similar fashion, we investigate how stock market uncertainty correlates to individuals' life satisfaction. To capture this effect, we build our hypothesis through the Anticipatory Utility framework, which suggests that people care about utility flow today and expected utility flows in the future. That is, the belief of a more optimistic future regarding employment status or wealth can bring contemporaneous enjoyment and correlate with higher utility. Conversely, a pessimistic future outlook can cause pain and disutility in the present. Specifically, we hypothesize that short-term volatility expectations relate to individuals' life satisfaction within the anticipatory framework in two major ways. First, increases in market uncertainty, which is directly related to stock market performance, negatively changes reported life satisfaction for stockholders through the income effect. Second, increases in market uncertainty may be negatively correlated with non-stockholders' reported life satisfaction through the fear of worsening economic conditions and other potential stressors.

Using observational survey data from the Behavioral Risk Factor Surveillance System (BRFSS), Current Population Survey (CPS) data, and Chicago Board Options Exchange (CBOE) Volatility Index (VIX) data from 2013 to 2017, this paper finds strong support for our hypothesis. Stock market uncertainty is measured using the S&P 500 options-implied volatility index (VIX), a 30 day-forward looking market index, which we use as our anticipatory indicator. Self-reported life satisfaction comes from BRFSS survey data, and stock ownership propensity is derived from the CPS data. Following prior research on this topic, we limit the income effect that would result from a change in macroeconomic conditions by controlling for unemployment, per capita personal income, and current market performance. Doing so allows us to capture the effects of market stress and uncertainty more effectively. This study reveals that the VIX negatively influences reported life satisfaction after adjusting for demographics, health conditions, and different fixed effects for time and states. Specifically, our results indicate that, at the mean, an additional percentage increase in the VIX decreases the probability of feeling "Very Satisfied" by 5.67\% and increases the likelihood of feeling "Dissatisfied" by 1.14\%. We also capture the presence of some income effects from stockholding activities. That is, the negative life satisfaction effect increases as the propensity to hold stocks increases, indicating that the stock market's impact is more prevalent for those with skin in the game, as expected.

During the recent stock market crashes, Americans reported large declines in self-reported life satisfaction (Deaton, 2011), exhibited increased symptoms of depression and poor mental well-being (McInerney, Mellor, & Nicholas, 2012), and experienced a spike in hospitalizations for psychological disorders (Engelberg & Parsons, 2013). Similar papers have used market price indicators, such as the Dow Jones Industrial Average (DJIA) as the independent variable of interest (Cotti, Dunn, and Tefft, 2013) to explore the market's impact on health measures. However, unlike previous research, we approach this question from a slightly different angle. Instead of assessing the correlation between life satisfaction and directional price changes in market indices, we measure the relationship between life satisfaction and changes in anticipated market uncertainty – options-implied market volatility. For example, the VIX index aggregates the S&P 500 call and put options in a way such that the index represents the implied volatility of the clearing prices of all S&P 500 options. Implied volatility in this sense is the amount of volatility required to set the options' expected value equal to zero, given the contracted prices. Therefore, the VIX can be thought of as an aggregate market sentiment regarding the anticipated price volatility of the S&P 500, expressed through the supply and demand dynamics of the options market. We believe this measure of future uncertainty is an improvement over past research for several reasons. 

First, periods of market turmoil are characteristically marked by large price movements in both directions, a well-documented phenomenon termed volatility clustering (Mandelbrot (1963), Granger and Ding (1993), and Ding and Granger (1996)). Large market declines may be followed by a large transitory rebound, which is then followed by another large decline. In fact, these transitory price increases are themselves an indicator of uncertainty, not recovery. Therefore, we propose that these temporary price increases amidst a broader crisis do not provide psychological relief in equal proportion to the distress caused by a price decrease of equal magnitude. Thus, our empirical model should capture the market's uncertainty level (expected volatility) rather than noisy directional price changes if our goal is to capture the effect of economic stress on life satisfaction. Second, it is documented that the VIX is asymmetrical in its response to price changes in the underlying S&P 500 index, rising more following a price decrease relative to a price increase (Low 2004). This evidence supports the Volatility Clustering concept presented above, whereby price increases do not alleviate uncertainty in equal proportion to their negative counterparts. This non-linearity of response between gain and loss domains is consistent with Prospect Theory (Kahneman and Tversky, 1979). The VIX may, therefore, provide an independent variable closely linked to the expected emotional responses related to changes in economic outcomes and outlooks. Third, the VIX's presence in the news media and widespread recognition as the market's fear gauge provides an additional property of interest for this study. Research demonstrates a significant yet complicated role for the news media in shaping economic perceptions. Through increasingly accessible and rapid media coverage, market signals reach a significant percentage of the general population and help shape sentiment regarding the economic outlook and confidence about one's current and future socio-economic life satisfaction (Procopio, Terrell, & Wu, 2010). Within this context, signals of increased uncertainty have a diminishing effect on one's life satisfaction, both economically and emotionally. Thus, the VIX both creates and is created by a general sense of uncertainty and fear about future macroeconomic conditions, which, we propose, drives psychological and physical malaise.

Our results have a range of significant implications. First, our findings support prior work postulating an effect of anticipatory feelings (e.g., Lowenstein, 1987) on individual desires and behaviors. Caplin and Leahy (2001) demonstrate, for example, that adding sentiment to the utility function can help explain time inconsistency in preferences. We show that the effect of forward-looking volatility fits into the Anticipatory Utility framework. Second, our findings add to the literature regarding feedback models (e.g., Shiller, 2002). Specifically, as Engelberg and Parsons (2016) have pointed out, most behavioral finance work concentrates on how investor behavior affects markets and often neglects the inverse effect. As a result, our finding introduces a new connection to how markets influence investor behavior.


This study is not the first to investigate the relationship between market uncertainty and commodities within the utility function. In fact, our study is motivated by recent behavioral finance papers (Engelberg and Parsons, 2016; Sias, 2017). However, this study differs from previous studies in two significant ways. First, this study is the first to use market-implied volatility as the leading independent variable of interest and more effectively capture uncertainty. Second, we link the effect of market volatility on life satisfaction through the anticipatory theoretical framework and show that our model deviates from the traditional Neo-classical model. Specifically, we show that the VIX acts as a natural anticipatory index, which can be built into the life satisfaction utility function (Stevenson, Wolfers 2008).

This paper proceeds as follows. Section 2 outlines the theoretical mechanisms of market uncertainty and individuals' well-being. Section 3 presents the data, descriptive statistics, and the empirical approach. Section 4 illustrates the methodologies and empirical results. This section also discusses the robustness check of the estimates produced in this paper. Section 5 concludes the paper.

\section{Theoretical Framework}

\subsection{Can measure of life satisfaction be interpreted as utility?}

In response to Stevenson, Wolfers (2008), "Economic growth and Subjective life satisfaction: reassessing the Easterlin Paradox," Becker and Rayo assert that this paper not only provides convincing evidence that self-reported happiness and measures of life satisfaction are positively correlated with income, both in rich and emerging countries, but also provides an interesting take on the relationship between utility and reported happiness or life satisfaction. They also conclude that, although there are grey areas in connecting self-reported life satisfaction to utility, it is quite acceptable to view happiness and life satisfaction as "noisy measures of utility." However, while they agree that there are connections between the two dimensions, "reported happiness and life satisfaction are no more measures of utility than are other dimensions of life satisfaction, such as health or consumption of material goods." Prior literature has shown that happiness levels are consistent with reported happiness, whether it is self-reported or reported by a trusted third party. Studies have also shown that we, as human beings, behave consistently with survey reports as well. For instance, we try to avoid bad situations that may result in a reduction in self-reported happiness.

At the same time, Becker and Rayo suggest an alternative interpretation of life-satisfaction and happiness data. These measures can be interpreted as a commodity in the utility function. In many perspectives, Becker and Rayo implicitly assert that just like owning a house or being healthy, happiness and life-satisfaction indeed describe the same commodity in the utility function. Besides, they take an extra step to show that there are ways to use the consumer utility maximization theory to test whether life satisfaction can be used as utility. In other words, it is possible to test whether happiness data is a commodity in the utility function. While discussion on whether happiness is identical with utility still needs more research, it is crucial to accept that life-satisfaction and happiness constitute useful measures in consumer utility maximization theory.

Given these reasons, while we do not necessarily take life-satisfaction or happiness as our utility measures, we are convinced that using self-reported life satisfaction could be a reasonable approach to measure utility, a commodity within a utility function, or at least a "noisy" measure of utility.

 
\subsection{Can the VIX be interpreted as anticipatory?}

In the Neo-classical model, people derive their utility from the consumption of physical goods, such as houses, food, health, leisure, etc. However, research has shown that people also derive their utility from their beliefs. For instance, being excited for a trip or worrying about health are anticipatory emotions derived from beliefs; believing food tastes better if made from quality ingredients is belief about consumables; avoiding negative information about oneself is belief about oneself, etc. There are some examples of belief-based preferences. Regarding consumption, people also have beliefs about their future consumption, which, importantly, can cause experienced utility in the present. This phenomenon is called anticipatory utility. The belief of diminished future wealth or a more pessimistic future can cause pain now. Similarly, the belief of a more optimistic future regarding employment status or wealth can bring contemporaneous enjoyment.

The VIX can be described as an anticipatory market-driven index for short-term price movements in the public equity market. It is, thus, useful for our research question since we are interested in how the anticipation of uncertainty affects contemporaneous utility. In simple terms, as the clearing prices for out-of-the-money options increase, so too does the value of the VIX.\footnote{Out-of-the-money option prices increase when the expectation of large price fluctuations increases, which in turn would result from new and important information. See VIX white paper for further details} Therefore, the VIX, which can be thought of as a derivative of the S&P 500, provides forward-looking information about anticipated uncertainty in equity markets. Further, since equity markets are a reflection of the economy, we propose that the VIX captures relevant information about uncertainty regarding the anticipated state of the world. The index is, therefore, a reasonable measure of future belief, and we hypothesize that it is a relevant proxy for anticipated uncertainty for both stockholder and non-stockholders that would influence contemporaneous life satisfaction under an Anticipatory Utility model.

The Anticipatory Utility model was presented in 1987 when Loewenstein first started addressing the notion of time discounting, which is the first implication of his anticipatory model. The anticipatory utility is built from beliefs about the future. The uncertainty surrounding these beliefs is a critical variable for our proposed hypothesis, which aims to link future uncertainty levels to an individual's contemporaneous life satisfaction. By definition, we believe that the VIX is an expression of short-term uncertainty and provides information about whether the economy is expected to be better or worse in the near future. In other words, just like the positive anticipation of a kiss in Loewenstein's model, anticipation about the future economy should have a similar effect. For instance, anticipation from a bright future economy should have both consumption and anticipation benefits. These benefits translate to a better total contemporaneous life satisfaction.

\subsection{Neo-classical vs. Anticipatory Theory}

Consider a $T-period$ model, where the agent decides in period $0$ when to consume. Let $x_t$ be consumption at time t. $0 \leq \delta \leq 1$ is the discount factor. Let utility from consumption in period $t$ be $u(x_t)$, then the total discount utility in Neo-classical model of preferences overtime is presented by: $U_0(x)= \sum_{t=0}^T \delta^t u(x_t)$. In anticipatory model, utility from anticipation in period $t$ is $\alpha \sum_{i=t+1}^T \delta^{i-t}u(x_i)$, where $\alpha > 0$. The total instantaneous utility at $t$ is then presented by: $u(x_t) + \alpha \sum_{i=t+1}^T \delta^{i-t}u(x_i)$. And the total anticipatory utility over consumption is: $U_t=\sum_{j=t}^T \delta^{j-t}[u(x_j)+\alpha \sum_{i=j+1}^T \delta^{i-j}u(x_i)]$.

The Neo-classical model only gives credit to physical goods, from which one's utility can derive. We clearly shy away from the Neo-classical model, and by adopting the anticipatory model, we assume the VIX to be the consumption for beliefs. Therefore, the effect of beliefs on instantaneous utility is presented by $u(x_t)$. In addition, to better present our hypothesis, we adapt an extended model by Caplin and Leahy (2001). 
 
Future events are indeed uncertain. They are also relevant to how one plans their actions in the present moment. Therefore, the acquisition of forward-looking information should affect the actions and thought processes of decision-makers. This effect can be both positive and negative under an Anticipatory Utility framework since beliefs about the future can cause changes to contemporaneous utility. As such, decision-makers may choose not to consume information under certain conditions, if such information is avoidable. Where information is unavoidable, the decision maker's utility may be influenced by events outside their consumption of physical goods and services. For example, a simplified Caplin and Leahy model assumes two periods. The decision-maker may receive information in period 1, and the outcome is realized in period 2. Let $x$ be random variable over support S and with distribution $f(x)$, and the consumption utility over $x$ by decision-maker is presented by $u(x)$. The decision-maker also has anticipatory utility in period 1, $b(x)$, over beliefs of $x$, and $a(b(x))$ is anticipatory utility. If there is no time discounting, the total expected utility in period 1 is: $U_1=a(b(x)) +\sum_{x \in S} b(x)u(x)$. Without information, the decision-maker has prior belief $f(x)$. With information, the decision-maker has posterior belief $g(x)$. The decision-maker only wants information if: $E_{f_x}[a(g(x)) + \sum_{x \in S} g(x)u(x)]>a(f(x)) + \sum_{x \in S} f(x)u(x)$. Given the equation, the decision-maker wants information if and only if: $E_{f_x}[a(g(x))] > a(f(x))$. Under the classical model, $a(b(x))$ does not exist, and thus with or without beliefs, the classical decision-maker should not be affected. 
  
To summarize the model, under a simplified two-period setup, a classical agent is entirely indifferent to receiving future information, whether the information is negative or positive. It implies that information on the VIX, which is forward-looking, should only influence the utility of agents who own stocks. In other words, the classical model would predict that non-stockholders should not be affected by this information. Controlling for stock ownership and current market movements of the S&P500, our paper shows otherwise.

\subsection{Mechanism of Market Uncertainty and Individuals' Life Satisfaction}

We propose three different explanations for the impact of market uncertainty, recorded using VIX, on individuals' life satisfaction. The first explanation is straight forward, with the effects of uncertainty explained through the anticipatory theoretical framework's mechanisms. In this case, market uncertainty imposes stress on life-satisfaction and behaviors, thus decreasing reported life-satisfaction. The effect of market volatility on different behavioral measures has been studied in many previous research efforts. For instance, Kalcheva, McLemore, Sias 2017 showed that VIX has a significant impact on different impulsive behaviors such as drinking and smoking. Schwandt 2014 shows evidence of market volatility under the S&P500 on physical health, mental health, and survival rates. Schwandt, Hannes 2018 updated on Schwandt 2014 to estimate the same effect during boom and bust markets in the U.S. and found that the impact of market volatility is much more substantial during bust periods. The stress that people experience during periods of poor market performance can be explained through uncertainty about their jobs, uncertainty about their future incomes, and uncertainty about their wealth, thus imposing strong adverse effects on different aspects of life such as health outcomes and behavioral outcomes. 
There is a possibility that market uncertainty, measured by VIX, will only affect stockholders in the current period due to trading activity, and therefore there is no anticipatory effect. In other words, through this explanation, the effect of the VIX on life satisfaction is only explained through traders whose life satisfaction is motivated by income derived from trading activity. In that case, any observed effects of the VIX on life satisfaction are likely resulting from changes in income and wealth levels of market participants, not through anticipatory utility changes. However, suppose there exist significant effects of the VIX on non-stockholders' life satisfaction, even after controlling for the current market movements (S&P500). In that case, we are convinced that our first explanation, anticipatory utility, is plausible. In other words, the VIX does indeed capture a general sense of economic uncertainty and turmoil, thus correlating to stress on individuals' life satisfaction. 

Another possibility is that the respondents are reporting their life satisfaction biasedly. They may be inclined to over-report their life satisfaction in the expectation that their life satisfaction will improve. While they may wish to report a false sense of life satisfaction for a myriad of reasons, previous research has shown that reported life satisfaction and health in survey data are mostly consistent with individuals' real states of being. 

\section{Data and Empirical framework}

\subsection{Data}

Our measure of life satisfaction outcome comes from the Behavioral Risk Factor Surveillance System (BRFSS). The dataset is maintained by the Center for Disease Control and Prevention (CDC) to monitor the health and behavioral risk in the United States of America. Using the BRFSS data from 2013-2017, we construct each individual's life satisfaction measure using an indicator for general life-satisfaction in the BRFSS survey. This variable is recorded from the very first general health question under the Health Status section, which asks, "In general, how satisfied are you with your life?" with the answers ranging from "Very satisfied," " Satisfied," "Dissatisfied" to "Very dissatisfied." We recode this variable to an ordinal outcome, ranking from "Dissatisfied," "Satisfied," and "Very satisfied."  Since there are very few observations in the group "Very dissatisfied," we group "Very dissatisfied" and "Dissatisfied" to one group, which we call "Dissatisfied." While not ideal, this grouping procedure was necessary, given the large imbalance between the dataset responses. By doing so, we achieve a well-classified model.

Further, we omitted respondents who answer, "Do not know/Not Sure," "Refused," and missing data. Given that the BRFSS survey is not consistent with the questions and the variable codes, especially before 2013, it is challenging to match all the same variables historically. Additionally, each year's dataset asks different questions, which populate different variables. Depending on the year, the number of variables can range from 180 to 300. Therefore, we only match variables that are consistent over the entire sample period. To further prepare the data, we only keep variables analyzed in this study and eliminate all NAs from this subset. Moreover, we also get rid of observations from respondents that did not complete the survey. For demographic variables, given the nature of the BRFSS survey, all of them are categorical variables. Recoding for all variables including  $age$, $gender$, $income$, $education-status$, $marital-status$, $employment-status$, $race$ are performed consistently, where missing data, "don't know," and "refused" are eliminated. For detailed descriptions and recoding process, see Appendix.

As discussed, our primary variable of interest is the Chicago Board Options Exchange (CBOE) Volatility Index or VIX. VIX has been referred to as the 'investor fear gauge' (Whaley (2000)) since high VIX levels coincided with high market turmoil degrees. The VIX index was initially designed to measure the market's expectation of 30-day volatility implied by the at-the-money S&P 100 Index (OEX Index) options prices. The popularity of the VIX Index made it soon become the premier benchmark for U.S. stock market volatility. The current VIX methodology was developed in 2003 and estimates expected volatility by averaging the weighted prices of SPX puts and calls over a wide range of strike prices for options across all companies in the S&P 500 Index (SPXSM). Daily percent changes in the VIX are used as the primary variable of interest in this analysis and are presented in the main panel.

Financial data is extracted from Yahoo! Finance. Since stock market data only have observations on 5 out of 7 days of the week (except some holidays), matching the VIX daily series to BRFSS historical survey data (which covers almost every day of the year, including the weekends) resulted in the loss of information. The daily series for the VIX is the adjusted price for the implied volatility index. 

To better assess market volatility's anticipatory effect on reported life satisfaction, we attempt to control for stock ownership in our analysis. The BRFSS survey, unfortunately, does not provide information on whether the respondents hold stocks. However, we were able to source this information from the Current Population Survey (CPS). \footnote{The CPS surveys, conducted monthly by the U.S. Bureau of the Census and the U.S. Bureau of Labor Statistics, are representative of the entire U.S. population} We perform a logistic regression to achieve the propensity to own stocks based on age, gender, income, education-status, and race (Kreinin et al. 1959). We then use these demographics, together with the propensity to own stocks yielding from the logistic regression, to merge the CPS dataset with the BRFSS dataset. Stockholding data spans from 2013 to 2017 with a total of 704,345 observations.\footnote{All variables from the CPS are recoded from continuous to categorical data to match the BRFSS demographic variables} For detailed deconstructions of the CPS data and the logistic regression, see the Methodology section. 

\subsection{Empirical framework}

Our baseline regression specification to examine the relation between stock market volatility and individuals' life satisfaction is:

\begin{footnotesize}
\begin{equation*}
LifeSatisfaction_{i,s,t}=\beta_1MarketVolatility_t + \beta_2StockOwnership_{i,t} + \beta_3CurrentMarket_t+\beta_XX_{i,s,t}+\tau_t+\gamma_s+\epsilon_{i,s,t}
\end{equation*}
\end{footnotesize}

In this model, the dependent variable is a measure of life satisfaction. In this analysis, $LifeSatisfaction_{i,s,t}$ is measured in reported General life satisfaction from the general American population.\footnote{see Descriptive Statistics for detailed deconstruction for life satisfaction} These surveys span from 2013 to 2017 and are made by an individual $i$ in-state $s$ on day $t$. $MarketVolatility_t$ is the daily index value of the VIX, or the Implied Volatility Index, on day $t$. We use the natural log of the VIX daily series, divided by 100. This variable takes the expression of $\frac{log(VIX)}{100}$. The expression does not change the significance of the estimates and helps with the interpretation. $StockOwnership_{i,t}$ is the propensity of owning stocks. This data is taken from the Current Population survey. \footnote{See Methodology} $CurrentMarket_t$ is the S&P500 return series and represents the current market performance. The set of control variables, which may influence life satisfaction, is suggested by previous work.\footnote{see, e.g., Engelberg and Parsons, 2016; Cotti, Dunn, and Tefft, 2015; Davalos, Fang, and French, 2011; Fiuzat, Shaw, Thomas, Felker and O'Connor, 2010; Ruhm, 2005; Ruhm and Black, 2002} Specifically, $X_{i,s,t}$  is a matrix of individual-level demographic data including $age$, $gender$, $income$, $education-status$, $marital-status$, $employment-status$, $race$, and the propensity to own stocks. In addition, $\tau_t$ are indicator variables for calendar months and calendar years. $\gamma_s$ are state fixed effects. 

$\tau_t$ is a vector of month fixed effects and year fixed effects, which accounts for the effect of seasonality on life satisfaction and behaviors. We also control for state fixed effects, $\gamma_s$, which would account for permanent differences across states. These controls account for lifestyle patterns, state infrastructures, social norms, and other unobserved state-level idiosyncrasies that may vary over time and influence life satisfaction changes. They also help us control for other state-level time-varying factors such as changes in health care delivery services that closely follow tax revenues. The $\epsilon_{i,s,t}$ is an idiosyncratic random error term.

\subsection{Descriptive Statistics}

Our final data is presented in the most interpretable format. All variables and sub-categories are put into two different columns based on our primary variable of interest, life satisfaction. The first column is 3, which presents respondents who reported "Very Satisfied." The second column is 2, which presents respondents who reported "Satisfied" while the last column represents "Dissatisfied. While one may argue on the reliability of these subjective questions, previous works (Apouey and Clark 2015) suggest that they capture an overall assessment of life satisfaction and a combination of mental and physical health. Further, Benjamins et al. 2004, Miilunpalo et al. 1997, Jylha 2009 have shown that these measures can predict various health outcomes, such as mortality and healthcare utilization. Thus, it is fair to conclude that self-reported life satisfaction measures from health surveys in our data plausibly correlate with objective health. These self-reported measures from reliable sources such as the BRFSS have been used in many economics studies. However, some previous studies suggest there may exist some reporting errors (Baker, Stabile, and Deri 2004), which can affect our estimates. As described above, we try our best to minimize these errors by omitting incomplete respondents, only including variables in the main panel, and getting rid of vaguely reported observations.\footnote{Depending on the questions, these are respondents that answer \textit{Do not Know} or \textit{Refused}}



In addition, our self-reported measures are particularly useful for a study of the short-run effects of market sentiments. It seems unlikely that more severe or objective measures of poor life satisfaction conditions (e.g., mortality, chronic conditions, hospitalizations) will respond in the short run to a change in market volatility (and the associated income and time cost changes). Thus, our analysis of self-reported life satisfaction, which captures how a person evaluates their life satisfaction at a point in time, is potentially more responsive, and therefore more suitable, for our study objectives than more severe or objective measures.

To provide a good view of the dataset, we show the following chart, representing the number of respondents who answered the survey throughout 2013-2017. Although there is a disproportion in the number of participants who respond to the survey across different months, our time control variables should capture seasonality in our model analysis. Note that we do not fully show the CPS data's descriptive statistics since they do not necessarily play a significant role in our analysis. However, a brief explanation of the CPS data is introduced in Section 4.

\begin{figure}[H]
\caption{Number of People Interviewed in the Sample}
\includegraphics[width=.5\textwidth]{sample_number_interviewed.png} 
\centering
\label{}
\end{figure}


As discussed in the prior section, $MarketVolatility_t$ is the natural log of the VIX daily series. In performing this operation, we not only get an interpretable value, but we also get a stationary VIX times series. As shown in the graphs below, all $MarketVolatility_t$ series appear stationary. However, because market volatility tends to cluster into two distinct regimes, low volatility and high volatility, the S&P 500 returns exhibit non-constant variance over the full time-series. We do not observe the same variance characteristics for the VIX, which exhibits less heteroscedasticity, thus adhering more closely to our generalized linear estimation model's assumptions. This fact provides additional empirical justification for considering the VIX as our primary variable of interest. Nevertheless, we do indeed control for the S&P500 ($CurrentMarket_t$), which controls for the current market movements and news. The S&P 500 return series, which are used as a control in this paper, pass standard stationarity tests at the 5\% level.


\begin{figure}[H]
\centering
\begin{subfigure}{.5\textwidth}
  \centering
  \includegraphics[width=1\linewidth]{vix_daily.png}
  \caption{Time Series - Log of VIX Daily}
  \label{fig:VIXts}
\end{subfigure}%
\begin{subfigure}{.5\textwidth}
  \centering
  \includegraphics[width=1\linewidth]{log_vix_daily.png}
  \caption{Time Series - Log of VIX Daily}
  \label{fig:logVIXts}
\end{subfigure}
\caption{VIX Times Series Data}
\label{fig:vixTS}
\end{figure}


\begin{footnotesize}
\captionsetup[table]{labelformat=empty,skip=1pt}
\begin{longtable}{lccc}
\caption*{
\large Descriptive Statistics \label{tab:descriptive1} \\ 
\small By life satisfaction (1 = Dissatisfied, 2 = Satisfied, 3 = Very Satisfied) \\ 
} \\ 
\toprule
\textbf{Characteristic} & \textbf{(1)}, N = 528 & \textbf{(2)}, N = 3,784 & \textbf{(3)}, N = 3,204 \\ 
\midrule
\textbf{Age}, n / N (\%) &  &  &  \\ 
\quad 18to24 & 3 / 528 (0.6\%) & 21 / 3,784 (0.6\%) & 15 / 3,204 (0.5\%) \\ 
\quad 25to34 & 13 / 528 (2.5\%) & 106 / 3,784 (2.8\%) & 90 / 3,204 (2.8\%) \\ 
\quad 35to44 & 38 / 528 (7.2\%) & 220 / 3,784 (5.8\%) & 174 / 3,204 (5.4\%) \\ 
\quad 45to54 & 121 / 528 (23\%) & 582 / 3,784 (15\%) & 397 / 3,204 (12\%) \\ 
\quad 55to64 & 199 / 528 (38\%) & 1,058 / 3,784 (28\%) & 781 / 3,204 (24\%) \\ 
\quad 65older & 154 / 528 (29\%) & 1,797 / 3,784 (47\%) & 1,747 / 3,204 (55\%) \\ 
\textbf{Gender}, n / N (\%) &  &  &  \\ 
\quad female & 328 / 528 (62\%) & 2,146 / 3,784 (57\%) & 1,789 / 3,204 (56\%) \\ 
\quad male & 200 / 528 (38\%) & 1,638 / 3,784 (43\%) & 1,415 / 3,204 (44\%) \\ 
\textbf{Income}, n / N (\%) &  &  &  \\ 
\quad 50more & 82 / 528 (16\%) & 1,144 / 3,784 (30\%) & 1,481 / 3,204 (46\%) \\ 
\quad 15to25K & 155 / 528 (29\%) & 899 / 3,784 (24\%) & 534 / 3,204 (17\%) \\ 
\quad 25to35K & 74 / 528 (14\%) & 511 / 3,784 (14\%) & 404 / 3,204 (13\%) \\ 
\quad 35to50K & 40 / 528 (7.6\%) & 585 / 3,784 (15\%) & 477 / 3,204 (15\%) \\ 
\quad le15K & 177 / 528 (34\%) & 645 / 3,784 (17\%) & 308 / 3,204 (9.6\%) \\ 
\textbf{Education}, n / N (\%) &  &  &  \\ 
\quad COLgrad & 115 / 528 (22\%) & 984 / 3,784 (26\%) & 1,122 / 3,204 (35\%) \\ 
\quad attendCOL & 176 / 528 (33\%) & 1,187 / 3,784 (31\%) & 908 / 3,204 (28\%) \\ 
\quad HSgrad & 160 / 528 (30\%) & 1,212 / 3,784 (32\%) & 934 / 3,204 (29\%) \\ 
\quad K & 77 / 528 (15\%) & 401 / 3,784 (11\%) & 240 / 3,204 (7.5\%) \\ 
\textbf{Marital Status}, n / N (\%) &  &  &  \\ 
\quad married & 144 / 528 (27\%) & 1,721 / 3,784 (45\%) & 1,989 / 3,204 (62\%) \\ 
\quad divorced & 151 / 528 (29\%) & 705 / 3,784 (19\%) & 359 / 3,204 (11\%) \\ 
\quad membermarriedcoup & 16 / 528 (3.0\%) & 53 / 3,784 (1.4\%) & 40 / 3,204 (1.2\%) \\ 
\quad nevermarried & 100 / 528 (19\%) & 492 / 3,784 (13\%) & 242 / 3,204 (7.6\%) \\ 
\quad separated & 38 / 528 (7.2\%) & 90 / 3,784 (2.4\%) & 43 / 3,204 (1.3\%) \\ 
\quad widowed & 79 / 528 (15\%) & 723 / 3,784 (19\%) & 531 / 3,204 (17\%) \\ 
\textbf{Employment Status}, n / N (\%) &  &  &  \\ 
\quad wagesemployed & 95 / 528 (18\%) & 1,097 / 3,784 (29\%) & 982 / 3,204 (31\%) \\ 
\quad homemaker & 14 / 528 (2.7\%) & 132 / 3,784 (3.5\%) & 139 / 3,204 (4.3\%) \\ 
\quad noworkless1 & 14 / 528 (2.7\%) & 84 / 3,784 (2.2\%) & 35 / 3,204 (1.1\%) \\ 
\quad noworkmore1 & 33 / 528 (6.2\%) & 97 / 3,784 (2.6\%) & 33 / 3,204 (1.0\%) \\ 
\quad retired & 126 / 528 (24\%) & 1,588 / 3,784 (42\%) & 1,561 / 3,204 (49\%) \\ 
\quad selfemployed & 12 / 528 (2.3\%) & 191 / 3,784 (5.0\%) & 217 / 3,204 (6.8\%) \\ 
\quad student & 2 / 528 (0.4\%) & 11 / 3,784 (0.3\%) & 20 / 3,204 (0.6\%) \\ 
\quad unable & 232 / 528 (44\%) & 584 / 3,784 (15\%) & 217 / 3,204 (6.8\%) \\ 
\textbf{Race}, n / N (\%) &  &  &  \\ 
\quad white & 429 / 528 (81\%) & 3,186 / 3,784 (84\%) & 2,749 / 3,204 (86\%) \\ 
\quad asian & 1 / 528 (0.2\%) & 20 / 3,784 (0.5\%) & 23 / 3,204 (0.7\%) \\ 
\quad black & 68 / 528 (13\%) & 441 / 3,784 (12\%) & 324 / 3,204 (10\%) \\ 
\quad native & 23 / 528 (4.4\%) & 71 / 3,784 (1.9\%) & 68 / 3,204 (2.1\%) \\ 
\quad other & 7 / 528 (1.3\%) & 60 / 3,784 (1.6\%) & 37 / 3,204 (1.2\%) \\ 
\quad pacific & 0 / 528 (0\%) & 6 / 3,784 (0.2\%) & 3 / 3,204 (\textless 0.1\%) \\ 
\bottomrule
\end{longtable}
\end{footnotesize}


Table~\ref{tab:descriptive1} presents descriptive statistics based on general life satisfaction. As presented, there is a larger number of respondents who report "Satisfied" (N = 3,707) and "Very satisfied" (N = 3,149) compared to those who report "Dissatisfied" (N = 514). To put this in perspective, we show charts based on each group's demographics.

\begin{small}
\begin{longtable}{lccccccc}
\caption{\large Descriptive Statistics \label{tab:descriptive2}} \\ 
\toprule
Statistic & \multicolumn{1}{c}{N} & \multicolumn{1}{c}{Mean} & \multicolumn{1}{c}{St. Dev.} & \multicolumn{1}{c}{Min} & \multicolumn{1}{c}{Pctl(25)} & \multicolumn{1}{c}{Pctl(75)} & \multicolumn{1}{c}{Max} \\ 
\hline 
\midrule
Health Insured & 7,516 & 0.949 & 0.219 & 0 & 1 & 1 & 1 \\ 
Heart Disease & 7,516 & 0.142 & 0.349 & 0 & 0 & 0 & 1 \\ 
Arthritis & 7,516 & 0.493 & 0.500 & 0 & 0 & 1 & 1 \\ 
Stroke & 7,516 & 0.085 & 0.280 & 0 & 0 & 0 & 1 \\ 
Asthma & 7,516 & 0.155 & 0.362 & 0 & 0 & 0 & 1 \\ 
Bronchitis & 7,516 & 0.125 & 0.331 & 0 & 0 & 0 & 1 \\ 
Depression & 7,516 & 0.266 & 0.442 & 0 & 0 & 1 & 1 \\ 
Cancer & 7,516 & 0.143 & 0.350 & 0 & 0 & 0 & 1 \\ 
Diabetes & 7,516 & 0.915 & 0.279 & 0 & 1 & 1 & 1 \\ 
\hline 
\bottomrule
\end{longtable}
\end{small}

Table~\ref{tab:descriptive2} shows the descriptive statistics for all chronic condition covariates in the dataset. On average, 95\% of respondents in the dataset have health insurance. About 14\% of the respondents have heart disease, 50\% have arthritis, more than 8\% have had a stroke, more than 15\% have asthma, 12\% have bronchitis, 27\% have suffered depression, 14\% have cancer, and about 91\% are diabetic.


\section{Methodology and Empirical Results}

\subsection{Methodology}

One challenging element of this analysis is the processing and merging of stockholding data extracted from the Current Population Survey. Unlike the BRFSS, where all of the variables are categorical, variables from the CPS are continuous. Thus, we recoded all demographic variables from the CPS to match those of the BRFSS dataset, allowing us to maintain variable consistency during the data integration process. The stock ownership variable within the CPS dataset is crucial to our analysis. Specifically, information regarding an individual's stockholdings helps us to define the effect of market uncertainty on the stockholding population and non-stock-holding population in our sample.

In order to properly extract and integrate stock ownership information with our BRFSS data, we developed a stock-ownership propensity score from the Current Population Survey (CPS). We do so by employing a logistic regression to derive the relationship between demographic variables and stock ownership. As suggested by previous literature (Kreinin et al. 1959), we use some of the most significant demographic predictors of stock ownership, including $age$, $gender$, $income$, $education-status$, and $race$, in our regression to achieve the propensity score at individual levels, which we were then able to merge onto the BRFSS dataset based on the demographics mentioned above.


\begin{figure}[H]
\centering
\begin{subfigure}{.5\textwidth}
  \centering
  \includegraphics[width=1\linewidth]{age.png}
  \caption{CPS Age Data}
  \label{fig:age}
\end{subfigure}%
\begin{subfigure}{.5\textwidth}
  \centering
  \includegraphics[width=1\linewidth]{race.png}
  \caption{CPS Race Data}
  \label{fig:race}
\end{subfigure}
\caption{CPS Demographics - Age and Race}
\label{fig:age_race}
\end{figure}


\begin{figure}[H]
\centering
\begin{subfigure}{.5\textwidth}
  \centering
  \includegraphics[width=1\linewidth]{education.png}
  \caption{CPS Education Data}
  \label{fig:education}
\end{subfigure}%
\begin{subfigure}{.5\textwidth}
  \centering
  \includegraphics[width=1\linewidth]{income.png}
  \caption{CPS Income Data}
  \label{fig:income}
\end{subfigure}
\caption{CPS Demographics - Education and Income}
\label{fig:education_income}
\end{figure}



Our propensity results indicate that males are more likely to own stocks compared to females, as are respondents who have a college degree compared to those who do not. Further, income and age play a predictable role in an individual's propensity to participate in the stock market. Respondents who make at least 50 thousand dollars a year are the most likely to own stock, as are 65 years of age and older. We also find that the Caucasian population is the most likely to own stocks, followed by mixed races and Asian. Following prior literature, we believe that propensity score methodology and subsequent results offer a decent representation of actual stock ownership characteristics.

\begin{figure}[H]
\caption{Propensity Score for Stock Ownership}
\includegraphics[width=.5\textwidth]{stock_ownship.png} 
\centering
\label{fig:propensity}
\end{figure}

\begin{small}
\def\sym#1{\ifmmode^{#1}\else\(^{#1}\)\fi}
\begin{longtable}[c]{lc} 
\caption{Regression Results for Stock Ownership \label{tab:cpsrsl}} \\
\hline\hline
& \multicolumn{1}{c}{ \textit{Dependent Variable}} \\
\cline{2-2}\\[-4ex]
& \multicolumn{1}{c}{Stock Ownership}\\
\\[-4ex]
                    &\multicolumn{1}{c}{(1)}\\
                    &\multicolumn{1}{c}{Logit}\\
\hline
 gendermale & 0.242$^{***}$ \\ 
  & (0.011) \\ 
 
 calculated\_educationattendCOL & $-$0.954$^{***}$ \\ 
  & (0.014) \\ 
 
 calculated\_educationHSgrad & $-$1.411$^{***}$ \\ 
  & (0.014) \\ 
 
 calculated\_educationK & $-$2.371$^{***}$ \\ 
  & (0.028) \\ 
 
 calculated\_income15to25K & $-$1.244$^{***}$ \\ 
  & (0.027) \\ 
 
 calculated\_income25to35K & $-$0.972$^{***}$ \\ 
  & (0.024) \\ 
 
 calculated\_income35to50K & $-$0.724$^{***}$ \\ 
  & (0.019) \\ 
 
 calculated\_incomele15K & $-$1.568$^{***}$ \\ 
  & (0.030) \\ 
 
 calculated\_age25to34 & 0.661$^{***}$ \\ 
  & (0.029) \\ 
 
 calculated\_age35to44 & 1.152$^{***}$ \\ 
  & (0.028) \\ 
 
 calculated\_age45to54 & 1.471$^{***}$ \\ 
  & (0.028) \\ 
 
 calculated\_age55to64 & 1.752$^{***}$ \\ 
  & (0.028) \\ 
 
 calculated\_age65older & 2.183$^{***}$ \\ 
  & (0.028) \\ 
 
 calculated\_raceasian & $-$0.287$^{***}$ \\ 
  & (0.022) \\ 
 
 calculated\_raceblack & $-$0.933$^{***}$ \\ 
  & (0.021) \\ 
 
 calculated\_racenative & $-$0.405$^{***}$ \\ 
  & (0.057) \\ 
 
 calculated\_raceother & $-$0.153$^{***}$ \\ 
  & (0.043) \\ 
 
 calculated\_racepacific & $-$0.717$^{***}$ \\ 
  & (0.088) \\ 
 
 Constant & $-$0.385$^{***}$ \\ 
  & (0.028) \\ 
 
\hline \\[-1.8ex] 
Observations & 177,296 \\ 
Log Likelihood & $-$98,710.070 \\ 
Akaike Inf. Crit. & 197,458.100 \\ 
\hline 
\hline \\[-1.8ex] 
\multicolumn{2}{l}{Note: Panel consists of the 2013-2017 survey sample waves of CPS.}\\
\multicolumn{2}{l}{Model estimates with a Logistic Regression. Model controls for}\\
\multicolumn{2}{l}{demographics, including age, gender, race, income, education status,}\\
\multicolumn{2}{l}{marital status,employment status.  See Appendix for detailed variable}\\
\multicolumn{2}{l}{definition, data source, and construction.}\\
\multicolumn{2}{l}{ $^{*}$p$<$0.1; $^{**}$p$<$0.05; $^{***}$p$<$0.01} \\ 
\hline\hline
\end{longtable}
\end{small}

Full Logistic regression results of demographics on stock ownership are presented in Table~\ref{tab:cpsrsl}. As shown in Figure~\ref{fig:propensity}, our propensity score distribution for stock ownership spreads perfectly from almost 0\% chance of owning stock to nearly 100\% chance of owning stocks. The results from CPS data indicate two significant signals. First, the logistic regression on stock ownership is appropriate. Each observation in the CPS dataset has its value for stock ownership. Second, the perfect propensity score will provide an advantage when merging in the BRFSS for analysis. It will ensure that each BRFSS observation will yield a unique propensity to own stocks, ranging from 0\% to 100\%. 

The analysis proceeds with the ordinal logistics model. Our dependent variable is Life Satisfaction, which takes into three categories of "Very Satisfied," "Satisfied," and "Dissatisfied." The main independent variable of interest is the natural log of the VIX, divided by 100. We also control for the current market performance (S&P500 return series), stock ownership (propensity to own stock), demographics, and a set of Fixed-Effects. As described above, controlling for a set of demographics variables is extremely useful in generating precise estimates, given different demographics in our survey. In addition, by controlling for State Fixed-Effects, Monthly Fixed-Effects, and Yearly Fixed-Effects, we aim to achieve the most precise estimates possible by accounting for the impact of seasonality that may exist in some behaviors such as physical activity (Ruhm 2005), permanent differences across states that may affect health and health behaviors, such as lifestyles patterns, state infrastructures on health care, and confounding factors that may trend linearly. All the regressions are weighted using the BRFSS sampling weights.

\subsection{Empirical Results}

As previously mentioned, we have a robust set of controls, including gender, age (6 different categories), income (5 different categories), education status (4 categories), marital status (6 different categories), race (6 different categories), chronic health conditions, and employment status (8 different categories). Moreover, a set of fixed effects for months, years, and states ensures that we can capture the effect while minimizing modeling errors and biases. Although this study does not necessarily focus on the effect of demographics on life satisfaction nor the effect of chronic conditions on life satisfaction, our results show that these effects across the board are expected.

The central panel, presented in Table~\ref{tab:mdl1rslt}, shows the effects of the VIX on life satisfaction outcomes. Across four different models, models (2) and (4) control for additional chronic health conditions, while models (1) and (3) do not. All models control for $psvalue.cps$, which is the propensity score value for stock ownership. Our results indicate that the effects of the VIX on life satisfaction are reasonably consistent overall. The magnitudes of the results do not fluctuate significantly across different specifications. Model (4) is our prime model, where it controls for chronic conditions, propensity score of stock ownership, the interaction of propensity score of stock ownership and the natural log of the VIX, and sample weights. The interaction term shows that as the market is under stress, increases in the likelihood of the respondents owning stock result in decreases in the probability of respondents moving toward the next category (feeling satisfied). We are likely capturing the income effect in the regression, as expected. In other words, the more likely it is that a respondent is a stockholder, the more likely they are to have poor life satisfaction during periods of market turmoil.

\begin{footnotesize}
\def\sym#1{\ifmmode^{#1}\else\(^{#1}\)\fi}
\begin{longtable}{l*{4}{c}}
\caption{Regression Results for Life-Satisfaction and Daily VIX \label{tab:mdl1rslt}} \\
\hline\hline
& \multicolumn{4}{c}{ \textit{Dependent Variable}} \\
\cline{2-5}\\[-4ex]
& \multicolumn{4}{c}{Life Satisfaction}\\
\\[-4ex]
                    &\multicolumn{1}{c}{(1)}&\multicolumn{1}{c}{(2)}&\multicolumn{1}{c}{(3)}&\multicolumn{1}{c}{(4)}\\
                    &\multicolumn{1}{c}{Ordinal Logit}&\multicolumn{1}{c}{Ordinal Logit}&\multicolumn{1}{c}{Ordinal Logit}&\multicolumn{1}{c}{Ordinal Logit}\\
\hline
logVIX.Adjusted     &      -26.380\sym{***}&      -24.346\sym{***}&      -14.041\sym{***}&      -13.848\sym{***}\\
                    &     (1.019)         &     (1.036)         &     (1.627)         &     (1.646)         \\
 
psvalue.cps         &       0.308\sym{***}&       0.356\sym{***}&       1.092\sym{***}&       1.025\sym{***}\\
                    &     (0.035)         &     (0.036)         &     (0.088)         &     (0.089)         \\
logVIX.Adjusted $\times$ psvalue.cps&                     &                     &     -30.208\sym{***}&     -25.798\sym{***}\\
                    &                     &                     &     (3.104)         &     (3.143)         \\
                    
SPY.rets            &       0.000\sym{*}  &      -0.000\sym{**} &       0.000\sym{*}  &      -0.000\sym{**} \\
                    &     (0.000)         &     (0.000)         &     (0.000)         &     (0.000)         \\
\hline \\[-1.8ex] 
Month FE  & Yes & Yes & Yes & Yes \\ 
State FE  & Yes & Yes & Yes & Yes \\ 
Year FE  & Yes & Yes & Yes & Yes \\
Chronic Health & No & Yes & No & Yes\\ 
Survey Weights & No & No & Yes & Yes\\ 
Observations  & 7,516 & 7,516 & 7,516 & 7,516 \\ 
\hline \\[-1.8ex] 
\multicolumn{5}{l}{Note: Full Table in the Appendix. Panel consists of the 2013-2017 survey sample waves of BRFSS.}\\
\multicolumn{5}{l}{BRFSS sample weights applied. Market volatility is defined as $\frac{log(VIX)}{100}$. All models estimated}\\
\multicolumn{5}{l}{with Logistic Regression. All models control for demographics, including age, gender, race,}\\
\multicolumn{5}{l}{education status, income, marital status, employment status. See Appendix for detailed variable}\\
\multicolumn{5}{l}{definition, data  source, and construction.}\\
\multicolumn{5}{l}{\textit{$^{*}$p$<$0.1; $^{**}$p$<$0.05; $^{***}$p$<$0.01}} \\ 
\hline\hline
\end{longtable}
\end{footnotesize}

To better understand the effects, we want to investigate the marginal effects of the VIX on life satisfaction. We achieve marginal effects at the mean and average marginal effects at Table \@ref(tab:marginal1). Our results indicate that at the mean, an additional percentage increase in the VIX decreases the probability of feeling "Very Satisfied" by 5.67\%, holding all else constant, which is significant at the 1\% level. Moreover, at the mean, an additional percentage increase in the VIX increases the probability of feeling "Dissatisfied" by 1.14\%, holding all else constant, which is significant at the 1\% level. On average, an additional percentage increase in the VIX decreases the probability of feeling "Very Satisfied" by 4.99\%, holding all else constant, which is significant at the 1\% level. Similarly, on average, an additional percentage increase in the VIX increases the probability of feeling "Dissatisfied" by 1.37\%, holding all else constant, which is significant at the 1\% level. Given the VIX's standard errors, we are confident to reject the null hypothesis, concluding a negative association between the VIX and better-reported life satisfaction. Further, the 95\% confidence interval lies within the negative zone of the effect, which also confirms that this effect is negatively correlated.


\begin{footnotesize}
\def\sym#1{\ifmmode^{#1}\else\(^{#1}\)\fi}
\begin{longtable}{l*{4}{c}}
\caption{Marginal Effect Table \label{tab:marginal1}} \\
\hline\hline
& \multicolumn{4}{c}{ \textit{Dependent Variable}} \\
\cline{2-5}\\[-4ex]
& \multicolumn{4}{c}{Life Satisfaction}\\
\\[-4ex]
                    &\multicolumn{1}{c}{(1)}&\multicolumn{1}{c}{(2)}&\multicolumn{1}{c}{(3)}&\multicolumn{1}{c}{(4)}\\
                    &\multicolumn{1}{c}{MEM}&\multicolumn{1}{c}{AME}&\multicolumn{1}{c}{MEM}&\multicolumn{1}{c}{AME}\\
                    &\multicolumn{1}{c}{Very Satisfied}&\multicolumn{1}{c}{Very Satisfied}&\multicolumn{1}{c}{Dissatisfied}&\multicolumn{1}{c}{Dissatisfied}\\
\hline
logVIX.Adjusted     &       -5.67\sym{***}&       -4.99\sym{***}&       1.14\sym{***}&      1.37\sym{***}\\
                    &     (0.016)         &     (0.016)         &     (0.016)         &     (0.016)         \\
 
psvalue.cps         &       0.082\sym{***}&       0.070\sym{***}&       -0.016\sym{***}&       -0.022\sym{***}\\
                    &     (0.089)         &     (0.089)         &     (0.089)         &     (0.089)         \\

SPY.rets            &      -0.000\sym{**} &      -0.000\sym{**} &      -0.000\sym{**} &      -0.000\sym{**} \\
                    &     (0.000)         &     (0.000)         &     (0.000)         &     (0.000)         \\
\hline \\[-1.8ex] 
Month FE  & Yes & Yes & Yes & Yes \\ 
State FE  & Yes & Yes & Yes & Yes \\ 
Year FE  & Yes & Yes & Yes & Yes \\
Survey Weights & No & No & Yes & Yes\\ 
Observations  & 7,516 & 7,516 & 7,516 & 7,516 \\ 
\hline \\[-1.8ex] 
\multicolumn{5}{l}{Note: Table includes Average Marginal Effects (AME) and Marginal Effects at}\\
\multicolumn{5}{l}{Mean (MEM). Market volatility is defined as $\frac{log(VIX)}{100}$. All models}\\
\multicolumn{5}{l}{control for demographics, including age, gender, race, education status, income,}\\
\multicolumn{5}{l}{marital status, employment status. See Appendix for detailed variable definition,}\\
\multicolumn{5}{l}{data  source, and construction.}\\
\multicolumn{5}{l}{ \textit{$^{*}$p$<$0.1; $^{**}$p$<$0.05; $^{***}$p$<$0.01}} \\ 
\hline\hline
\end{longtable}
\end{footnotesize}



\subsection{Additional Robustness Check}

Based on previous literature, we initially did not control for chronic conditions. However, we demonstrate our results' robustness by controlling for a series of additional controls, including the propensity of stock ownership, and especially chronic conditions. The results show that our models are robust and consistent across different models of the VIX series. During the analysis, we suspect that fluctuations in the implied volatility series, VIX, could come from the stock market's actual performance overall. Thus, we controlled for the return series of stock performance using the S&P 500. All of our models indeed control for the S&P500 return series, and we find strong consistency across each model's coefficients. 

It is also important to demonstrate consistency in results when using different windows to calculate our VIX series. It is possible that the daily series of the VIX do not necessarily affect respondents since they may have time to check the news or their stock positions before submitting their responses on interview day. By generating a lag series of the VIX (i.e., 1-day lag, 2-day lag, 3-day lag, 4-day lag, and 1-week lag), we capture the short-term effect of market volatility on life satisfaction in addition to our contemporaneous model. The results are consistent across different windows of the VIX. Although we do not interpret these results in Table~\ref{tab:mdl2rslt}, we can see that the direction of the estimates is still consistent for both the VIX and the interaction term. It indicates that our models are robust, and we confidently reject our null hypothesis.

\begin{footnotesize}
\def\sym#1{\ifmmode^{#1}\else\(^{#1}\)\fi}
\begin{longtable}{l*{5}{c}}

\caption{Regression Results for Life-Satisfaction and Daily VIX \label{tab:mdl2rslt}} 

& \multicolumn{5}{c}{\textit{Dependent Variable: Life Satisfaction}} \\ 
\cline{2-6} 
                    &\multicolumn{1}{c}{(1)}&\multicolumn{1}{c}{(2)}&\multicolumn{1}{c}{(3)}&\multicolumn{1}{c}{(4)}&\multicolumn{1}{c}{(5)}\\
                    &\multicolumn{1}{c}{Ordinal Logit}&\multicolumn{1}{c}{Ordinal Logit}&\multicolumn{1}{c}{Ordinal Logit}&\multicolumn{1}{c}{Ordinal Logit}&\multicolumn{1}{c}{Ordinal Logit}\\
\hline
loglag1vixadjusted100&      -0.013$^{**}$ &                     &                     &                     &                     \\
                    &     (0.005)         &                     &                     &                     &                     \\
 
psvalue.cps         &       1.021$^{***}$&       1.003$^{***}$&       1.305$^{***}$&       1.189$^{***}$&       1.122$^{***}$\\
                    &     (0.093)         &     (0.093)         &     (0.093)         &     (0.093)         &     (0.093)         \\
 
loglag1vixadjusted100 $\times$ psvalue.cps&      -0.013         &                     &                     &                     &                     \\
                    &     (0.008)         &                     &                     &                     &                     \\
 
lag1.SPY.rets       &      -0.000         &                     &                     &                     &                     \\
                    &     (0.000)         &                     &                     &                     &                     \\

 
loglag2vixadjusted100&                     &      -0.045$^{***}$&                     &                     &                     \\
                    &                     &     (0.005)         &                     &                     &                     \\
 
loglag2vixadjusted100 $\times$ psvalue.cps&                     &      -0.000         &                     &                     &                     \\
                    &                     &     (0.009)         &                     &                     &                     \\
 
lag2.SPY.rets       &                     &      -0.000$^{***}$&                     &                     &                     \\
                    &                     &     (0.000)         &                     &                     &                     \\
 
loglag3vixadjusted100&                     &                     &      -0.014$^{**}$ &                     &                     \\
                    &                     &                     &     (0.005)         &                     &                     \\
 
loglag3vixadjusted100 $\times$ psvalue.cps&                     &                     &      -0.105$^{***}$&                     &                     \\
                    &                     &                     &     (0.008)         &                     &                     \\
 
lag3.SPY.rets       &                     &                     &      -0.000$^{***}$&                     &                     \\
                    &                     &                     &     (0.000)         &                     &                     \\
 
loglag4vixadjusted100&                     &                     &                     &      -0.028$^{***}$&                     \\
                    &                     &                     &                     &     (0.005)         &                     \\
 
loglag4vixadjusted100 $\times$ psvalue.cps&                     &                     &                     &      -0.071$^{***}$&                     \\
                    &                     &                     &                     &     (0.008)         &                     \\
 
lag4.SPY.rets       &                     &                     &                     &       0.000$^{***}$&                     \\
                    &                     &                     &                     &     (0.000)         &                     \\
 
logweeklagvixadjusted100&                     &                     &                     &                     &      -0.018$^{***}$\\
                    &                     &                     &                     &                     &     (0.005)         \\
 
logweeklagvixadjusted100 $\times$ psvalue.cps&                     &                     &                     &                     &      -0.046$^{***}$\\
                    &                     &                     &                     &                     &     (0.008)         \\
 
weeklag.SPY.rets    &                     &                     &                     &                     &       0.000$^{***}$\\
                    &                     &                     &                     &                     &     (0.000)         \\

\hline \\[-1.8ex] 
Month FE  & Yes & Yes & Yes & Yes & Yes \\ 
State FE  & Yes & Yes & Yes & Yes & Yes \\ 
Year FE  & Yes & Yes & Yes & Yes & Yes \\ 
Survey Weights & No & No & Yes & Yes & Yes\\ 
Observations  & 7,516 & 7,516 & 7,516 & 7,516 & 7,516 \\ 
\hline \\[-1.8ex] 
\multicolumn{6}{l}{Note: Full Table in the Appendix Panel consists of the 2013-2017 survey sample waves of BRFSS. BRFSS sample}\\
\multicolumn{6}{l}{weights applied. Market volatility is defined as $\frac{log(VIX)}{100}$. All models estimated with a Logistic Regression. All models}\\
\multicolumn{6}{l}{control for demographics, including age, gender, race, education status, income, marital status, employment}\\
\multicolumn{6}{l}{status. See Appendix for detailed variable definition, data source, and construction.}\\
\multicolumn{6}{l}{\textit{$^{*}$p$<$0.1; $^{**}$p$<$0.05; $^{***}$p$<$0.01}} \\ 
\hline\hline

\end{longtable} 

\section{Conclusion}

This study explores the impact of financial and economic uncertainty on self-reported life satisfaction. Our results show clear patterns, similar to previous research, in which self-reported life satisfaction is worsened during periods of market turmoil and uncertainty. Using market volatility, or the VIX, we find evidence that self-reported life satisfaction is more likely to be reported in the category of "Very Satisfied" compared to the category of "Dissatisfied" when the implied volatility index declines, or when the market volatility indicates a relative decline in economic uncertainty. This study is novel in the sense that we employ expected volatility as our primary variable of interest instead of other mainstream stock market indicators, such as the S&P 500 or the Dow Jones Industrial Average. Thus, we assess the relationship between forward-looking volatility expectations and individual life satisfaction metrics.

Furthermore, since the future economic conditions are pertinent for both non-stockholders and stockholders, it is expected that human responses are widespread and not merely restricted to individuals actively participating in the market. In sum, with various tests and robustness checks, our results strongly support our hypothesis and confirm previous evidence on mainstream market indicators, such as the Dow Jones. Further, we fit our primary model into the Anticipatory Utility framework and show that the VIX daily series, acting as an anticipatory index, influences survey respondents' life satisfaction, which acts as our primary utility measure. This novel approach takes the financial market's association with life satisfaction and life satisfaction in a new direction. Prior research has extensively investigated the relationship of human behaviors on the stock market, but little work explores the inverse effects. We hope that this study can add to the behavioral economic and modern finance literature using that particular perspective.

Previous research has shown that the effect mechanism between financial markets and life satisfaction is derived from several possible factors. Earlier research (Brenner & Mooney, 1983; Catalano & Dooley, 1983) suggests that the level of stress due to market conditions may lead to self-medication. Risky health behaviors can also be the result of market downturns. Behaviors such as smoking, overeating, and binge drinking are more likely to occur when market performance is poorer (Colman & Dave, 2011; Cotti & Tefft, 2011; 23 Ruhm & Black, 2002; Ruhm, 2005). In addition, Cotti, Dunn, Tefft 2013 found a diminished income effect when assessing the impact of the Dow Jones on health, suggesting that market and economic stress play a role in one's inclination to participate in risky health behaviors. Therefore, our findings help explain why behavioral biases are more severe when expected market volatility is high (Kumar, 2009).

While the study's estimates are intensely investigated, there are several limitations to this study. Although we show a deep channel of how market volatility affects life satisfaction, we cannot conclude that this effect is causal. The income effect indeed plays a significant role in the negative relationship between market volatility and life satisfaction. Our partial effects show this to be the case. Nonetheless, we are not able to fully control for potential endogeneity issues. Individuals can be affected by market uncertainty in many ways, including market crashes, potential job loss, etc. Although we show that the VIX can act as an anticipatory index for the market uncertainty, we do not fully understand the link that connects market-implied financial indicators to human behaviors. For example, stock market participation has increased in recent years, capturing new demographics of individuals who can now more easily open trading accounts. Such increases in non-institutional trade activity may increase the presence of noise traders, which in turn may influence levels of volatility. We do not fully understand the impact of these changing market dynamics on our results. 

Nonetheless, the paper underscores the exciting cross-section between the fields of behavioral economics, finance, and health. Although a handful of previous literature inspires the study, we are unaware of existing research that is similar or identical to our study. Finally, by better understanding the impact of stock market behavior on human behavior and life satisfaction, this paper sheds additional light on the contemporaneous consequences of an individual's anticipated financial and economic uncertainty.

\section{Appendix}

In this analysis, we refer life satisfaction to $life satisfaction$. For demographic variables, given the nature of the BRFSS survey, all of them are categorical variables. $gender$ is recoded from "sex" in BRFSS from 2013-2017. The question asked, "What is your sex? or What was your sex at birth? Was it…" We recoded this variable with 1 being "Male", 0 being "Female" and got rid of "Don't know/Not Sure", and "Refused". Variable $marital\_status$ is recoded from section "Demographics", under label "Marital Status." This variable is a categorical variable with "Married", "Divorced", "Widowed", "Separated", "Never married", and "A member of an unmarried couple". We got rid of options "Refused", and missing data for this variable. Variable $employment\_status$ is also under "Demographics" section. The question asked "Question: Are you currently…?" and the answers ranging from "Employed for wages", "Self-employed", "Out of work for 1 year or more", "Out of work for less than 1 year", "A homemaker", "A student", "Retired", "Unable to work." Options "Refused" and "BLANK" (missing data) are eliminated from the analysis. Variable $calculated\_race$ is a calculated race variable with 6 categories "White", "Black or African American", "American Indian or Alaskan Native", "Asian", "Native Hawaiian or other Pacific Islander" and "Other race." I got rid of "No preferred race", "Don't know/Not sure", "Refused" and missing data. Variable $calculated\_education$ represents education status with 4 main categories "Did not graduate High School", "Graduated High School", "Attended College or Technical School" and "Graduated from College or Technical School." We also got rid of "Don't know/Not sure/Missing" value. Variable $calculated\_income$ represents income brackets from "Less than $\$15,000$", "$\$15,000$ to less than $\$25,000$", "$\$25,000$ to less than $\$35,000$", "$\$35,000$ to less than $\$50,000$", and "$\$50,000$ or more." Category "Don’t know/Not sure/Missing" is eliminated. Variable $calculated\_age$ represent calculated age variable with categories "Age 18 to 24", "Age 25 to 34", "Age 35 to 44", "Age 45 to 54", "Age 55 to 64", "Age 65 or older."  Missing data is also eliminated. 

The CPS survey spans from 2013 to 2017. Almost all variables used are continuous, and thus we have to recode them to be categorical. Stock ownership, $ownstock$, is recoded from the question, which asks, "A anytime during 20.., did you have shares of stock in corporations or mutual funds?" We recoded this variable with 1 being "Yes," 0 being "No," and got rid of "Not in universe." $gender$ is recoded with 1 being "Male," 0 being "Female." $calculated\_education$ is recoded from a question regarding "education attainment." In specific, "Did not graduate High School" which takes in "Less than 1st grade", "1st, 2nd, 3rd, or 4th grade", "5th or 6th grade", "7th and 8th grade", "9th grade", "10th grade", "11th grade" and "12th grade no diploma" in the CPS. "Graduated High School" takes "High school graduate - high school diploma or equivalent" in the CPS. "Attended College or Technical School" takes "Some college but no degree," "Associate degree in college - occupation/vocation program," and "Associate degree in college - academic program" in the CPS. "Graduated from College or Technical School" takes "Bachelor's degree (for example: BA, AB, BS)," "Master's degree (for example: MA, MS, MENG, MED, MSW, MBA)," "Professional school degree (for example: MD, DDS, DVM, LLB, JD)" and "Doctorate degree (for example: PHD, EDD)." Variable $calculated\_age$ is recoded from a continuous age variable from the CPS to fit in the age categories of the BRFSS, which are "Age 18 to 24", "Age 25 to 34", "Age 35 to 44", "Age 45 to 54", "Age 55 to 64", "Age 65 or older." Variable $calculated\_income$ is also recoded from a continuous income variable from the CPS to fit in the income categories in the BRFSS, which are "Less than \$15,000", "\$15,000 to less than \$25,000", "\$25,000 to less than \$35,000", "\$35,000 to less than \$50,000", and "\$50,000 or more." Variable $calculated\_race$ is a calculated race variable with 6 categories "White," "Black or African American," "American Indian or Alaskan Native," "Asian," "Native Hawaiian or other Pacific Islander," and "Other race."


\begin{footnotesize}
\def\sym#1{\ifmmode^{#1}\else\(^{#1}\)\fi}
\begin{longtable}{l*{4}{c}}
\caption{Regression Results for Life-Satisfaction and Daily VIX} \\
\hline\hline
& \multicolumn{4}{c}{ \textit{Dependent Variable}} \\
\cline{2-5}\\[-4ex]
& \multicolumn{4}{c}{Life Satisfaction}\\
\\[-4ex]
                    &\multicolumn{1}{c}{(1)}&\multicolumn{1}{c}{(2)}&\multicolumn{1}{c}{(3)}&\multicolumn{1}{c}{(4)}\\
                    &\multicolumn{1}{c}{Ordinal Logit}&\multicolumn{1}{c}{Ordinal Logit}&\multicolumn{1}{c}{Ordinal Logit}&\multicolumn{1}{c}{Ordinal Logit}\\
\hline
logVIX.Adjusted     &      -26.380\sym{***}&      -24.346\sym{***}&      -14.041\sym{***}&      -13.848\sym{***}\\
                    &     (1.019)         &     (1.036)         &     (1.627)         &     (1.646)         \\
 
psvalue.cps         &       0.308\sym{***}&       0.356\sym{***}&       1.092\sym{***}&       1.025\sym{***}\\
                    &     (0.035)         &     (0.036)         &     (0.088)         &     (0.089)         \\
logVIX.Adjusted $\times$ psvalue.cps&                     &                     &     -30.208\sym{***}&     -25.798\sym{***}\\
                    &                     &                     &     (3.104)         &     (3.143)         \\
                    
SPY.rets            &       0.000\sym{*}  &      -0.000\sym{**} &       0.000\sym{*}  &      -0.000\sym{**} \\
                    &     (0.000)         &     (0.000)         &     (0.000)         &     (0.000)         \\
 
health\_insured      &       0.541\sym{***}&       0.588\sym{***}&       0.540\sym{***}&       0.588\sym{***}\\
                    &     (0.006)         &     (0.006)         &     (0.006)         &     (0.006)         \\
 
gender1             &      -0.087\sym{***}&      -0.141\sym{***}&      -0.086\sym{***}&      -0.140\sym{***}\\
                    &     (0.003)         &     (0.004)         &     (0.003)         &     (0.004)         \\
 
HSgrad              &       0.033\sym{**} &       0.051\sym{***}&       0.031\sym{**} &       0.049\sym{***}\\
                    &     (0.011)         &     (0.011)         &     (0.011)         &     (0.011)         \\
 
K                   &       0.095\sym{***}&       0.188\sym{***}&       0.089\sym{***}&       0.183\sym{***}\\
                    &     (0.016)         &     (0.016)         &     (0.016)         &     (0.016)         \\
 
attendCOL           &      -0.086\sym{***}&      -0.056\sym{***}&      -0.088\sym{***}&      -0.057\sym{***}\\
                    &     (0.008)         &     (0.008)         &     (0.008)         &     (0.008)         \\
 
25to35K             &      -0.017\sym{***}&      -0.112\sym{***}&      -0.018\sym{***}&      -0.113\sym{***}\\
                    &     (0.005)         &     (0.005)         &     (0.005)         &     (0.005)         \\
 
35to50K             &       0.172\sym{***}&       0.069\sym{***}&       0.173\sym{***}&       0.069\sym{***}\\
                    &     (0.006)         &     (0.006)         &     (0.006)         &     (0.006)         \\
 
50more              &       0.543\sym{***}&       0.398\sym{***}&       0.544\sym{***}&       0.399\sym{***}\\
                    &     (0.010)         &     (0.010)         &     (0.010)         &     (0.010)         \\
 
le15K               &      -0.092\sym{***}&      -0.051\sym{***}&      -0.093\sym{***}&      -0.052\sym{***}\\
                    &     (0.005)         &     (0.005)         &     (0.005)         &     (0.005)         \\
 
25to34              &       0.151\sym{***}&       0.012         &       0.151\sym{***}&       0.012         \\
                    &     (0.016)         &     (0.016)         &     (0.016)         &     (0.016)         \\
 
35to44              &      -0.215\sym{***}&      -0.228\sym{***}&      -0.215\sym{***}&      -0.228\sym{***}\\
                    &     (0.016)         &     (0.016)         &     (0.016)         &     (0.016)         \\
 
45to54              &      -0.329\sym{***}&      -0.278\sym{***}&      -0.328\sym{***}&      -0.276\sym{***}\\
                    &     (0.017)         &     (0.017)         &     (0.017)         &     (0.017)         \\
 
55to64              &       0.012         &       0.055\sym{**} &       0.013         &       0.056\sym{**} \\
                    &     (0.018)         &     (0.018)         &     (0.018)         &     (0.018)         \\
 
65older             &       0.155\sym{***}&       0.117\sym{***}&       0.156\sym{***}&       0.118\sym{***}\\
                    &     (0.020)         &     (0.020)         &     (0.020)         &     (0.020)         \\
 
black               &       0.175\sym{***}&       0.324\sym{***}&       0.169\sym{***}&       0.318\sym{***}\\
                    &     (0.016)         &     (0.016)         &     (0.016)         &     (0.016)         \\
 
native              &      -0.646\sym{***}&      -0.279\sym{***}&      -0.649\sym{***}&      -0.281\sym{***}\\
                    &     (0.018)         &     (0.019)         &     (0.018)         &     (0.019)         \\
 
other               &       0.384\sym{***}&       0.519\sym{***}&       0.381\sym{***}&       0.516\sym{***}\\
                    &     (0.018)         &     (0.019)         &     (0.018)         &     (0.019)         \\
 
pacific             &       1.123\sym{***}&       1.622\sym{***}&       1.142\sym{***}&       1.641\sym{***}\\
                    &     (0.042)         &     (0.043)         &     (0.042)         &     (0.043)         \\
 
white               &      -0.219\sym{***}&       0.059\sym{***}&      -0.222\sym{***}&       0.057\sym{***}\\
                    &     (0.015)         &     (0.016)         &     (0.015)         &     (0.016)         \\
 
married             &       0.636\sym{***}&       0.618\sym{***}&       0.633\sym{***}&       0.615\sym{***}\\
                    &     (0.004)         &     (0.004)         &     (0.004)         &     (0.004)         \\
 
membermarriedcoup   &      -0.145\sym{***}&      -0.226\sym{***}&      -0.150\sym{***}&      -0.230\sym{***}\\
                    &     (0.012)         &     (0.012)         &     (0.012)         &     (0.012)         \\
 
nevermarried        &       0.156\sym{***}&       0.085\sym{***}&       0.153\sym{***}&       0.083\sym{***}\\
                    &     (0.006)         &     (0.006)         &     (0.006)         &     (0.006)         \\
 
separated           &      -0.181\sym{***}&      -0.036\sym{***}&      -0.185\sym{***}&      -0.038\sym{***}\\
                    &     (0.010)         &     (0.010)         &     (0.010)         &     (0.010)         \\
 
widowed             &       0.309\sym{***}&       0.243\sym{***}&       0.307\sym{***}&       0.241\sym{***}\\
                    &     (0.006)         &     (0.006)         &     (0.006)         &     (0.006)         \\
 
noworkless1         &      -1.195\sym{***}&      -1.137\sym{***}&      -1.194\sym{***}&      -1.135\sym{***}\\
                    &     (0.012)         &     (0.012)         &     (0.012)         &     (0.012)         \\
 
noworkmore1         &      -1.914\sym{***}&      -1.551\sym{***}&      -1.914\sym{***}&      -1.550\sym{***}\\
                    &     (0.012)         &     (0.012)         &     (0.012)         &     (0.012)         \\
 
retired             &      -0.457\sym{***}&      -0.265\sym{***}&      -0.457\sym{***}&      -0.264\sym{***}\\
                    &     (0.007)         &     (0.007)         &     (0.007)         &     (0.007)         \\
 
selfemployed        &      -0.202\sym{***}&      -0.128\sym{***}&      -0.201\sym{***}&      -0.126\sym{***}\\
                    &     (0.009)         &     (0.009)         &     (0.009)         &     (0.009)         \\
 
student             &       0.935\sym{***}&       0.907\sym{***}&       0.939\sym{***}&       0.911\sym{***}\\
                    &     (0.021)         &     (0.022)         &     (0.021)         &     (0.022)         \\
 
unable              &      -1.394\sym{***}&      -0.879\sym{***}&      -1.395\sym{***}&      -0.880\sym{***}\\
                    &     (0.007)         &     (0.008)         &     (0.007)         &     (0.008)         \\
 
wagesemployed       &      -0.532\sym{***}&      -0.477\sym{***}&      -0.533\sym{***}&      -0.477\sym{***}\\
                    &     (0.007)         &     (0.007)         &     (0.007)         &     (0.007)         \\
cancer              &                     &       0.021\sym{***}&                     &       0.022\sym{***}\\
                    &                     &     (0.004)         &                     &     (0.004)         \\
 
heart\_disease       &                     &      -0.000         &                     &      -0.000         \\
                    &                     &     (0.004)         &                     &     (0.004)         \\
 
arthritis           &                     &      -0.254\sym{***}&                     &      -0.254\sym{***}\\
                    &                     &     (0.003)         &                     &     (0.003)         \\
 
diabetes            &                     &      -0.422\sym{***}&                     &      -0.422\sym{***}\\
                    &                     &     (0.006)         &                     &     (0.006)         \\
 
stroke              &                     &      -0.110\sym{***}&                     &      -0.110\sym{***}\\
                    &                     &     (0.005)         &                     &     (0.005)         \\
 
asthma              &                     &      -0.178\sym{***}&                     &      -0.179\sym{***}\\
                    &                     &     (0.004)         &                     &     (0.004)         \\
 
bronchitis          &                     &      -0.353\sym{***}&                     &      -0.353\sym{***}\\
                    &                     &     (0.005)         &                     &     (0.005)         \\
 
depression          &                     &      -0.936\sym{***}&                     &      -0.936\sym{***}\\
                    &                     &     (0.004)         &                     &     (0.004)         \\
\hline
logit[P(Y $\leq$ 1)]                &                     &                     &                     &                     \\
Constant            &      -3.942\sym{***}&      -4.531\sym{***}&      -3.627\sym{***}&      -4.263\sym{***}\\
                    &     (0.037)         &     (0.038)         &     (0.049)         &     (0.050)         \\
\hline
logit[P(Y $\leq$ 2)]                &                     &                     &                     &                     \\
Constant            &      -0.651\sym{***}&      -1.081\sym{***}&      -0.336\sym{***}&      -0.813\sym{***}\\
                    &     (0.037)         &     (0.038)         &     (0.049)         &     (0.050)         \\

\hline \\[-1.8ex] 
Month FE  & Yes & Yes & Yes & Yes \\ 
State FE  & Yes & Yes & Yes & Yes \\ 
Year FE  & Yes & Yes & Yes & Yes \\ 
Survey Weights & No & No & Yes & Yes\\ 
Observations  & 7,516 & 7,516 & 7,516 & 7,516 \\ 
\hline \\[-1.8ex] 
\multicolumn{5}{l}{Note: Panel consists of the 2013-2017 survey sample waves of BRFSS. BRFSS sample weights applied.}\\
\multicolumn{5}{l}{Market volatility is defined as $\frac{log(VIX)}{100}$. All models estimated with Logistic Regression. All models}\\
\multicolumn{5}{l}{control for demographics, including age, gender, race, education status, income, marital status,}\\
\multicolumn{5}{l}{employment status. See Appendix for detailed variable definition, data  source, and construction.}\\
\multicolumn{5}{l}{\textit{$^{*}$p$<$0.1; $^{**}$p$<$0.05; $^{***}$p$<$0.01}} \\ 
\hline\hline
\end{longtable}
\end{footnotesize}


\begin{tiny}
\def\sym#1{\ifmmode^{#1}\else\(^{#1}\)\fi}
\begin{longtable}{l*{5}{c}}
\caption{Regression Results for Life-Satisfaction and Daily VIX} \\
\hline\hline
& \multicolumn{5}{c}{ \textit{Dependent Variable}} \\
\cline{2-6}\\[-4ex]
& \multicolumn{5}{c}{Life Satisfaction}\\
\\[-4ex]
                    &\multicolumn{1}{c}{(1)}&\multicolumn{1}{c}{(2)}&\multicolumn{1}{c}{(3)}&\multicolumn{1}{c}{(4)}&\multicolumn{1}{c}{(5)}\\
                    &\multicolumn{1}{c}{Ordinal Logit}&\multicolumn{1}{c}{Ordinal Logit}&\multicolumn{1}{c}{Ordinal Logit}&\multicolumn{1}{c}{Ordinal Logit}&\multicolumn{1}{c}{Ordinal Logit}\\
\hline
loglag1vixadjusted100&      -0.013\sym{**} &                     &                     &                     &                     \\
                    &     (0.005)         &                     &                     &                     &                     \\
 
psvalue.cps         &       1.021\sym{***}&       1.003\sym{***}&       1.305\sym{***}&       1.189\sym{***}&       1.122\sym{***}\\
                    &     (0.093)         &     (0.093)         &     (0.093)         &     (0.093)         &     (0.093)         \\
 
loglag1vixadjusted100 $\times$ psvalue.cps&      -0.013         &                     &                     &                     &                     \\
                    &     (0.008)         &                     &                     &                     &                     \\
 
lag1.SPY.rets       &      -0.000         &                     &                     &                     &                     \\
                    &     (0.000)         &                     &                     &                     &                     \\

 
loglag2vixadjusted100&                     &      -0.045\sym{***}&                     &                     &                     \\
                    &                     &     (0.005)         &                     &                     &                     \\
 
loglag2vixadjusted100 $\times$ psvalue.cps&                     &      -0.000         &                     &                     &                     \\
                    &                     &     (0.009)         &                     &                     &                     \\
 
lag2.SPY.rets       &                     &      -0.000\sym{***}&                     &                     &                     \\
                    &                     &     (0.000)         &                     &                     &                     \\
 
loglag3vixadjusted100&                     &                     &      -0.014\sym{**} &                     &                     \\
                    &                     &                     &     (0.005)         &                     &                     \\
 
loglag3vixadjusted100 $\times$ psvalue.cps&                     &                     &      -0.105\sym{***}&                     &                     \\
                    &                     &                     &     (0.008)         &                     &                     \\
 
lag3.SPY.rets       &                     &                     &      -0.000\sym{***}&                     &                     \\
                    &                     &                     &     (0.000)         &                     &                     \\
 
loglag4vixadjusted100&                     &                     &                     &      -0.028\sym{***}&                     \\
                    &                     &                     &                     &     (0.005)         &                     \\
 
loglag4vixadjusted100 $\times$ psvalue.cps&                     &                     &                     &      -0.071\sym{***}&                     \\
                    &                     &                     &                     &     (0.008)         &                     \\
 
lag4.SPY.rets       &                     &                     &                     &       0.000\sym{***}&                     \\
                    &                     &                     &                     &     (0.000)         &                     \\
 
logweeklagvixadjusted100&                     &                     &                     &                     &      -0.018\sym{***}\\
                    &                     &                     &                     &                     &     (0.005)         \\
 
logweeklagvixadjusted100 $\times$ psvalue.cps&                     &                     &                     &                     &      -0.046\sym{***}\\
                    &                     &                     &                     &                     &     (0.008)         \\
 
weeklag.SPY.rets    &                     &                     &                     &                     &       0.000\sym{***}\\
                    &                     &                     &                     &                     &     (0.000)         \\

cancer              &       0.027\sym{**} &       0.029\sym{**} &       0.032\sym{**} &       0.026\sym{**} &       0.027\sym{**} \\
                    &     (0.010)         &     (0.010)         &     (0.010)         &     (0.010)         &     (0.010)         \\
 
health\_insured      &       0.577\sym{***}&       0.577\sym{***}&       0.577\sym{***}&       0.578\sym{***}&       0.580\sym{***}\\
                    &     (0.014)         &     (0.014)         &     (0.014)         &     (0.014)         &     (0.014)         \\
 
heart\_disease       &      -0.155\sym{***}&      -0.159\sym{***}&      -0.160\sym{***}&      -0.158\sym{***}&      -0.157\sym{***}\\
                    &     (0.010)         &     (0.010)         &     (0.010)         &     (0.010)         &     (0.010)         \\
 
arthritis           &      -0.227\sym{***}&      -0.228\sym{***}&      -0.227\sym{***}&      -0.226\sym{***}&      -0.227\sym{***}\\
                    &     (0.007)         &     (0.007)         &     (0.007)         &     (0.007)         &     (0.007)         \\
 
diabetes            &      -0.310\sym{***}&      -0.309\sym{***}&      -0.310\sym{***}&      -0.305\sym{***}&      -0.309\sym{***}\\
                    &     (0.013)         &     (0.013)         &     (0.013)         &     (0.013)         &     (0.013)         \\
 
stroke              &      -0.302\sym{***}&      -0.303\sym{***}&      -0.304\sym{***}&      -0.304\sym{***}&      -0.305\sym{***}\\
                    &     (0.012)         &     (0.012)         &     (0.012)         &     (0.012)         &     (0.012)         \\
 
asthma              &      -0.112\sym{***}&      -0.110\sym{***}&      -0.107\sym{***}&      -0.105\sym{***}&      -0.111\sym{***}\\
                    &     (0.009)         &     (0.009)         &     (0.009)         &     (0.009)         &     (0.009)         \\
 
bronchitis          &      -0.217\sym{***}&      -0.214\sym{***}&      -0.218\sym{***}&      -0.221\sym{***}&      -0.219\sym{***}\\
                    &     (0.011)         &     (0.011)         &     (0.011)         &     (0.011)         &     (0.011)         \\
 
depression          &      -1.045\sym{***}&      -1.044\sym{***}&      -1.044\sym{***}&      -1.044\sym{***}&      -1.043\sym{***}\\
                    &     (0.008)         &     (0.008)         &     (0.008)         &     (0.008)         &     (0.008)         \\
 
gender1             &      -0.209\sym{***}&      -0.211\sym{***}&      -0.209\sym{***}&      -0.209\sym{***}&      -0.210\sym{***}\\
                    &     (0.008)         &     (0.008)         &     (0.008)         &     (0.008)         &     (0.008)         \\
 
HSgrad              &       0.265\sym{***}&       0.274\sym{***}&       0.271\sym{***}&       0.269\sym{***}&       0.270\sym{***}\\
                    &     (0.028)         &     (0.028)         &     (0.028)         &     (0.028)         &     (0.028)         \\
 
K                   &       0.465\sym{***}&       0.477\sym{***}&       0.466\sym{***}&       0.465\sym{***}&       0.465\sym{***}\\
                    &     (0.040)         &     (0.040)         &     (0.040)         &     (0.040)         &     (0.040)         \\
 
attendCOL           &       0.067\sym{***}&       0.072\sym{***}&       0.070\sym{***}&       0.068\sym{***}&       0.072\sym{***}\\
                    &     (0.020)         &     (0.020)         &     (0.020)         &     (0.020)         &     (0.020)         \\
 
25to35K             &       0.120\sym{***}&       0.121\sym{***}&       0.116\sym{***}&       0.120\sym{***}&       0.117\sym{***}\\
                    &     (0.013)         &     (0.013)         &     (0.013)         &     (0.013)         &     (0.013)         \\
 
35to50K             &       0.283\sym{***}&       0.285\sym{***}&       0.287\sym{***}&       0.293\sym{***}&       0.284\sym{***}\\
                    &     (0.015)         &     (0.015)         &     (0.015)         &     (0.015)         &     (0.015)         \\
 
50more              &       0.444\sym{***}&       0.442\sym{***}&       0.444\sym{***}&       0.452\sym{***}&       0.442\sym{***}\\
                    &     (0.025)         &     (0.025)         &     (0.025)         &     (0.025)         &     (0.025)         \\
 
le15K               &       0.213\sym{***}&       0.215\sym{***}&       0.210\sym{***}&       0.212\sym{***}&       0.210\sym{***}\\
                    &     (0.012)         &     (0.012)         &     (0.012)         &     (0.012)         &     (0.012)         \\

25to34              &      -0.047         &      -0.077         &      -0.056         &      -0.041         &      -0.049         \\
                    &     (0.040)         &     (0.040)         &     (0.041)         &     (0.040)         &     (0.040)         \\
 
35to44              &      -0.302\sym{***}&      -0.332\sym{***}&      -0.309\sym{***}&      -0.293\sym{***}&      -0.303\sym{***}\\
                    &     (0.041)         &     (0.041)         &     (0.041)         &     (0.041)         &     (0.041)         \\
 
45to54              &      -0.455\sym{***}&      -0.488\sym{***}&      -0.464\sym{***}&      -0.448\sym{***}&      -0.456\sym{***}\\
                    &     (0.043)         &     (0.043)         &     (0.043)         &     (0.043)         &     (0.043)         \\
 
55to64              &      -0.198\sym{***}&      -0.230\sym{***}&      -0.206\sym{***}&      -0.191\sym{***}&      -0.199\sym{***}\\
                    &     (0.046)         &     (0.046)         &     (0.046)         &     (0.046)         &     (0.046)         \\
 
65older             &      -0.061         &      -0.092         &      -0.072         &      -0.055         &      -0.061         \\
                    &     (0.050)         &     (0.050)         &     (0.050)         &     (0.050)         &     (0.050)         \\

black               &       0.882\sym{***}&       0.882\sym{***}&       0.888\sym{***}&       0.902\sym{***}&       0.888\sym{***}\\
                    &     (0.043)         &     (0.043)         &     (0.043)         &     (0.043)         &     (0.043)         \\
 
native              &       0.537\sym{***}&       0.514\sym{***}&       0.536\sym{***}&       0.549\sym{***}&       0.541\sym{***}\\
                    &     (0.047)         &     (0.047)         &     (0.047)         &     (0.048)         &     (0.047)         \\
 
other               &       0.789\sym{***}&       0.774\sym{***}&       0.797\sym{***}&       0.802\sym{***}&       0.796\sym{***}\\
                    &     (0.048)         &     (0.048)         &     (0.048)         &     (0.048)         &     (0.048)         \\
 
pacific             &       1.203\sym{***}&       1.169\sym{***}&       1.251\sym{***}&       1.241\sym{***}&       1.225\sym{***}\\
                    &     (0.125)         &     (0.125)         &     (0.125)         &     (0.125)         &     (0.125)         \\
 
white               &       0.422\sym{***}&       0.414\sym{***}&       0.430\sym{***}&       0.442\sym{***}&       0.429\sym{***}\\
                    &     (0.042)         &     (0.042)         &     (0.042)         &     (0.042)         &     (0.042)         \\
 
married             &       0.665\sym{***}&       0.667\sym{***}&       0.660\sym{***}&       0.659\sym{***}&       0.666\sym{***}\\
                    &     (0.010)         &     (0.010)         &     (0.010)         &     (0.010)         &     (0.010)         \\
 
membermarriedcoup   &       0.003         &       0.001         &      -0.008         &      -0.006         &       0.001         \\
                    &     (0.030)         &     (0.030)         &     (0.030)         &     (0.030)         &     (0.030)         \\
 
nevermarried        &       0.079\sym{***}&       0.075\sym{***}&       0.071\sym{***}&       0.071\sym{***}&       0.078\sym{***}\\
                    &     (0.013)         &     (0.013)         &     (0.013)         &     (0.013)         &     (0.013)         \\
 
separated           &       0.014         &       0.012         &       0.007         &      -0.000         &       0.008         \\
                    &     (0.023)         &     (0.023)         &     (0.023)         &     (0.023)         &     (0.023)         \\
 
widowed             &       0.329\sym{***}&       0.331\sym{***}&       0.329\sym{***}&       0.329\sym{***}&       0.329\sym{***}\\
                    &     (0.012)         &     (0.012)         &     (0.012)         &     (0.012)         &     (0.012)         \\

noworkless1         &      -1.167\sym{***}&      -1.166\sym{***}&      -1.174\sym{***}&      -1.178\sym{***}&      -1.166\sym{***}\\
                    &     (0.030)         &     (0.030)         &     (0.030)         &     (0.030)         &     (0.030)         \\
 
noworkmore1         &      -1.530\sym{***}&      -1.526\sym{***}&      -1.531\sym{***}&      -1.538\sym{***}&      -1.528\sym{***}\\
                    &     (0.029)         &     (0.029)         &     (0.029)         &     (0.029)         &     (0.029)         \\
 
retired             &      -0.408\sym{***}&      -0.405\sym{***}&      -0.403\sym{***}&      -0.408\sym{***}&      -0.399\sym{***}\\
                    &     (0.019)         &     (0.019)         &     (0.019)         &     (0.019)         &     (0.019)         \\
 
selfemployed        &      -0.087\sym{***}&      -0.081\sym{***}&      -0.080\sym{***}&      -0.080\sym{***}&      -0.080\sym{***}\\
                    &     (0.022)         &     (0.022)         &     (0.022)         &     (0.022)         &     (0.022)         \\
 
student             &       0.668\sym{***}&       0.661\sym{***}&       0.663\sym{***}&       0.662\sym{***}&       0.691\sym{***}\\
                    &     (0.057)         &     (0.057)         &     (0.057)         &     (0.057)         &     (0.057)         \\
 
unable              &      -0.954\sym{***}&      -0.949\sym{***}&      -0.954\sym{***}&      -0.962\sym{***}&      -0.946\sym{***}\\
                    &     (0.020)         &     (0.020)         &     (0.020)         &     (0.020)         &     (0.020)         \\
 
wagesemployed       &      -0.543\sym{***}&      -0.538\sym{***}&      -0.544\sym{***}&      -0.545\sym{***}&      -0.536\sym{***}\\
                    &     (0.018)         &     (0.018)         &     (0.018)         &     (0.018)         &     (0.018)         \\
\hline
logit[P(Y $\leq$ 1)]                &                     &                     &                     &                     &                     \\
Constant            &      -3.143\sym{***}&      -3.336\sym{***}&      -3.189\sym{***}&      -3.144\sym{***}&      -3.099\sym{***}\\
                    &     (0.064)         &     (0.064)         &     (0.064)         &     (0.064)         &     (0.064)         \\
\hline
logit[P(Y $\leq$ 2)]                &                     &                     &                     &                     &                     \\
Constant            &       0.305\sym{***}&       0.114         &       0.264\sym{***}&       0.308\sym{***}&       0.351\sym{***}\\
                    &     (0.064)         &     (0.064)         &     (0.064)         &     (0.064)         &     (0.064)         \\
                   &                     &     (0.004)         &                     &     (0.004)         \\
\hline \\[-1.8ex] 
Month FE  & Yes & Yes & Yes & Yes & Yes \\ 
State FE  & Yes & Yes & Yes & Yes & Yes \\ 
Year FE  & Yes & Yes & Yes & Yes & Yes \\ 
Survey Weights & No & No & Yes & Yes & Yes\\ 
Observations  & 7,516 & 7,516 & 7,516 & 7,516 & 7,516 \\ 
\hline \\[-1.8ex] 
\multicolumn{6}{l}{Note: Panel consists of the 2013-2017 survey sample waves of BRFSS. BRFSS sample weights applied. Market volatility is defined}\\
\multicolumn{6}{l}{as $\frac{log(VIX)}{100}$. All models estimated with a Logistic Regression. All models control for demographics, including age, gender, race,}\\
\multicolumn{6}{l}{education status, income, marital status,employment status. See Appendix for detailed variable definition, data  source, and}\\
\multicolumn{6}{l}{construction.}\\
\multicolumn{6}{l}{ \textit{$^{*}$p$<$0.1; $^{**}$p$<$0.05; $^{***}$p$<$0.01}} \\ 
\hline\hline
\end{longtable}
\end{tiny} 


\begin{thebibliography}{39}

\bibitem{Apouey, Benedicte, and Clark 2015}
Apouey, Benedicte, and Andrew E. Clark. “Winning Big but Feeling No Better? The Effect of Lottery Prizes on Physical and Mental Health.” Health Economics 24, no. 5 (2015): 516–38.

\bibitem{Baker, Michael, Mark Stabile, and Deri 2004}
Baker, Michael, Mark Stabile, and Catherine Deri. “What Do Self-Reported, Objective, Measures of Health Measure?” Journal of Human Resources XXXIX, no. 4 (October 2, 2004): 1067–93.

\bibitem{Benjamin et al. 2004}
Benjamin, Emelia J., Martin G. Larson, Michelle J. Keyes, Gary F. Mitchell, Ramachandran S. Vasan, John F. Keaney, Birgitta T. Lehman, Shuxia Fan, Ewa Osypiuk, and Joseph A. Vita. “Clinical Correlates and Heritability of Flow-Mediated Dilation in the Community: The Framingham Heart Study.” Circulation 109, no. 5 (February 10, 2004): 613–19.

\bibitem{Böckerman and Ilmakunnas 2009}
Böckerman, Petri, and Pekka Ilmakunnas. “Unemployment and Self-Assessed Health: Evidence from Panel Data.” Health Economics 18, no. 2 (2009): 161–79. 

\bibitem{}
Brenner, M. Harvey, and Anne Mooney. “Unemployment and Health in the Context of Economic Change.” Social Science & Medicine 17, no. 16 (1983): 1125–38. 

\bibitem{}
Carleton, R. Nicholas. “The Intolerance of Uncertainty Construct in the Context of Anxiety Disorders: Theoretical and Practical Perspectives.” Expert Review of Neurotherapeutics 12, no. 8 (August 1, 2012): 937–47. 

\bibitem{}
Catalano, Ralph A., and David Dooley. “Health Effects of Economic Instability: A Test of Economic Stress Hypothesis.” Journal of Health and Social Behavior 24, no. 1 (1983): 46–60. 

\bibitem{}
Cotti, Chad, Richard A. Dunn, and Nathan Tefft. “The Dow Is Killing Me: Risky Health Behaviors and the Stock Market.” Health Economics 24, no. 7 (2015): 803–21.

\bibitem{}
Cotti, Chad, and Nathan Tefft. “Decomposing the Relationship between Macroeconomic Conditions and Fatal Car Crashes during the Great Recession: Alcohol- and Non-Alcohol-Related Accidents in: The B.E. Journal of Economic Analysis & Policy Volume 11 Issue 1 (2011).” Accessed March 30, 2020. 

\bibitem{}
Cottini, Elena, and Claudio Lucifora. “Mental Health and Working Conditions in Europe.” ILR Review 66, no. 4 (July 1, 2013): 958–88. 

\bibitem{}
Dávalos, María E., Hai Fang, and Michael T. French. “Easing the Pain of an Economic Downturn: Macroeconomic Conditions and Excessive Alcohol Consumption.” Health Economics 21, no. 11 (November 2012): 1318–35. 

\bibitem{}
Dave, Dhaval M, Jennifer Tennant, and Gregory J Colman. “Isolating the Effect of Major Depression on Obesity: Role of Selection Bias.” Working Paper. Working Paper Series. National Bureau of Economic Research, May 2011. 

\bibitem{}
Deaton, Angus S. “The Financial Crisis and the life satisfaction of Americans.” Working Paper. Working Paper Series. National Bureau of Economic Research, June 2011. 

\bibitem{}
Engelberg, Joseph, and Christopher A. Parsons. “Worrying about the Stock Market: Evidence from Hospital Admissions.” The Journal of Finance 71, no. 3 (2016): 1227–50. 

\bibitem{}
Fiuzat, Mona, Linda K. Shaw, Laine Thomas, G. Michael Felker, and Christopher M. O’Connor. “United States Stock Market Performance and Acute Myocardial Infarction Rates in 2008-2009 (from the Duke Databank for Cardiovascular Disease).” The American Journal of Cardiology 106, no. 11 (December 1, 2010): 1545–49.

\bibitem{}
Goidel, Kirby, Stephen Procopio, Dek Terrell, and H. Denis Wu. “Sources of Economic News and Economic Expectations.” American Politics Research 38, no. 4 (July 1, 2010): 759–77. 

\bibitem{}
Gravelle, Hugh, and Matt Sutton. “Income, Relative Income, and Self-Reported Health in Britain 1979–2000.” Health Economics 18, no. 2 (2009): 125–45. 

\bibitem{}
Horn, Brady P., Johanna Catherine Maclean, and Michael R. Strain. “Do Minimum Wage Increases Influence Worker Health?” Economic Inquiry 55, no. 4 (2017): 1986–2007.

\bibitem{}
Jylhä, Marja. “What Is Self-Rated Health and Why Does It Predict Mortality? Towards a Unified Conceptual Model.” Social Science & Medicine (1982) 69, no. 3 (August 2009): 307–16. 

\bibitem{}
Kahneman, Daniel, and Amos Tversky. “Prospect Theory: An Analysis of Decision under Risk.” Econometrica 47, no. 2 (1979): 263–91. 

\bibitem{}
Kumar, Alok. “Who Gambles in the Stock Market? - KUMAR - 2009 - The Journal of Finance - Wiley Online Library,” 2009. 

\bibitem{}
Lenhart, Amanda, Aaron Smith, Monica Anderson, Maeve Duggan, and Andrew Perrin. “Teens, Technology and Friendships,” 2015. 

\bibitem{}
Lindeboom, Maarten, and Eddy van Doorslaer. “Cut-Point Shift and Index Shift in Self-Reported Health.” Journal of Health Economics 23, no. 6 (November 1, 2004): 1083–99. 

\bibitem{}
MacLean, Paul S., Rena R. Wing, Terry Davidson, Leonard Epstein, Bret Goodpaster, Kevin D. Hall, Barry E. Levin, et al. “NIH Working Group Report: Innovative Research to Improve Maintenance of Weight Loss.” Obesity (Silver Spring, Md.) 23, no. 1 (January 2015): 7–15. 

\bibitem{}
MacLean, Richard. “Organizational Design: Benchmarking.” Environmental Quality Management 22, no. 3 (2013): 95–108. 

\bibitem{}
McInerney, Melissa, Jennifer M. Mellor, and Lauren Hersch Nicholas. “Recession Depression: Mental Health Effects of the 2008 Stock Market Crash.” CESifo Working Paper Series. CESifo Working Paper Series. CESifo Group Munich, 2013.

\bibitem{}
Miilunpalo, S., I. Vuori, P. Oja, M. Pasanen, and H. Urponen. “Self-Rated Health Status as a Health Measure: The Predictive Value of Self-Reported Health Status on the Use of Physician Services and on Mortality in the Working-Age Population.” Journal of Clinical Epidemiology 50, no. 5 (May 1997): 517–28. 

\bibitem{}
Ruhm, Christopher, and William E. Black. “Does Drinking Really Decrease in Bad Times?” Journal of Health Economics 21, no. 4 (2002): 659–78.

\bibitem{}
Ruhm, Christopher J. “Healthy Living in Hard Times.” Journal of Health Economics 24, no. 2 (March 2005): 341–63. 

\bibitem{}
Vilares, Iris, James D. Howard, Hugo L. Fernandes, Jay A. Gottfried, and Konrad P. Kording. “Differential Representations of Prior and Likelihood Uncertainty in the Human Brain.” Current Biology: CB 22, no. 18 (September 25, 2012): 1641–48.

\bibitem{}
Vilares, Iris, and Konrad Kording. “Bayesian Models: The Structure of the World, Uncertainty, Behavior, and the Brain.” Annals of the New York Academy of Sciences 1224, no. 1 (April 2011): 22–39. 


\end{document}
